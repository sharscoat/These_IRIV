\documentclass[12pt,]{article}
\usepackage{lmodern}
\usepackage{amssymb,amsmath}
\usepackage{ifxetex,ifluatex}
\usepackage{fixltx2e} % provides \textsubscript
\ifnum 0\ifxetex 1\fi\ifluatex 1\fi=0 % if pdftex
  \usepackage[T1]{fontenc}
  \usepackage[utf8]{inputenc}
\else % if luatex or xelatex
  \ifxetex
    \usepackage{mathspec}
  \else
    \usepackage{fontspec}
  \fi
  \defaultfontfeatures{Ligatures=TeX,Scale=MatchLowercase}
\fi
% use upquote if available, for straight quotes in verbatim environments
\IfFileExists{upquote.sty}{\usepackage{upquote}}{}
% use microtype if available
\IfFileExists{microtype.sty}{%
\usepackage{microtype}
\UseMicrotypeSet[protrusion]{basicmath} % disable protrusion for tt fonts
}{}
\usepackage[margin=1in]{geometry}
\usepackage{hyperref}
\hypersetup{unicode=true,
            pdftitle={Bibliographie\_These\_IRIV},
            pdfauthor={Sébastien HARSCOAT},
            pdfborder={0 0 0},
            breaklinks=true}
\urlstyle{same}  % don't use monospace font for urls
\usepackage{longtable,booktabs}
\usepackage{graphicx,grffile}
\makeatletter
\def\maxwidth{\ifdim\Gin@nat@width>\linewidth\linewidth\else\Gin@nat@width\fi}
\def\maxheight{\ifdim\Gin@nat@height>\textheight\textheight\else\Gin@nat@height\fi}
\makeatother
% Scale images if necessary, so that they will not overflow the page
% margins by default, and it is still possible to overwrite the defaults
% using explicit options in \includegraphics[width, height, ...]{}
\setkeys{Gin}{width=\maxwidth,height=\maxheight,keepaspectratio}
\IfFileExists{parskip.sty}{%
\usepackage{parskip}
}{% else
\setlength{\parindent}{0pt}
\setlength{\parskip}{6pt plus 2pt minus 1pt}
}
\setlength{\emergencystretch}{3em}  % prevent overfull lines
\providecommand{\tightlist}{%
  \setlength{\itemsep}{0pt}\setlength{\parskip}{0pt}}
\setcounter{secnumdepth}{5}
% Redefines (sub)paragraphs to behave more like sections
\ifx\paragraph\undefined\else
\let\oldparagraph\paragraph
\renewcommand{\paragraph}[1]{\oldparagraph{#1}\mbox{}}
\fi
\ifx\subparagraph\undefined\else
\let\oldsubparagraph\subparagraph
\renewcommand{\subparagraph}[1]{\oldsubparagraph{#1}\mbox{}}
\fi

%%% Use protect on footnotes to avoid problems with footnotes in titles
\let\rmarkdownfootnote\footnote%
\def\footnote{\protect\rmarkdownfootnote}

%%% Change title format to be more compact
\usepackage{titling}

% Create subtitle command for use in maketitle
\newcommand{\subtitle}[1]{
  \posttitle{
    \begin{center}\large#1\end{center}
    }
}

\setlength{\droptitle}{-2em}
  \title{Bibliographie\_These\_IRIV}
  \pretitle{\vspace{\droptitle}\centering\huge}
  \posttitle{\par}
  \author{Sébastien HARSCOAT}
  \preauthor{\centering\large\emph}
  \postauthor{\par}
  \predate{\centering\large\emph}
  \postdate{\par}
  \date{5 février 2018}

\usepackage[french]{babel}
\setlength\parindent{24pt}\setlength{\parskip}{0.0pt plus 1.0pt}
\usepackage{graphicx}
\usepackage{float}

\begin{document}
\maketitle

{
\setcounter{tocdepth}{2}
\tableofcontents
}
\pagebreak

\listoffigures

\pagebreak

\listoftables

\pagebreak

\section{Introduction}\label{introduction}

Les pathologies respiratoires sont un motif fréquent consultation aux
urgences et nécessitent, dans la majorité des cas, une hospitalisation
dans un service pour la suite de la prise en charge. Une part importante
de ces patients sont atteints de pathologies pulmonaires atteignant les
voies de conduction respiratoire. La principale caractéristique de ces
pathologies est de présenter un syndrome obstructif par augmentation des
résistances des bronchioles distales. Les deux pathologies les plus
connues de cette famille sont l'asthme et la Bronco-Pneumopathie
Chronique Obstructive (BPCO) associée ou non à un emphysème. Ces deux
pathologies sont responsables d'une morbi-mortalité importante avec pour
conséquences de nombreuses hospitalisations pour décompensation aiguë.
Actuellement, il n'existe que deux moyens non invasifs de surveiller de
façon continue une pathologie pulmonaire. Il s'agit de la saturation
pulsée en oxygène (SpO2) et de la capnographie par lunettes nasales. La
capnographie est un monitoring en temps réel du CO2 expiré, disponible
et facilement réalisable dans les services d'urgences. Des modifications
de la courbe de capnographie sont décrites dans la littérature en
association avec l'atteinte obstructive de certaines pathologies comme
l'asthme en rapport avec l'augmentation des résistances respiratoires.
Cette approche descriptive ne permet pas d'objectiver quantitativement
les troubles respiratoires sous-jacents. Pour le moment, aucun système
ne permet de quantifier les variations et les modifications de ces
courbes de CO2 expiré. A ce jour, les seuls examens permettant
d'apprécier les résistances des voies aériennes et par conséquent leurs
modifications, sont les explorations fonctionnelles respiratoires (EFR)
avec l'étude des volumes et des débits respiratoires, notamment lors de
la spirométrie. Les EFR sont considérés comme étant les examens de
référence dans la mesure précise et objective d'une obstruction
bronchique surtout chez le sujet asthmatique et BPCO. La réalisation
d'EFR comporte des contraintes matérielles et physiques qui rendent son
utilisation difficile dans un contexte de pathologie aiguë chez un
patient instable sur le plan respiratoire ou cardiologique et
nécessitant souvent une oxygénothérapie. Par ailleurs, il s'agit souvent
de patients âgés qui n'ont plus la capacité de se soumettre à ce type
d'explorations, qui demandent de la coopération et la mobilisation de
grands volumes pulmonaires. Dans ce contexte, l'utilisation de la
capnographie permettrait une évaluation et une surveillance plus aisée
de la fonctionnalité respiratoire. Peu d'études montrent l'existence
d'une relation quantifiable entre les courbes de capnographie et les
résultats des EFR. L'objectif de ce travail est d'objectiver l'apport de
la transformée en ondelettes sur l'analyse de la déformation de la
courbe de capnographie permettant la distiction des patients
pathologiques présentant un syndrome obstructif. Dans ce but, nous
établiront dans un premier temps l'état des connaissances sur la
technique de capnographie et particulièrement sur l'analyse de la courbe
ainsi que de ses tentatives de modélisation. Puis nous développeront
l'analyse du signal de capnographie par la transformée en ondelettes
tout d'abord par une phase de simulation permettant d'observer le
comportement de cette transformation en fonction de la déformation de la
courbe. Nous terminerons par une phase expérimentale à l'aide de données
de capnographie et de spirométrie recueillies sur des patients.

\pagebreak

\begin{center}\rule{0.5\linewidth}{\linethickness}\end{center}

\section{Première Partie - Pathologie
Obstructive}\label{premiere-partie---pathologie-obstructive}

\subsection{Rappel d'Anatomie et de Physiologie
Respiratoire}\label{rappel-danatomie-et-de-physiologie-respiratoire}

\subsection{Maladie Obstructive}\label{maladie-obstructive}

\pagebreak

\begin{center}\rule{0.5\linewidth}{\linethickness}\end{center}

\section{Deuxième Partie - Modèle de
Poumon}\label{deuxieme-partie---modele-de-poumon}

\pagebreak

\begin{center}\rule{0.5\linewidth}{\linethickness}\end{center}

\section{Troisième Partie -
Capnographie}\label{troisieme-partie---capnographie}

\subsection{Histoire de la
capnographie}\label{histoire-de-la-capnographie}

En 1620, le chimiste belge Van Helmont décrit pour la première fois la
molécule CO2. En 1754, le Dr Black publie dans sa thèse que le CO2 est
expiré dans l'air. Le CO2 est considéré comme toxique et mortel jusqu'en
1824. Il faudra attendre l'américain Henderson en 1925 pour démontrer
l'intérêt physiologique du dioxyde de carbone dans le métabolisme. En
1928, le CO2 est associé aux narcotiniques dans les anesthésies
humaines, mais les patients convulsaient au réveil, raison pour laquelle
cette méthode a été abandonnée. Luft développe le principe de la
capnométrie pendant la seconde guerre mondiale en partant du principe
que le CO2 est un gaz qui absorbe les rayons infrarouges{[}1{]}{[}2{]}.
Il dépose deux brevets en 1943 pour des applications pratiques cliniques
: usine de caoutchouc et surveillance environnementale dans les
sous-marins. C'est en 1950 qu'Etan et Liston introduisent le concept de
monitorage du CO2 expiré pendant l'anesthésie par absorption des
infrarouges. Les rayons IR sont absorbés par tous les gaz composés de
plus de deux atomes par molécule, ou par deux atomes différents lorsque
la molécule est composée d'uniquement deux atomes. Cependant, à cause du
poids, du volume et du coût, ces appareils ne sont pas utilisés en
clinique courante. Puis une méthode de capnographie basée sur la
spectrographie photoacoustique est développée au Danemark. Elle repose
sur la détection au microphone des sons produits par la lumière pulsée
en fonction des longueurs d'onde des IR. Le spectromètre de masse, basé
sur la séparation des gaz en fonction de leur masse moléculaire, peut,
de façon séquentielle, échantillonner tous les gaz inspirés et expirés.
Au début du XXe siècle, les chercheurs ont exploré la nature de la
courbe expiratoire du dioxyde de carbone, y compris les composants
physiologiques comme l'espace mort et les gaz alvéolaires, autant par la
méthode de capnographie standard (temporelle) qu'avec la capnographie
volumétrique par échantillonnage séquentiel rapide du volume de
gaz(1){[}3{]}. Dans les années 1950 et les années 1960, les premières
études ont examiné la forme de la courbe de capnographie normal et ont
recherché aux niveaux du CO2 alvéolaire un modèle utilisant des
équations de dilution dépendantes du temps (2)(3){[}4{]}{[}5{]}.
Cependant, ce n'est qu'en attendant la disponibilité des systèmes
commerciaux des analyseurs infrarouges rapides que la détermination de
la pression partielle du dioxyde de carbone dans la respiration a pu
passer du laboratoire de recherche de physiologie au « chevet » du
patient(4){[}6{]}. Cette technologie a d'abord trouvé une application
dans l'anesthésie et plus tard aux soins intensifs pour un nombre limité
d'indications. Ce n'est que récemment que cette technologie a fait son
apparition dans d'autre spécialité comme les urgences s'ouvrant à de
nouvelles applications cliniques. C'est à la fin des années 1950 que les
chercheurs ont commencé à appliquer des approches tout d'abord manuelles
puis informatiques pour l'ajustement du dioxyde de carbone et l'étude
des courbes d'écoulement. Ces courbes de CO2 en fonction du temps encore
appelées capnographie standard et rapporté au volume expiré encore
appelées capnographie volumétrique ont permis l'estimation de paramètres
ventilatoires comme la PCO2 alvéolaire (5){[}7{]}. Ces travaux ont été
initialement entrepris avec des méthodes manuelles consommant du temps
(6){[}8{]} puis ont été rapidement suivis par des méthodes analogiques
(5){[}7{]}, analogiques / numériques combinés (7){[}9{]} et, plus tard,
avec des données numériques coûteuses et complexes (8){[}10{]}. Les
premiers capnographes s'appuyaient sur des méthodes électroniques
analogiques pour déterminer les paramètres de la fréquence respiratoire
(FR) et de la pression partielle de CO2. L'introduction du
microprocesseur au début des années 1970 a conduit à des logiciels basés
sur la mesure de la fréquence respiratoire et de la PETCO2, avec un
fabricant intégrant le CO2 et le volume pour calculer la ventilation
alvéolaire et l'élimination du dioxyde de carbone. Cependant, malgré le
développement de technologie informatique pour identifier et classer les
changements en temps réel du capnogramme dès la fin des années 1980
(9)(10){[}11{]}{[}12{]}, les algorithmes logiciels des dispositifs
commerciaux ne permettent pas à l'heure actuelle l'analyse de la courbe
en clinique courante. Ces dispositifs restent concentrés sur une
estimation robuste du taux de CO2 en fin d'expiration (PETCO2) et de la
fréquence respiratoire.

\subsection{Technique d'enregistrement de
capnographie}\label{technique-denregistrement-de-capnographie}

\subsubsection{Détection du CO2 expiré}\label{detection-du-co2-expire}

Il existe différents systèmes pour détecter et analyser le CO2 expiré.

\paragraph{La spectrométrie de masse
(11){[}2{]}}\label{la-spectrometrie-de-masse-11bhavani1992capnometry}

Elle est basée sur le concept de séparation des ions en fonction de leur
poids moléculaire et de leur charge. Il existe deux types de
spectromètre de masse, celui à secteur magnétique et détecteurs fixés et
le quadripôle. Un champ électrique charge les molécules de gaz et les
transforme en cations qui seront accélérées et analysées. Cette
technique a pour avantages d'analyser plusieurs gaz en même temps, et
voir le CO2 expiré de plusieurs patients en même temps dans différentes
salles opératoires (maximum 31). Mais le spectromètre de masse est un
appareil volumineux et cher qui ne trouve pas son utilité au pied du lit
du malade.

\paragraph{La spectrographie Raman}\label{la-spectrographie-raman}

Elle a été inventée en 1928 par Mr Raman. Un rayon laser d'argon
monochromatique de haute intensité traverse un échantillon de gaz. Les
molécules de gaz dispersent la lumière avec des longueurs d'ondes
caractéristiques, qui sera collectée par des lentilles. Différents
filtres sélectionnent alors la lumière en fonction des gaz mesurés. Le
nombre « d'impacts » par 100 ms est utilisé pour déterminer la
concentration des gaz. Cette technique permet une analyse multi gaz,
rapide.

\paragraph{Méthode colorimétrique}\label{methode-colorimetrique}

Il s'agit d'une méthode récente peu utilisée en raison de l'absence d'un
monitoring continu et d'affichage graphique des variations de CO2. C'est
un système qui détecte la présence de CO2 dans un gaz à l'aide d'un
papier filtre ou un autre révélateur sensible au pH.

\paragraph{La spectrophotométrie d'absorption
infrarouge}\label{la-spectrophotometrie-dabsorption-infrarouge}

La spectrophotométrie par rayonnement infrarouge (IR) est la technique
de capnographie actuellement la plus utilisée en médecine (11){[}2{]}.
Un faisceau IR traverse un mélange gazeux, l'intensité de la lumière
transmise est fonction de la longueur de la cellule d'absorption, de la
pression partielle du gaz et de son coefficient d'absorption. Chaque gaz
absorbe plus spécifiquement une longueur d'onde donnée, il est donc
possible de mesurer l'absorption d'un gaz spécifique, qui sera
proportionnelle à la concentration. Le CO2 absorbe les IR à une longueur
d'onde de 4.3 µm. En comparant la différence d'absorption d'un
rayonnement IR qui traverse deux cellules (une dont la concentration en
CO2 est connue, l'autre dont la concentration reste à mesurer), il est
possible de mesurer la concentration en CO2 du mélange.

\subsubsection{Système d'analyse de
l'ETCO2}\label{systeme-danalyse-de-letco2}

On distingue deux types de système d'analyse de l'ETCO2 (12){[}13{]}.

\paragraph{Système non aspiratif : « Mainstream
»}\label{systeme-non-aspiratif-mainstream}

Le système non aspiratif, mainstream, où la mesure est effectuée
directement sur le flux gazeux du circuit respiratoire du patient,
permet de fournir une analyse rapide et sans distorsion de la
concentration en CO2. La mesure se fait en continu, sans décalage
temporel. L'inconvénient de ce système réside dans le fait que le
capteur alourdit le dispositif et risque de modifier la position de la
sonde d'intubation chez les patients en ventilation invasive.

\paragraph{Système aspiratrif : « Sidestream
»}\label{systeme-aspiratrif-sidestream}

Le système sidestream est un système aspiratif où l'échantillon de gaz
est aspiré à débit constant à travers un tube fin pour être analysé par
une cellule de mesure. Le débit aspiratif est de 150ml/min. Il permet
une analyse multi gaz en utilisant plusieurs longueurs d'ondes. Ce
capnographe a l'avantage d'être léger mais les sécrétions pulmonaires et
l'eau de condensation ont tendance à obstruer le tube d'analyse. Toute
fuite sur le circuit provoque une aspiration d'air ambiant et crée une
mesure anormalement basse du CO2 expiré. Par ailleurs, le délai
d'analyse de l'ETCO2 est allongé. Les progrès technologiques ont permis
de faire un système microstream, qui est un système aspiratif amélioré
où le CO2 transite par une fine tubulure avant d'atteindre la chambre de
mesure. Ce dispositif permet de diminuer le temps d'analyse et permet
ainsi une synchronisation entre flux expiré et courbe capnographique.

Actuellement les systèmes de mesure sont compacts, petits, légers et
faciles d'utilisation. Pour une utilisation clinique, les capnomètres
doivent pouvoir mesurer une concentration de CO2 dans une gamme de 0 à
10 \% (76 mmHg). La précision avec un système IR est d'environ 0.2\%, ce
qui est suffisant en clinique. Les résultats sont faussés par la
présence de protoxyde d'azote ou d'oxygène mais le système est conçu
pour annuler automatiquement ces biais. Par ailleurs, les variations de
température ambiante et de pression atmosphérique peuvent intervenir sur
les résultats de concentration en CO2 notamment sur les systèmes
sidestream.

\subsection{Capnographie et pathologie
respiratoire}\label{capnographie-et-pathologie-respiratoire}

Il existe deux types de capnogrammes, soit il s'agit d'une mesure du CO2
expiré en fonction du temps (capnographie temporelle ou standard), soit
les mesures sont réalisées en fonction du volume expiré (courbes SBT-CO2
ou capnogramme volumétrique).

\subsubsection{La capnographie temporelle ou
sandard}\label{la-capnographie-temporelle-ou-sandard}

Le terme « capnographie standard », c'est-à-dire la capnographie basé
sur le temps, se réfère généralement au tracé du dioxyde de carbone en
fonction du temps par opposition à la capnographie en fonction du volume
expiré volumétrique appelé « capnographie volumétrique » où la phase
inspiratoire de la courbe n'est pas représentée. Les deux formes d'onde
sont subdivisées en trois phases expiratoires avec une quatrième
inspiratoire pour la capnographie standard) (13)(14){[}14{]}{[}15{]}
\emph{figure 1}:

\begin{itemize}
\tightlist
\item
  Phase I : appelée base inspiratoire, elle correspond à la vidange de
  l'espace mort anatomique, la pression partielle en CO2 est proche de 0
  puisque le mélange gazeux de cet espace n'a pas été en contact avec
  les échanges gazeux alvéolaires. Sa composition est comparable à celle
  de l'air inspiré avec peu ou pas de CO2.
\item
  Phase II : de montée expiratoire, on observe une augmentation rapide
  du CO2 expiré, sous forme d'une courbe sigmoïde, correpsondant à la
  transition entre l'espace mort anatomique et l'arrivée des gaz
  alvéolaires. Durant cette étape, le gaz de l'espace mort anatomique
  est remplacé par le gaz alvéolaire. La verticalité de la phase II est
  en faveur d'une distinction nette entre l'espace mort anatomique et
  l'air alvéolaire. L'élimination de l'air alvéolaire est normalement
  synchrone, ce qui est prouvée par l'ascension soudaine de cette phase
  (14){[}15{]}.
\item
  Phase III : on parle de plateau expiratoire, provenant de la vidange
  séquentielle des volumes alvéolaires, Elle correspond à l'expiration
  du CO2 provenant uniquement des alvéoles. Cette phase est en plateau
  voir légèrement ascendant et varie en fonction des pathologies
  respiratoires. Une majoration de la pente correspond à une vidange
  asynchrone des alvéoles lors de l'expiration, ce qui est visible dans
  les pathologies obstructives comme l'asthme et la BPCO. La fin du
  plateau correspond à la PETCO2, affichée sur les écrans de monitoring.
  La PETCO2 est un reflet de la pression artérielle en CO2 chez un sujet
  sain avec un métabolisme stable.
\item
  Phase 0 : c'est l'inspiration d'un gaz dépourvu de CO2 qui provoque
  une chute brutale du CO2 jusqu'à la ligne de base (c'est-à-dire 0
  mmHg). A la phase 0 succède la phase I. Néanmoins, il est parfois
  difficile de faire la distinction entre fin d'expiration et début
  d'inspiration.
\end{itemize}

Les phases II et III sont séparées par un angle alpha, décrit dans la
littérature, dont la norme est comprise entre 100° et 110°. Cet angle
augmente lorsque la pente de la phase III s'accentue et que la phase II
présente une croissance plus progressive. L'angle beta sépare les phases
III et 0, il est proche de 90°. Il permet d'estimer l'importance du
phénomène de réinhalation. C'est ainsi que les différentes phases se
succèdent sur le capnogramme, au fil des mouvements respiratoires.
L'analyse du capnogramme est un élément important dans la surveillance
du patient, aussi bien pour les valeurs numériques que pour l'analyse
des quatre phases et de la modification de l'aspect des courbes
(14){[}15{]}. Il existe deux vitesses de balayage des courbes. Une
lente, avec une vitesse de 25 mm/min, qui montre les tendances générales
de la concentration en CO2 expiré. Une rapide, qui défile à 12.5 mm/sec
soit 750 mm/min, et qui permet l'analyse des détails d'un cycle
respiratoire.

\begin{figure}[h!]

{\centering \includegraphics[width=250px]{figure/courbe_capno} 

}

\caption{Courbe de Capnographie Standard et ses différentes phases.}\label{fig:unnamed-chunk-2}
\end{figure}

\subsubsection{Capnographie
Volumétrique}\label{capnographie-volumetrique}

La capnographie volumétrique correspond à la PCO2 mesurée et exprimée en
fonction du volume lors de la phase d'expiration. La terminologie
utilisée pour les différentes phases, les transitions de phase, les
angles et les pentes sont similaires pour la capnographie standard et la
capnographie volumétrique. Le capnogramme volumétrique a été subdivisé
en volume pour chaque phase (15){[}16{]} et associé à des fonctions
(16){[}17{]} pour mieux caractériser de manière plus fiable la forme
d'onde et produire des estimations dérivées avec une plus grande
reproductibilité. L'analyse de la courbe de capnographie volumétrique
permet d'estimer un certain nombre de mesures physiologiques et
physiquement interprétables de la fonction respiratoire, telles que
l'espace mort anatomique et physiologique ainsi que le rapport des deux,
l'élimination du CO2 et le flux sanguin capillaire pulmonaire. Ces
mesures permettent de découvrir de nombreux troubles
cardio-respiratoires comme le syndrome de détresse respiratoire aiguë,
la BPCO, l'asthme et l'embolie pulmonaire. Ces paramètres
morphologiques, qui divisent le capnogramme volumétrique en phases I-III
(17){[}18{]}, permet une caractérisation et une interprétation accrues
du capnogramme volumétrique à l'aide d'un certain nombre de
caractéristiques dérivées. L'un des intérêts de cette analyse est de
permettre une mesure substitutive non invasive pour l'espace mort
anatomique, l'espace mort des voies aériennes, estimé par la méthode de
Fowler (18){[}19{]} qui utilise les parties centrales `' linéaires''
ajustées en phase III (par exemple, 40 à 80\% du volume expiré) et la
phase II. Cette mesure permet de calculer l'espace mort des voies
respiratoires sur une base respiratoire pour chaque patient plutôt que
de compter sur la règle plus approximative de 1 mL/kg d'espace mort en
fonction du poids. Brewer et al. (19){[}20{]} a constaté que cette
règle, dérivée des moyennes de population, par rapport à l'espace mort
mesuré de la voie aérienne, ne présentait aucune corrélation (r2 =
0,0002). Au fil du temps, les chercheurs s'inquiètent de l'effet des
variations de mesure sur les estimations du volume de l'espace mort de
la méthode Fowler et de la variation résultante de ces paramètres
calculés. Cela a conduit au développement de méthodes à faible
dispersion, qui comprenaient l'ajustement du volume expiratoire de CO2
par rapport à la courbe de volume expiratoire entre 40 et 80\% du volume
expiré en utilisant des polynômes de première, seconde et troisième
ordre (20){[}21{]}. Plus récemment, un ajustement rapide du capnogramme
volumétrique à l'aide d'une approximation fonctionnelle déterminée avec
un algorithme non linéaire de moindres carrés a été utilisé
(16){[}17{]}, ce qui a entraîné une diminution de la variabilité
respiratoire intra-patient ainsi que la dispersion des paramètres
calculés. Les caractéristiques discutées dans cette section sont
destinées à être interprétées en utilisant des concepts physiologiques
ou cliniques connus.

\subsubsection{Pathologies respiratoires et courbes de
capnographie}\label{pathologies-respiratoires-et-courbes-de-capnographie}

L'analyse des données fournies par la capnographie doit être réalisée de
manière systématique. Une variation de la PETCO2 doit tenir compte des
modifications de la forme du capnogramme et des trois éléments qui
interviennent sur la PETCO2, à savoir la production métabolique de CO2,
son transport en fonction du débit sanguin et son élimination par la
ventilation.

\paragraph{Intubation oesophagienne}\label{intubation-oesophagienne}

Affirmer que la sonde d'intubation se trouve dans la trachée est souvent
difficile, et l'auscultation pulmonaire ne permet pas toujours de faire
la part des choses, notamment chez les sujets obèses. C'est pourquoi la
vérification de la position de la sonde par le suivi de la courbe de
capnographie est devenue une pratique courante en pré-hospitalier et en
anesthésie(21){[}22{]}. En cas d'intubation oesophagienne, on observera
soit l'absence de courbe de capnographie, soit quelques courbes dont
l'amplitude va décroître progressivement. Quand celles-ci sont visibles,
elles reflètent le gaz entré dans l'estomac lors de la ventilation non
invasive au BAVU, ou l'ingestion au préalable de traitements
anti-acides, carbonatés ou effervescents créant une réaction chimique
avec production de CO2 intra-gastrique. Mais ce CO2 est rapidement
évacué par la ventilation, provoquant une chute de l'EtCO2 et un
aplatissement des courbes. Il est donc recommandé d'attendre 6 cycles
respiratoires avant de confirmer l'intubation endotrachéale.

\paragraph{Intubation sélective}\label{intubation-selective}

La bronche souche droite forme un angle de 25° avec la trachée, elle est
donc pratiquement dans son axe, ce qui explique que dans la majorité des
cas d'intubation sélective accidentelle, la sonde se trouve dans la
bronche droite. Lors d'une intubation sélective, il existe deux
phénomènes. Le premier est lié à une augmentation de l'espace mort, un
élargissement du gradient alvéolo-artériel en CO2 et une baisse de la
PETCO2. Le second est une diminution de la ventilation alvéolaire, une
augmentation de la PaCO2 et donc de la PETCO2 au niveau de la bronche
non intubée(21){[}22{]}.

\paragraph{Embolie pulmonaire
(22)(23){[}23{]}{[}24{]}}\label{embolie-pulmonaire-2223kline1998preliminarywiegand2000effectiveness}

L'embolie pulmonaire augmente l'espace mort alvéolaire, entraînant une
diminution du CO2 expiré par dilution et par conséquent une PETCO2
inférieure à des valeurs normales. Le gradient alvéolo-artériel et par
conséquent la P(a-ET)CO2 augmente. Plusieurs études ont été menées pour
discuter de l'intérêt de la mesure du CO2 expiré dans le diagnostic de
l'embolie pulmonaire(13)(22){[}14{]}{[}23{]}. Une des pistes de travail
est d'exclure ce diagnostic quand le patient a une probabilité clinique
faible et un gradient P(a-ET)CO2 normal malgré des D-dimères positifs.
Mais plus récemment, c'est l'aire sous la courbe de capnographie qui est
utilisée pour diagnostiquer l'EP. Celle-ci diminue quand le diagnostic
est confirmé. L'aire moyenne sous la courbe de patients ayant une EP est
d'environ la moitié moindre de celle des patients sains. Ces données
suggèrent qu'un patient présentant des symptômes pouvant évoquer une EP
avec courbe de capnographie ayant une faible aire sous la courbe doit
être considéré comme étant à haut risque.

\paragraph{Trouble ventilatoire obstructif : Asthme et
BPCO}\label{trouble-ventilatoire-obstructif-asthme-et-bpco}

L'asthme se caractérise par une hyperréactivité bronchique, une
inflammation et un remodelage bronchique (24){[}25{]}. On observe
essentiellement une diminution du calibre des voies aériennes appelée
obstruction bronchique. Le bronchospasme modifie l'allure du capnogramme
avec une augmentation de la pente en phase III. Le bronchospasme
entraîne une diminution de la ventilation alvéolaire sans diminution de
la perfusion pulmonaire ce qui diminue le rapport ventilation-perfusion
(V/Q). Or, la réduction de diamètre des bronches n'est pas homogène dans
l'ensemble poumon, certaines zones étant plus spastiques que d'autres,
de ce fait on assiste également à une diminution du rapport V/Q
alvéolaire local. On parle d'hétérogénéité en parallèle du rapport V/Q
alvéolaire(25){[}26{]}. Cela cause une déformation du capnogramme
marquée par une (14){[}15{]} \emph{figure 2}:

\begin{itemize}
\tightlist
\item
  Diminution de la verticalité en phase II
\item
  Ouverture de l'angle alpha
\item
  Augmentation de la pente en phase III
\end{itemize}

Dans les cas sévères, cette courbe prend une forme triangulaire. La
pente en phase III et l'angle alpha sont associés au degré de sévérité
de la crise d'asthme, plus ceux-ci sont importants, plus le
bronchospasme est majeur (26){[}27{]}. Des études ont montré que l'ETCO2
des patients faisant une crise d'asthme est plus bas que les sujets
sains (35)(26){[}28{]}{[}27{]}. Dans ce contexte la PETCO2 n'est plus le
reflet de la pression alvéolaire en CO2 du fait d'un trapping alvéolaire
important. On observe une augmentation du temps expiratoire, ce qui
entraîne une augmentation du gradient P(a-ET)CO2. Une étude américaine a
montré que le rapport dCO2/dt, qui reflète la pente du plateau
alvéolaire du capnogramme, permet de détecter l'obstruction
bronchique(27){[}29{]}. L'auscultation pulmonaire avec la recherche de
sibilants ne permet pas toujours de prédire la présence ou non d'une
obstruction bronchique, et la sévérité des sibilants n'est pas forcément
corrélée au diamètre des voies aériennes, d'où l'intérêt d'une mesure
objective (28){[}30{]}. Krauss et al.(29){\textbf{???}} ont montré dans
les troubles obstructifs de type asthme et BPCO, une déformation de la
courbe de capnographie caractéristique en forme d'« aileron de requin »
ou « shark's fin » en anglais. Ces modifications chez les malades de
pathologie pulmonaire obstructive (BPCO et asthme) sont marquées par une
courbure importante {[}5{]}{[}31{]}, {[}10{]}{[}32{]}, {[}11{]}{[}33{]}
et résulte de l'arrivée décalée du gaz alvéolaire {[}3{]}{[}34{]}. Ces
modifications de courbe chez les patients obstructifs étaient corrélées
aux altérations retrouvées lors des mesures de spirométrie.
Comparativement, il y a peu de données sur la morphologie du capnogramme
chez les patients présentant une insuffisance cardiaque aiguë ou
d'autres maladies pulmonaires restrictives.

\begin{figure}[h!]

{\centering \includegraphics[width=250px]{figure/courbe_capno_patho} 

}

\caption{Courbe de Capnographie Standard normale (en noir) et pahtologique (en rose).}\label{fig:unnamed-chunk-3}
\end{figure}

\pagebreak

\subsection{Analyse d'article - Bibliographie sur la
Capnographie}\label{analyse-darticle---bibliographie-sur-la-capnographie}

\subsubsection{Expiratory capnography in asthma- evaluation of various
shape indices
{[}31{]}}\label{expiratory-capnography-in-asthma--evaluation-of-various-shape-indices-you1994expiratory}

\subsubsection{Capnography for monitoring non-intubated spontaneously
breathing patients in an emergency room setting
{[}30{]}}\label{capnography-for-monitoring-non-intubated-spontaneously-breathing-patients-in-an-emergency-room-setting-egleston1997capnography}

\subsubsection{Segmented Wavelet Decomposition for Capnogram Feature
Extraction in Asthma Classification
{[}35{]}}\label{segmented-wavelet-decomposition-for-capnogram-feature-extraction-in-asthma-classification-betancourt2014segmented}

\subsubsection{Investigation of capnogram signal characteristics using
statistical methods
{[}36{]}}\label{investigation-of-capnogram-signal-characteristics-using-statistical-methods-kazemi2012investigation}

\subsubsection{New Prognostic Index to Detect the Severity of Asthma
Automatically Using Signal Processing Techniques of Capnogram
{[}37{]}}\label{new-prognostic-index-to-detect-the-severity-of-asthma-automatically-using-signal-processing-techniques-of-capnogram-kazemi2016new}

\subsubsection{\texorpdfstring{Automated Quantitative Analysis of
Capnogram Shape for COPD--Normal and COPD--CHF Classification
{[}38{]}}{Automated Quantitative Analysis of Capnogram Shape for COPD--Normal and COPD--CHF Classification ,{[}38{]}}}\label{automated-quantitative-analysis-of-capnogram-shape-for-copdnormal-and-copdchf-classification-mieloszyk2014automated}

Cette article a été écrit par le \emph{Mieloszyk R. et al.} du
\emph{Massachusetts Institute of Technology Cambridge USA} et pupblié en
2014. L'objectif principal de ce travail était de distinguer en les
classifiant de façon automatique à l'aide de la courbe de capnographie,
les patients indemnes de pathologies pulmonaires, les patients
présentant une exacerbation de BPCO et les patients en insuffisance
cardiaque aiguë. L'objectif principal était de distinguer les patients
en exacerbation de BPCO par rapport au patient en insuffisance cardiaque
aiguë.

La section III traite du prétraitement du capnogramme et de la
conception du classificateur (algorithme d'apprentissage) à l'aide de
l'ensemble d'apprentissage.

Les données ont été recueillies prospectivement sur des échantillon de
patients provenant de trois centres d'inclusion et sur une période de
sept ans. Les patients sains indemne de toutes pathologie pulmonaire ont
été inclus sur deux sites. Le recrutement des patients présentant une
insuffisance cardiaque aiguë ou une décompensation de BPCO ont été
réalisées sur le troisièmes site dans le service des urgences chez des
patients présentant une détresse respiratoire aiguë.

Les enregistrements des patients des urgences étaient réalisés en
position assise à l'aide d'une canule nasale à un capnographe portable
\emph{(Capnostream 20, Oridion Medical, Needham, MA, USA)}. Le
capnographe recueille un échantillon continu à un débit de 50 mL / min
et enregistre le PeCO2 instantané toutes les 50 ms. Pour ces patients
les enregistrements étaient effectués sur une durée 10 à 30 min.
Concernant les sujets normaux, après avoir obtenu leur consentement
éclairé , l'enregistrement s'est fait dans les même conditions en
position assis avec une canule nasale avec le même type de matériel
(Capnostream 20). Les durées d'enregistrement étaient de 3 min ou 15 min
en fonction des sites pour les sujets normaux. L'analyse a été réalisé
après partitionnement des données en deux groupes. Un premier groupe de
données permettant de développer et d'entrainer l'algorithme de
classification. Le second groupe correspond à groupe test permettant
d'évaluer et de valider le modèle à l'aide de test (d'analyse) de
perfomance en comparant les résultats de classification à l'aide de
l'algorithme de classification avec les résultats établis par les
cliniciens. Au total 143 patients ont été inclus. Quatre patients ont
été exclus car ils présentaient un tableau mixte de décompensation
cardiaque et d'exacerbation de BPCO. Le groupe d'entrainement presentait
84 patients distribué en : 20 patients normaux, 31 patients en ICA et 33
patients en exacerbation de BPCO. Le groupe test quand à lui comprenait
55 patients : 10 patients normaux, 22 patients en ICA et 23 patients en
exacerbation de BPCO. Le prétraitement des capnogrammes comprennait :
l'isolement des capnogrammes un par un en identifiant et marquant le
début et la fin de chaque expiration; un capnogramme type était produit
en moyennant les profils expiratoire pour chaque patient et en éliminant
les profils abérants correspondant une déviation exagérer par rapport au
profil moyen; et pour finir l'extraction des caractéristiques
physiologiques sélectionnées de chaque expiration. Le début et la fin de
chaque segment d'expiration pour chaque enregistrement ont été
identifiés à l'aide d'un algorithme reconnaissant le début des pentes
positive et négative. Un profil de capnogramme moyen a ensuite été
construit en superposant les expirations en les centrants sur une valeur
prédéterminée de 15 mmHg de PeCO2 (condidérée comme une valeur précoce
et reproductible pour chaque expiration), et en moyennant toutes les
valeurs à chaque pas de temps (ou chaque instant). L'amplitude de
variation par rapport au modèle a été déterminée en calculant, à chaque
instant, l'écart-type des expirations superposées. Un exemple d'analyse
et montré dans la \emph{figure 3}.

\begin{figure}[h!]

{\centering \includegraphics[width=250px]{figure/mieloszyk2014_fig1} 

}

\caption{Profil expiratoire d'un capnogramme de patient BPCO. Les exhalaisons de capnogrammes (en bleu) sont alignées à 15 mmHg de PeCO2 puis moyennées verticalement pour construire le profil moyen du capnogramme (en rouge).}\label{fig:unnamed-chunk-4}
\end{figure}

Les expirations atypiques ont été reconnues et exclues en utilisant le
profil moyen comme référence de capnogrammele modèle de capnogramme
comme exemple de respiration. Chaque expiration a été notée sur la base
d'une mesure globale des déviations par rapport au profil moyen
standard, en se référant à l'écart-type approprié (associé). Les
expirations, dont l'écart global dépassaient un certain seuil, étaient
exclues de l'analyse et de la classiffication. Après le prétraitement
automatisé de chaque enregistrement pour éliminer les exhalaisons des
valeurs aberrantes, les 80 premières expirations valides de chaque
enregistrement ont été utilisées pour l'analyse.

Les quatre caractéristiques physiologiques utilisées pour la conception
de l'électeur ont été sélectionnées à l'aide de l'analyse de validation
croisée décrite plus loin et comprenaient les éléments suivants
\emph{figure 4}:

\begin{itemize}
\tightlist
\item
  durée de l'expiration;
\item
  PeCO2 maximum ou PeCO2 de fin d'expiration (ETCO2);
\item
  temps passé à ETCO2;
\item
  pente de fin d'expiration.
\end{itemize}

La durée d'expiration est la durée entre le début de l'expiration et la
fin de celle-ci. PeCO2 maximum est la valeur PeCO2 à la fin de
l'expiration. Le temps passé à ETCO2 est la durée pendant laquelle PeCO2
reste à sa valeur maximale. La pente d'expiration finale a été calculée
comme la pente d'une droite correspondant aux cinq dernières valeurs de
PeCO2 de l'exhalation.

\begin{figure}[h!]

{\centering \includegraphics[width=450px]{figure/mieloszyk2014_fig2} 

}

\caption{Quatre caractéristiques extraites du capnogramme et utilisées pour la classification. Celles-ci comprennent : la durée de l'expiration, le CO2 en fin d'expiration (ETCO2), la pente de l'expiration finale et le temps passé à l'ETCO2.}\label{fig:unnamed-chunk-5}
\end{figure}

L'entraînement et le vote de chaque électeur se déroulent dans l'espace
à quatre dimensions défini par les caractéristiques. Deux approches pour
la constitution et le vote de chaque électeur ont été suivi. Dans une
première approche, respiration par respiration, les caractéristiques de
chaque expiration dans l'ensemble des respirations constituent un point
de l'espace \(\mathbb R^{4}\) des caractéristiques, de sorte qu'il y a
autant de points que d'expiration. Dans la seconde approche par moyenne
des caractéristiques, les valeurs de ces caractéristiques respectives
sont moyennées sur l'ensemble des expirations pour un enregistrement
donné permet de définir un point unique dans l'espace \(\mathbb R^{4}\)
des caractéristiques.

Dans cette article les auteurs fond par des besoins informatiques, du
coût qu'on nécessité les calculs de ces analyses. Les calculs pour le
prétraitement, l'extraction de caractéristiques, la validation croisée
et la construction ROC sur l'ensemble de l'ensemble ont duré moins de
2,7 minutes sur un ordinateur portable MacBook Pro 2012 (Apple,
Cupertino, Californie) avec 4 Go de RAM et un processeur Intel Core i7
de 2,2 GHz. processeur exécutant MATLAB 2013a (MathWorks, Natick,
Massachusetts).

La section III traite du prétraitement du capnogramme et de la
conception du classificateur à l'aide de l'ensemble d'apprentissage.

\subsubsection{Model-Based Estimation of Pulmonary Compliance and
Resistance Parameters from Time-Based Capnography
{[}39{]}}\label{model-based-estimation-of-pulmonary-compliance-and-resistance-parameters-from-time-based-capnography-abid2015model}

Cette article a été écrit par le \emph{Abid A. et al.} du
\emph{Massachusetts Institute of Technology Cambridge USA} et pupblié en
2015. Elle porte sur l'analyse de la courbe de capnographie par un
modèle mathématique de type mécanistique. Pour réaliser cette analyse 15
patients asthmatiques ont été inclus avec un âge médian de 48 ans en
majorité des femmes (73\%). L'enregistrement des courbes de capnographie
associé à la mesure du VEMS (Volume Expiratoire Maximal Seconde) a été
fait avant, pendant et après un test à la Metacholine. La réponse à la
Métacholine était considérée comme positive pour une diminution de plus
de 20\% du VEMS. Des Bêta2-mimétiques étaient administrées comme
bronchodilateur après le test à la Metacholine pour permettre un retrour
à la normale. Les mesures de capnographie de longueur variable (entre 2
et 5 minutes) ont été enregistrés en utilisant un capnographe portable
\emph{(Capnostream 20, Covidien, Mansfield, Massachusetts)} à l'aide
d'une canule nasale avec une fréquence de mesure à 20 Hz. L'analyse de
capnographie a été réalisée par post-traitement, capnogramme par
capnogramme, en éliminant les segments de la phase inspiratoire et en
gardant uniquement les segments de la phase d'expiration. L'analyse a
consisté en un ajustement de la courbe, des segments expiratoires
mesurés, à un modèle mathématique de type mécanistique \emph{(fitting)}.
Ce modèle est basé sur l'hypothèse de deux sous-systèmes interconnectés.
Le premier est le sous-système de flux d'air, qui régit le débit d'air
total, mobilisé par les différences de pression dans différents
compartiments pulmonaires. Le second est le sous-système de mélange des
gaz, qui s'appuie sur le sous-système de flux d'air, et régit
l'évolution du taux de CO2 lors du mélange des zones riches en CO2 des
poumons avec l'air pauvre en CO2 correspondant à l'espace mort.
Concernant le sous-système de flux d'air : Un modèle simple est utilisé,
celui d'un compartiment alvéolaire unique, qui modélise la région des
poumons qui s'étend (en grande partie l'espace alvéolaire) comme une
paire de compartiments coulissants, comme le montre la figure 1. Ces
compartiments sont reliés les uns aux autres avec un ressort de
constante de \(C_{l}\), représentant la compliance du tissu alvéolaire,
et reliés à l'atmosphère par un tuyau de résistance \(R_{l}\),
représentant les résistances des voies aériennes supérieures. Concernant
le sous système de mélange : Lors de l'inspiration, l'espace mort, qui
est la région du poumons ne participant pas aux échanges gazeux, est
rincée avec de l'air atmosphérique contenant une fraction de CO2
négligeable. L'air dans la région alvéolaire, cependant, a des quantités
substantielles de CO2 (à pression partielle \(p_{A}\)) en raison des
échanges gazeux avec les capillaires pulmonaires. Lors de l'expiration,
l'air riche en CO2 des alvéoles se mélange à l'air pauvre en CO2 de
l'espace mort. Nous modélisons ce processus comme un mélange instantané
d'air: une quantité infinitésimale d'air pénètre dans l'espace mort et
se mélange instantanément pour créer un mélange homogène d'air dans
l'espace mort, de concentration \(p(t)\) au temps \(t\). Une quantité
infinitésimale de ceci est expulsée par la bouche et le nez pour
maintenir le volume de l'espace mort, \(V_{D}\), fixé. Ce modèle est
basé sur le principe qu'un mélange important se produit dans l'espace
mort par la turbulence et la ramification des voies aériennes dans les
voies respiratoires supérieures. En se basant sur ces hypothèses et en
considérant \(p_{D}(t)\) au temps zéro, égale à zéro soit
\(p_{D}(0)=0\), les auteurs retouve une équation du modèle mécanistique
du type : \[
p_{D}(t)=p_{A} - p_{A} \mathrm{e}^{\alpha} \mathrm{e}^{\alpha\mathrm{e}^{-\frac{t}{\tau}}} 
\tag{1}
\] avec \(\alpha= \frac{-C_{l}\Delta P}{V_{D}}\) et \(\tau=R_{l}C_{l}\).

\begin{itemize}
\tightlist
\item
  \(C_{l}\) : la compliance pulmonaire
\item
  \(R_{l}\) : la résistance des voies aériennes
\item
  \(\Delta P\) : gradient de pression moteur en expiration
\item
  \(p_{A}\) : pression partielle de CO2 alvéolaire
\item
  \(V_{D}\) : volume de l'espace mort anatomique
\end{itemize}

Tableau 1: Paramètres

\begin{longtable}[]{@{}cl@{}}
\toprule
paramétres & description\tabularnewline
\midrule
\endhead
\(C_{l}\) & la compliance pulmonaire\tabularnewline
\(R_{l}\) & la résistance des voies aériennes\tabularnewline
\(\Delta P\) & gradient de pression moteur en expiration\tabularnewline
\(p_{A}\) & pression partielle de CO2 alvéolaire\tabularnewline
\(V_{D}\) & volume de l'espace mort anatomique\tabularnewline
\bottomrule
\end{longtable}

Les expirations inférieures à 1,25 seconde ont été exclus. Les 2 200
expirations résultantes étaient chacune ajustées à la forme de
l'équation en choisissant des valeurs pour \(\alpha\) et \(\tau\)
(alternativement, \(R_{l}\) et \(C_{l}\) pour une fraction fixe
\(\frac{\Delta P}{V_{D}}\)) qui minimisant l'erreur quadratique moyenne
entre le CO2 calculé et le profil du CO2 mesuré. La valeur de \(p_{A}\)
a été fixée par les auteurs à 40 mmHg en considérant l'hypothèse que
cette valeur ne variait pas pendant la provocation à la méthacholine. Un
premier résultat globale non apparié montre des valeurs Pour une
majorités de patient, après le test à la méthacholine, une tendance
semble se dégager avec une augmentation du paramètre \(\tau\) ou une
diminution du pramètre \(\alpha\), voir les deux. Ces résultats
s'expliquent par le fait que la métacholine agissant comme un
bronchoconstricteur augmente les résistances pulmonaires \(R_{l}\) et
ainsi la valeur \(\tau\). Des études suggèrent également que
l'obstruction des voies aériennes augmente l'espace mort physiologique
et ainsi une diminution du paramètre \(\alpha\). Bien qu'une
correspondance entre les paramètres estimés par capnographie et le VEMS
ne soit pas retrouvé chez tous les sujets, la plupart des sujets
affichent une tendance avec une augmentation de \(\tau\) et une
diminution de \(\alpha\) lors de l'administration de la méthacholine. Ce
modèle mécanique simple mono-alvéolaire de poumon semble suffisament
complexe pour extraire des paramètres permettant de décrir et de
discriminer différents état du poumons. Ce modèle reste assez simple
pour estimer les paramètres de résistance et de compliance pulmonaire,
ainsi que les relations existant entre ces deux propriétés
physiologiques. Dans ce travail aucune données n'est communiqués sur les
résidus et l'erreur standard entre les données mesurées et les
estmiations du modèle. Les auteurs nottent également le fait de ne pas
avoir pris en compte les variablités de concentration du CO2 alvéolaire,
l'hétérogénéité de concentration entre les alvéoles et le bruit dû au
capteur.

\subsubsection{Model-Based Estimation of Respiratory Parameters from
Capnography, with Application to Diagnosing Obstructive Lung Disease
{[}40{]}}\label{model-based-estimation-of-respiratory-parameters-from-capnography-with-application-to-diagnosing-obstructive-lung-disease-abid2017model}

\[ 
p_{D}(t)=p_{A} (1-\mathrm{e}^{-\alpha} \mathrm{e}^{\alpha\mathrm{e}^{-\frac{t}{\tau}}})  
\tag{2}
\]

Le paramètre \(\tau=R_{l}C_{l}\) représente la constante de temps
pulmonaire. Le parmètre \(\alpha\) définit par
\(\alpha= \frac{C_{l}\Delta P}{V_{d}} =\frac{V_{T}}{V_{D}}\) où la
quantité \(C_{l}\Delta P=V_{T}\) représente le volume courant (en
supposant une compliance pulmonaire linéaire), donc \(\alpha\) est
approximativement l'inverse de la fraction de l'espace mort pulmonaire

\subsubsection{Bayesian Tracking of a Nonlinear Model of the Capnogram
{[}28{]}}\label{bayesian-tracking-of-a-nonlinear-model-of-the-capnogram-den2006bayesian}

\subsection{Discussion}\label{discussion}

\subsubsection{Prétraitement}\label{pretraitement}

\begin{itemize}
\item
  problèmatique d'analyse de la courbe. Deux modes principaux :
\item
  Un mode d'analyse (éparce) en découpant la courbe capnogramme par
  capnogramme.

  \begin{itemize}
  \tightlist
  \item
    Avec secondairement une analyse de chaque capnogramme pour un
    enregistrement pour avoir des mesures des caractéristiques de la
    courbe capnogrammme par capnogramme permettant d'en estimé une
    moyenne, une médiane et une déviation standard (ou un écart type).
    Problème de selection des capnogrammes correctes et abérrants.
    Mesure de la variablité : méthode et intérêt.
  \item
    OU analyse après production d'un profil moyen en moyennant les
    courbes après supperposition centrées sur une valeur, un maqueur, un
    criètre des différents capnogramme facielement identifiable : une
    valeur fixé (15 mmHg comme chez Mieloszik {[}38{]}) ou la valeur
    maximale de la pente identifiable par le pic positif (valeur max) de
    la dérivée première.
  \end{itemize}
\item
  Un mode d'analyse globale en prenant un compte un essemble définit de
  la courbe avec plusieurs capnogramme analysé. Utilisé dans l'analyse
  par transformée de Fourier mais pourrait s'appliquer à l'analyse par
  trasnformée en ondelettes.
\end{itemize}

\subsubsection{features ou caractèristiques
analysées}\label{features-ou-caracteristiques-analysees}

\subsubsection{Méthode de d'analyse de la
courbe}\label{methode-de-danalyse-de-la-courbe}

\begin{itemize}
\tightlist
\item
  mesure simple de paramètres (temps, EtCO2, pente)
\item
  Ajustement par rappor à une équation
\item
  trasnfromée de Fourier
\item
  transformée en ondelette
\end{itemize}

\subsubsection{Méthode de
Classification}\label{methode-de-classification}

\pagebreak

\begin{center}\rule{0.5\linewidth}{\linethickness}\end{center}

\section{Conclusion}\label{conclusion}

\pagebreak

\begin{center}\rule{0.5\linewidth}{\linethickness}\end{center}

\section*{References}\label{references}
\addcontentsline{toc}{section}{References}

\hypertarget{refs}{}
\hypertarget{ref-luft1943neue}{}
{[}1{]} K. Luft, ``Über eine neue methode der registrierenden gasanalyse
mit hilfe der absorption ultraroter strahlen ohne spektrale zerlegung,''
\emph{Z. tech. Phys}, vol. 24, pp. 97--104, 1943.

\hypertarget{ref-bhavani1992capnometry}{}
{[}2{]} K. Bhavani-Shankar, H. Moseley, A. Kumar, and Y. Delph,
``Capnometry and anaesthesia,'' \emph{Canadian Journal of anaesthesia},
vol. 39, no. 6, pp. 617--632, 1992.

\hypertarget{ref-aitken1928fluctuation}{}
{[}3{]} R. Aitken and A. Clark-Kennedy, ``On the fluctuation in the
composition of the alveolar air during the respiratory cycle in muscular
exercise,'' \emph{The Journal of physiology}, vol. 65, no. 4, pp.
389--411, 1928.

\hypertarget{ref-chilton1952mathematical}{}
{[}4{]} A. B. Chilton and R. W. Stacy, ``A mathematical analysis of
carbon dioxide respiration in man,'' \emph{The bulletin of mathematical
biophysics}, vol. 14, no. 1, pp. 1--18, 1952.

\hypertarget{ref-yamamoto1960mathematical}{}
{[}5{]} W. S. Yamamoto, ``Mathematical analysis of the time course of
alveolar co2,'' \emph{Journal of applied physiology}, vol. 15, no. 2,
pp. 215--219, 1960.

\hypertarget{ref-jaffe2008infrared}{}
{[}6{]} M. B. Jaffe, ``Infrared measurement of carbon dioxide in the
human breath:`breathe-through' devices from tyndall to the present
day,'' \emph{Anesthesia \& Analgesia}, vol. 107, no. 3, pp. 890--904,
2008.

\hypertarget{ref-bellville1959respiratory}{}
{[}7{]} J. W. Bellville and J. Seed, ``Respiratory carbon dioxide
response curve computer: It gives more complete alveolar
ventilation-pcoco2 response curves than could formerly be obtained,''
\emph{Science}, vol. 130, no. 3382, pp. 1079--1083, 1959.

\hypertarget{ref-berengo1961single}{}
{[}8{]} A. Berengo and A. Cutillo, ``Single-breath analysis of carbon
dioxide concentration records,'' \emph{Journal of Applied Physiology},
vol. 16, no. 3, pp. 522--530, 1961.

\hypertarget{ref-murphy1899analogue}{}
{[}9{]} T. Murphy, ``Analogue-digital data processing of respiratory
parameters,'' in \emph{Afips}, 1899, p. 253.

\hypertarget{ref-noe1963computer}{}
{[}10{]} F. Noe, ``Computer analysis of curves from an infrared co2
analyzer and screen-type airflow meter,'' \emph{Journal of Applied
Physiology}, vol. 18, no. 1, pp. 149--157, 1963.

\hypertarget{ref-bao1992expert}{}
{[}11{]} W. Bao, P. King, J. Zheng, and B. Smith, ``Expert capnogram
analysis,'' \emph{IEEE Engineering in Medicine and Biology Magazine},
vol. 11, no. 1, pp. 62--66, 1992.

\hypertarget{ref-ventzas1994capnex}{}
{[}12{]} D. Ventzas, ``CAPNEX: An expert system for capnography (co2
respiration analysis),'' \emph{Transactions of the Institute of
Measurement and Control}, vol. 16, no. 5, pp. 233--244, 1994.

\hypertarget{ref-yosefy2004end}{}
{[}13{]} C. Yosefy, E. Hay, Y. Nasri, E. Magen, and L. Reisin, ``End
tidal carbon dioxide as a predictor of the arterial pco2 in the
emergency department setting,'' \emph{Emergency Medicine Journal}, vol.
21, no. 5, pp. 557--559, 2004.

\hypertarget{ref-jabre2010place}{}
{[}14{]} P. Jabre, X. Combes, and F. Adnet, ``Place de la surveillance
de la capnographie dans les détresses respiratoires aiguës,''
\emph{Réanimation}, vol. 19, no. 7, pp. 633--639, 2010.

\hypertarget{ref-howe2011use}{}
{[}15{]} T. A. Howe, K. Jaalam, R. Ahmad, C. K. Sheng, and N. H. N. Ab
Rahman, ``The use of end-tidal capnography to monitor non-intubated
patients presenting with acute exacerbation of asthma in the emergency
department,'' \emph{Journal of Emergency Medicine}, vol. 41, no. 6, pp.
581--589, 2011.

\hypertarget{ref-rayburn1994neural}{}
{[}16{]} D. B. Rayburn, T. M. Fitzpatrick, and S. Van Albert, ``Neural
network evaluation of slopes from sequential volume segments of
expiratory carbon dioxide curves,'' in \emph{Neural networks, 1994. ieee
world congress on computational intelligence., 1994 ieee international
conference on}, 1994, vol. 6, pp. 3530--3533.

\hypertarget{ref-tusman2011capnography}{}
{[}17{]} G. Tusman, F. SUAREZ-SIPMANN, S. Bohm, J. Borges, and G.
Hedenstierna, ``Capnography reflects ventilation/perfusion distribution
in a model of acute lung injury,'' \emph{Acta anaesthesiologica
Scandinavica}, vol. 55, no. 5, pp. 597--606, 2011.

\hypertarget{ref-fletcher1981concept}{}
{[}18{]} R. Fletcher, B. Jonson, G. Cumming, and J. Brew, ``The concept
of deadspace with special reference to the single breath test for carbon
dioxide,'' \emph{British journal of anaesthesia}, vol. 53, no. 1, pp.
77--88, 1981.

\hypertarget{ref-fowler1948lung}{}
{[}19{]} W. S. Fowler, ``Lung function studies. ii. the respiratory dead
space,'' \emph{American Journal of Physiology-Legacy Content}, vol. 154,
no. 3, pp. 405--416, 1948.

\hypertarget{ref-brewer2008anatomic}{}
{[}20{]} L. M. Brewer, J. A. Orr, and N. L. Pace, ``Anatomic dead space
cannot be predicted by body weight,'' \emph{Respiratory care}, vol. 53,
no. 7, pp. 885--891, 2008.

\hypertarget{ref-tang2007systematic}{}
{[}21{]} Y. Tang, M. Turner, and A. Baker, ``Systematic errors and
susceptibility to noise of four methods for calculating anatomical dead
space from the co2 expirogram,'' \emph{British journal of anaesthesia},
vol. 98, no. 6, pp. 828--834, 2007.

\hypertarget{ref-jung2008modalites}{}
{[}22{]} B. Jung, G. Chanques, M. Sebbane, D. Verzilli, and S. Jaber,
``Les modalités de l'intubation en urgence et ses complications,''
\emph{Réanimation}, vol. 17, no. 8, pp. 753--760, 2008.

\hypertarget{ref-kline1998preliminary}{}
{[}23{]} J. A. Kline and M. Arunachlam, ``Preliminary study of the
capnogram waveform area to screen for pulmonary embolism,'' \emph{Annals
of emergency medicine}, vol. 32, no. 3, pp. 289--296, 1998.

\hypertarget{ref-wiegand2000effectiveness}{}
{[}24{]} U. K. Wiegand, V. Kurowski, E. Giannitsis, H. A. Katus, and H.
Djonlagic, ``Effectiveness of end-tidal carbon dioxide tension for
monitoring of thrombolytic therapy in acute pulmonary embolism,''
\emph{Critical care medicine}, vol. 28, no. 11, pp. 3588--3592, 2000.

\hypertarget{ref-ozier2011pivotal}{}
{[}25{]} A. Ozier, ``The pivotal role of airway smooth muscle in asthma
pathophysiology,'' \emph{Journal of allergy}, vol. 2011, 2011.

\hypertarget{ref-hisamuddin2009correlations}{}
{[}26{]} N. N. Hisamuddin, A. Rashidi, K. Chew, J. Kamaruddin, Z.
Idzwan, and A. Teo, ``Correlations between capnographic waveforms and
peak flow meter measurement in emergency department management of
asthma,'' \emph{International journal of emergency medicine}, vol. 2,
no. 2, pp. 83--89, 2009.

\hypertarget{ref-langhan2008quantitative}{}
{[}27{]} M. L. Langhan, M. R. Zonfrillo, and D. M. Spiro, ``Quantitative
end-tidal carbon dioxide in acute exacerbations of asthma,'' \emph{The
Journal of pediatrics}, vol. 152, no. 6, pp. 829--832, 2008.

\hypertarget{ref-den2006bayesian}{}
{[}28{]} J. O. Den Buijs, L. Warner, N. W. Chbat, and T. K. Roy,
``Bayesian tracking of a nonlinear model of the capnogram,'' in
\emph{Engineering in medicine and biology society, 2006. embs'06. 28th
annual international conference of the ieee}, 2006, pp. 2871--2874.

\hypertarget{ref-yaron1996utility}{}
{[}29{]} M. Yaron, P. Padyk, M. Hutsinpiller, and C. B. Cairns,
``Utility of the expiratory capnogram in the assessment of
bronchospasm,'' \emph{Annals of emergency medicine}, vol. 28, no. 4, pp.
403--407, 1996.

\hypertarget{ref-egleston1997capnography}{}
{[}30{]} C. Egleston, H. B. Aslam, and M. Lambert, ``Capnography for
monitoring non-intubated spontaneously breathing patients in an
emergency room setting.'' \emph{Emergency Medicine Journal}, vol. 14,
no. 4, pp. 222--224, 1997.

\hypertarget{ref-you1994expiratory}{}
{[}31{]} B. You, R. Peslin, C. Duvivier, V. D. Vu, and J. Grilliat,
``Expiratory capnography in asthma: Evaluation of various shape
indices,'' \emph{European Respiratory Journal}, vol. 7, no. 2, pp.
318--323, 1994.

\hypertarget{ref-smalhout1975atlas}{}
{[}32{]} B. Smalhout and Z. Kalenda, \emph{An atlas of capnography}.
Institute of Anaesthesiology, University Hospital Utrecht, 1975.

\hypertarget{ref-kelsey1962expiratory}{}
{[}33{]} J. Kelsey, E. Oldham, and S. Horvath, ``Expiratory carbon
dioxide concentration curve a test of pulmonary function,''
\emph{Diseases of the Chest}, vol. 41, no. 5, pp. 498--503, 1962.

\hypertarget{ref-dubois1952alveolar}{}
{[}34{]} A. DuBois, R. Fowler, A. Soffer, and W. Fenn, ``Alveolar co2
measured by expiration into the rapid infrared gas analyzer,''
\emph{Journal of applied physiology}, vol. 4, no. 7, pp. 526--534, 1952.

\hypertarget{ref-betancourt2014segmented}{}
{[}35{]} J. P. Betancourt, ``Segmented wavelet decomposition for
capnogram feature extraction in asthma classification,'' \emph{Journal
of Advanced Computational Intelligence and Intelligent Informatics},
vol. 18, no. 4, pp. 480--488, 2014.

\hypertarget{ref-kazemi2012investigation}{}
{[}36{]} M. Kazemi and M. Malarvili, ``Investigation of capnogram signal
characteristics using statistical methods,'' in \emph{Biomedical
engineering and sciences (iecbes), 2012 ieee embs conference on}, 2012,
pp. 343--348.

\hypertarget{ref-kazemi2016new}{}
{[}37{]} M. Kazemi and A. H. Teo, ``New prognostic index to detect the
severity of asthma automatically using signal processing techniques of
capnogram,'' \emph{Journal of Intelligent Procedures in Electrical
Technology}, vol. 7, no. 26, 2016.

\hypertarget{ref-mieloszyk2014automated}{}
{[}38{]} R. J. Mieloszyk, ``Automated quantitative analysis of capnogram
shape for copd--Normal and copd--CHF classification,'' \emph{IEEE
Transactions on Biomedical Engineering}, vol. 61, no. 12, pp.
2882--2890, 2014.

\hypertarget{ref-abid2015model}{}
{[}39{]} A. Abid, R. J. Mieloszyk, G. C. Verghese, B. S. Krauss, and T.
Heldt, ``Model-based estimation of pulmonary compliance and resistance
parameters from time-based capnography,'' in \emph{Engineering in
medicine and biology society (embc), 2015 37th annual international
conference of the ieee}, 2015, pp. 1687--1690.

\hypertarget{ref-abid2017model}{}
{[}40{]} A. Abid, R. J. Mieloszyk, G. C. Verghese, B. S. Krauss, and T.
Heldt, ``Model-based estimation of respiratory parameters from
capnography, with application to diagnosing obstructive lung disease,''
\emph{IEEE Transactions on Biomedical Engineering}, vol. 64, no. 12, pp.
2957--2967, 2017.


\end{document}
