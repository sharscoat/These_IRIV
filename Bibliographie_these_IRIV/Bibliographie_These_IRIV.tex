\documentclass[12pt,]{article}
\usepackage{lmodern}
\usepackage{amssymb,amsmath}
\usepackage{ifxetex,ifluatex}
\usepackage{fixltx2e} % provides \textsubscript
\ifnum 0\ifxetex 1\fi\ifluatex 1\fi=0 % if pdftex
  \usepackage[T1]{fontenc}
  \usepackage[utf8]{inputenc}
\else % if luatex or xelatex
  \ifxetex
    \usepackage{mathspec}
  \else
    \usepackage{fontspec}
  \fi
  \defaultfontfeatures{Ligatures=TeX,Scale=MatchLowercase}
\fi
% use upquote if available, for straight quotes in verbatim environments
\IfFileExists{upquote.sty}{\usepackage{upquote}}{}
% use microtype if available
\IfFileExists{microtype.sty}{%
\usepackage{microtype}
\UseMicrotypeSet[protrusion]{basicmath} % disable protrusion for tt fonts
}{}
\usepackage[margin=1in]{geometry}
\usepackage{hyperref}
\hypersetup{unicode=true,
            pdftitle={Bibliographie\_These\_IRIV},
            pdfauthor={Sébastien HARSCOAT},
            pdfborder={0 0 0},
            breaklinks=true}
\urlstyle{same}  % don't use monospace font for urls
\usepackage{longtable,booktabs}
\usepackage{graphicx,grffile}
\makeatletter
\def\maxwidth{\ifdim\Gin@nat@width>\linewidth\linewidth\else\Gin@nat@width\fi}
\def\maxheight{\ifdim\Gin@nat@height>\textheight\textheight\else\Gin@nat@height\fi}
\makeatother
% Scale images if necessary, so that they will not overflow the page
% margins by default, and it is still possible to overwrite the defaults
% using explicit options in \includegraphics[width, height, ...]{}
\setkeys{Gin}{width=\maxwidth,height=\maxheight,keepaspectratio}
\IfFileExists{parskip.sty}{%
\usepackage{parskip}
}{% else
\setlength{\parindent}{0pt}
\setlength{\parskip}{6pt plus 2pt minus 1pt}
}
\setlength{\emergencystretch}{3em}  % prevent overfull lines
\providecommand{\tightlist}{%
  \setlength{\itemsep}{0pt}\setlength{\parskip}{0pt}}
\setcounter{secnumdepth}{5}
% Redefines (sub)paragraphs to behave more like sections
\ifx\paragraph\undefined\else
\let\oldparagraph\paragraph
\renewcommand{\paragraph}[1]{\oldparagraph{#1}\mbox{}}
\fi
\ifx\subparagraph\undefined\else
\let\oldsubparagraph\subparagraph
\renewcommand{\subparagraph}[1]{\oldsubparagraph{#1}\mbox{}}
\fi

%%% Use protect on footnotes to avoid problems with footnotes in titles
\let\rmarkdownfootnote\footnote%
\def\footnote{\protect\rmarkdownfootnote}

%%% Change title format to be more compact
\usepackage{titling}

% Create subtitle command for use in maketitle
\providecommand{\subtitle}[1]{
  \posttitle{
    \begin{center}\large#1\end{center}
    }
}

\setlength{\droptitle}{-2em}

  \title{Bibliographie\_These\_IRIV}
    \pretitle{\vspace{\droptitle}\centering\huge}
  \posttitle{\par}
    \author{Sébastien HARSCOAT}
    \preauthor{\centering\large\emph}
  \postauthor{\par}
      \predate{\centering\large\emph}
  \postdate{\par}
    \date{5 février 2018}

\usepackage[french]{babel}
\usepackage[utf8]{inputenc}
\setlength\parindent{24pt}\setlength{\parskip}{0.0pt plus 1.0pt}
\usepackage{graphicx}
\usepackage{float}
\usepackage{pdflscape}
\newcommand{\blandscape}{\begin{landscape}}
\newcommand{\elandscape}{\end{landscape}}
\usepackage{amsmath,amsfonts,amssymb}
\usepackage[french]{babel}

\begin{document}
\maketitle

{
\setcounter{tocdepth}{2}
\tableofcontents
}
\pagebreak

\listoffigures

\pagebreak

\listoftables

\pagebreak

\hypertarget{introduction}{%
\section{Introduction}\label{introduction}}

Les pathologies respiratoires sont un motif fréquent consultation aux
urgences et nécessitent, dans la majorité des cas, une hospitalisation
dans un service pour la suite de la prise en charge. Une part importante
de ces patients sont atteints de pathologies pulmonaires atteignant les
voies de conduction respiratoire. La principale caractéristique de ces
pathologies est de présenter un syndrome obstructif par augmentation des
résistances des bronchioles distales. Les deux pathologies les plus
connues de cette famille sont l'asthme et la Broncho-Pneumopathie
Chronique Obstructive (BPCO) associée ou non à un emphysème. Ces deux
pathologies sont responsables d'une morbi-mortalité importante avec pour
conséquences de nombreuses hospitalisations pour décompensation aiguë.
Actuellement, il n'existe que deux moyens non invasifs de surveiller de
façon continue une pathologie pulmonaire. Il s'agit de la saturation
pulsée en oxygène (SpO2) et de la capnographie par lunettes nasales. La
capnographie est un monitoring en temps réel du CO2 expiré, disponible
et facilement réalisable dans les services d'urgences. Des modifications
de la courbe de capnographie sont décrites dans la littérature en
association avec l'atteinte obstructive de certaines pathologies comme
l'asthme en rapport avec l'augmentation des résistances respiratoires.
Cette approche descriptive ne permet pas d'objectiver quantitativement
les troubles respiratoires sous-jacents. Pour le moment, aucun système
ne permet de quantifier les variations et les modifications de ces
courbes de CO2 expiré. A ce jour, les seuls examens permettant
d'apprécier les résistances des voies aériennes et par conséquent leurs
modifications, sont les explorations fonctionnelles respiratoires (EFR)
avec l'étude des volumes et des débits respiratoires, notamment lors de
la spirométrie. Les EFR sont considérés comme étant les examens de
référence dans la mesure précise et objective d'une obstruction
bronchique surtout chez le sujet asthmatique et BPCO. La réalisation
d'EFR comporte des contraintes matérielles et physiques qui rendent son
utilisation difficile dans un contexte de pathologie aiguë chez un
patient instable sur le plan respiratoire ou cardiologique et
nécessitant souvent une oxygénothérapie. Par ailleurs, il s'agit souvent
de patients âgés qui n'ont plus la capacité de se soumettre à ce type
d'explorations, qui demandent de la coopération et la mobilisation de
grands volumes pulmonaires. Dans ce contexte, l'utilisation de la
capnographie permettrait une évaluation et une surveillance plus aisée
de la fonctionnalité respiratoire. Peu d'études montrent l'existence
d'une relation quantifiable entre les courbes de capnographie et les
résultats des EFR. L'objectif de ce travail est d'objectiver l'apport de
la transformée en ondelettes sur l'analyse de la déformation de la
courbe de capnographie permettant la distinction des patients
pathologiques présentant un syndrome obstructif. Dans ce but, nous
établiront dans un premier temps l'état des connaissances sur la
technique de capnographie et particulièrement sur l'analyse de la courbe
ainsi que de ses tentatives de modélisation. Puis nous développeront
l'analyse du signal de capnographie par la transformée en ondelettes
tout d'abord par une phase de simulation permettant d'observer le
comportement de cette transformation en fonction de la déformation de la
courbe. Nous terminerons par une phase expérimentale à l'aide de données
de capnographie et de spirométrie recueillies sur des patients.

\pagebreak

\begin{center}\rule{0.5\linewidth}{\linethickness}\end{center}

\hypertarget{premiere-partie---physiologie-physiopathologie-du-poumon}{%
\section{Première Partie - Physiologie \& Physiopathologie du
Poumon}\label{premiere-partie---physiologie-physiopathologie-du-poumon}}

\hypertarget{rappel-danatomie-et-de-physiologie-respiratoire}{%
\subsection{Rappel d'Anatomie et de Physiologie
Respiratoire}\label{rappel-danatomie-et-de-physiologie-respiratoire}}

{[}\protect\hyperlink{ref-hogg2013small}{1}{]} La frontière entre les
voies respiratoires purement conductrices et les tissus échangeurs de
gaz est formée par la division d'environ 44 510 +/- 15 574 bronchioles
terminales (SD moyenne) en bronchioles de transition ou en bronchioles
respiratoires du premier ordre où apparaissent d'abord les ouvertures
alvéolaires (Fig 1) .8, 9 Les ouvertures alvéolaires augmentent
progressivement avec les générations successives de ramifications
jusqu'à occuper toute la surface de la lumière des conduits alvéolaires
et les conduits ramifiés pendant plusieurs générations avant de se
terminer dans des sacs alvéolaires fermés.8 Le terme «acinus» désigne
une unité de poumon fournie par une bronchiole terminale unique; le
terme lobule secondaire fait référence à des groupes d'acini identifiés
soit par un septa du tissu conjonctif environnant (Fig. 1B), soit par un
déplacement du motif de ramification bronchiolaire (Fig 1C) par rapport
à un groupe dont les bronchioles prélobulaires se ramifient environ tous
les centimètres en bronchioles intralobulaires groupe seulement
millimètres les uns des autres.8,10 La variation de la longueur du
trajet de la trachée à la surface d'échange de gaz distribue les voies
respiratoires de 2 mm de la quatrième à la 14e génération de
ramification des voies aériennes et les bronchioles intralobulaires
entre la huitième et la 22e génération de branchement des voies
respiratoires.11,12 La carte codée par couleur (Fig. 2B) montre que des
générations uniques de ramifications contiennent des voies aériennes de
différentes tailles en reproduisant les données de Weibel concernant
l'emplacement des voies aériennes de différentes tailles.

les premières mesures directes ont montré que les voies aériennes de
\textless{} 2 mm de diamètre représentaient \textless{} 20\% de la
résistance totale des voies respiratoires inférieures dans les poumons
des animaux anesthésiés et post mortem.

\hypertarget{physiopathologie-pulmonaire}{%
\subsection{Physiopathologie
Pulmonaire}\label{physiopathologie-pulmonaire}}

\hypertarget{asthme}{%
\subsubsection{Asthme}\label{asthme}}

{[}\protect\hyperlink{ref-busse2000pathophysiology}{2}{]} L'asthme
affecte environ 8\% de la population adulte et jusqu'à 20\% des enfants
en Amérique du Nord, en Europe et en Australie. La plupart de ces
patients ont une maladie légère à modérée qui peut être contrôlée avec
des corticostéroïdes inhalés et un agoniste \(\beta\)-adrénergique
inhalé à courte action. Cependant, on estime qu'environ 5\% à 10\% des
patients asthmatiques présentent une maladie grave resistante au
traitement habituelles, notamment l'administration de corticostéroïdes
systémiques. Ce groupe de patients représente un sous-groupe important
de patients asthmatiques. Par exemple, ils ont la plus grande déficience
de leur mode de vie, leur maladie a de profondes répercussions sur le
système de santé et leur traitement utilise un montant proportionnel de
dollars consacrés aux soins de santé. En outre, ce fardeau est partagé
de manière disproportionnée par les minorités et les femmes. De plus,
notre compréhension de la physiopathologie de cet aspect de l'asthme est
mal comprise. Compte tenu de l'importance de l'asthme sévère sur le coût
du système de santé et de l'absence de connaissances bien définies sur
les mécanismes de la maladie, l'Institut national du cœur, des poumons
et du sang a parrainé un atelier intitulé «Physiopathologie de l'asthme
sévère». évaluer un certain nombre de domaines d'intérêt / objectifs:
(1) définir les caractéristiques de l'asthme sévère, (2) identifier les
facteurs mécanistes associés à l'asthme sévère, (3) discuter des
approches potentielles d'étude et de traitement, et (4) présenter des
recommandations pour les besoins et l'orientation futurs de la
recherche. Ce qui suit est un résumé de la présentation et de la
discussion de l'atelier.

+++ {[}\protect\hyperlink{ref-tillie2004physiopathologie}{3}{]}

L'asthme est une maladie multifactorielle résultant de la conjonction de
facteurs congénitaux (terrain atopique, anomalies constitutionnelles du
muscle lisse\ldots{}) et d'éléments liés à l'environnement(allergènes,
polluants, infections\ldots{}). Malgré de nombreux travaux, la
physiopathologie de l'asthme reste controversée : l'efficacité des
médiateurs {[}beta-adrdnergiques sur le spasme bronchique a conduit à
mettre en avant le dysfonctionnement du systéme nerveux autonome, mais
d' autres facteurs interviennent, comme la réponse immunitaire locale
aux allergènes, agents viraux et irritants tels que les polluants
extérieurs. Tous ces mécanismes concourent, des degrés divers, au
développement de la réaction inflammatoire bronchique, commune 5
l'asthme intrinsèque ou extrinsbque (allergique) {[}1{]}. Les phénomènes
d'obstruction bronchique, variables dans le temps, associent de façon
intriquée bronchospasme, sur lequel agissent les {[}Beta-2-mimétiques,
et inflammation bronchique intégrant oedème paridtal, hypersécrétion et,
de fagon plus tardive, remodelage (remodeling) des voies aériennes.
Cette inflammation est présente un stade précoce de la maladie, parfois
de faqon latente.

Aspects anatomopathologiques L'inflammation bronchique au cours de la
maladie asthmatique a été démontrée à la fois par les études post mortem
et par l'analyse de biopsies bronchiques et du LBA. Chez les patients
décédés d'asthme, on note une hypertrophie du muscle lisse, un
épaississement de la membrane basale, des bouchons de mucus obstruant
les bronchioles terminales avec hyperplasie des cellules glandulaires et
la présence de cellules inflammatoires infiltrant la muqueuse
bronchique. Cet infiltrat est composé de polynucléaires éosinophiles,
neutrophiles, de lymphocytes et de cellules mononuclées {[}3{]}.
L'épaississement de la membrane basale, présenté à un stade précoce de
la maladie, correspond à un dépôt de collagène de types I, III et V dans
la région sous-épithéliale. Cette fibrose sous-épithéliale traduit
l'activation des myofibroblastes présents en nombre accru dans la
sous-muqueuse et est le reflet du remodeling {[}14{]}.

+++

+++++++ {[}\protect\hyperlink{ref-west1988physiopathologie}{4}{]} La
bronchocontstriction dans l'asthme est du à une baisse de l'AMPcyclique
intracellulaire par relargage de médiateur (Histamine, fracteur
chimiotactique des ésoinophiles et neutrophile, prostaglandine,
bradykinine, PAF, Triptase, TNF, dérivées de l'acide arachidonique,
cytokines). Augmente la perméablité capillaire créant un oedème
péribronchique. Augmentation de la CPT et de la CRF, une certaine perte
de la rétraction élastique entrainant un déplacement de la courbe
pression-volume en haut et à gauche (comme dans l'emphysème). Il est
vraisemblable que le bronchospasme touche les voies aériennes de tout
calibre et la relation entre la conductance des voies aéreinnes et la
pression de rétraction élastique est fortement perturbée. Le
rétrécissement des voies aériennes principales larges est moyennes
peut-être observé en bronchoscopie. L'hypoxémie artérielle est fréquente
dans l'asthme. Elle est provoquée par une inégalité du rapport
ventilation-perfusion (V/Q) avec des régions hypoventilées occasionnant
un effet shunt. Des inégalités marquées de distribution du débit sanguin
sont également observées. L'espace mort physiologique et le shunt
physiologique sont généralement tous les deux anormalement élevés.

+++++++

{[}\protect\hyperlink{ref-plantier2016mechanisms}{5}{]}

+++++++

\hypertarget{bronchite-chronique-obstructive---bpco}{%
\subsubsection{Bronchite Chronique Obstructive -
BPCO}\label{bronchite-chronique-obstructive---bpco}}

{[}\protect\hyperlink{ref-hogg2004pathophysiology}{6}{]} ++++ La
bronchopneumopathie chronique obstructive (BPCO) se caractérise par une
limitation irréversible du débit d'air mesurée pendant l'expiration
forcée (1,2), provoquée soit par une augmentation de la résistance des
petites voies aériennes conductrices (3,4,5), soit par une augmentation
de la compliance pulmonaire due à une destruction emphysémateuse du
parenchyme pulmonaire, (6) ou les deux. Les unités de résistance des
voies respiratoires sont en cm H2O / L /s et pour la compliance en L /
cm H2O, et leur produit (temps) fournit la constante de temps pour la
vidange des poumons (7). Ces constantes sont reflétées par les mesures
du volume d'air expirable en une seconde (VEMS) et de son rapport à la
capacité vitale forcée (VEMS / FVC), qui sont des outils de dépistage
fiables, car ils sont affectés à la fois par l'obstruction des voies
respiratoires et par l'emphysème. Les lésions pulmonaires retrouvé dans
la BPCO sont secondaire à une réponse immunitaire inflammatoire en
réaction à une exposition chronique à des gaz et particules toxiques
inhalées. Ce processus inflammatoire contribu à l'obstruction des
petites voies aériennes conductrices au niveau des bronchioles \ldots{}.
Les atteintes histologiques comprennent une rupture de la barrière
épithéliale, l'altération de l'appareil mucociliaire entrainant une
baisse de la clairance avec accumulation d'exsudats inflammatoires
muqueux dans la lumières des petites voies respiratoires, l'infiltration
par des cellules inflammatoires ainsi que le dépot de tissu conjonctif
dans les paroies de petites voies respiratories. Ce remodelage et cette
réparation épaississent les parois des voies respiratoires, réduisent le
calibre de la lumière et limitent l'augmentation normale du calibre
produite par le gonflement des poumons. La destruction des poumons
emphysémateux est associée à une infiltration du même type de cellules
inflammatoires que celles trouvées dans les voies respiratoires. Le
schéma centrolobulaire de destruction emphysémateuse est le plus
étroitement associé au tabagisme. Bien que ciblant initialement les
bronchioles respiratoires, des lésions distinctes se combinent pour
détruire de gros volumes de tissu pulmonaire. Le schéma panacinaire de
l'emphysème se caractérise par une implication plus uniforme de l'acinus
et est associé à un déficit en alpha-1 antitrypsine. Ces processus
inflammatoires fond intervenir l'immunité inée et l'immunité acquise.
Durant la maladie plusieurs stade sont visibles. Intialement cela
commence par une inflamation des bronches correspondant à la bronchite
chronique. L'inflammation associée à la bronchite chronique se situe
dans l'épithélium des voies aériennes centrales (diamètre interne
supérieur à 4 mm), où elle s'étend le long des canaux des glandes
jusqu'aux glandes produisant du mucus. Ce processus inflammatoire est
associé à une augmentation de production de mucus, d'une clairance
mucociliaire défectueuse et une altération de la barrière épithéliale
induit par le système immunitaire inné de l'hôte. La bronchite chronique
est également associée avec un épaississement des parois bronchiques
principalement liée à une augmentation des dépôts de tissu conjonctif.
Par la suite l'évolution naturelle de l'hypersécrétionde mucus associée
à l'exposition chronique de tabac se fait au détriment d'une
hypersecrétion purulente aggravant les phénomènes inflammatoire
aboutissant à un syndrome obstructif. Bien que les termes bronchite
chronique et obstruction des voies respiratoires soient souvent utilisés
de manière interchangeable, le principal site d'obstruction se trouve en
réalité dans les voies respiratoires conductrices plus petites (diamètre
inférieur à 2 mm) .3--5 Ces voies respiratoires sont réparties entre la
quatrième et la 14e génération de ramification des voies respiratoires,
puisque les branches de l'arbre bronchique humain ne sont pas
dichotomiques73. L'augmentation du nombre de voies respiratoires à
ramification progressive augmente rapidement leur section totale et
diminue leur résistance. Dans les poumons sains, environ 75\% de la
résistance totale des voies respiratoires inférieures se situe dans les
voies respiratoires principales dont le diamètre interne est supérieur à
2 mm, contre 25\% dans les bronches plus petites et les bronchioles.3,5
En outre, les voies respiratoires situées sous la le larynx ne
représente que 50\% de la résistance totale mesurée à la bouche; les
voies respiratoires d'un diamètre interne inférieur à 2 mm ne
représentent que 10-15\% de la résistance totale des voies
respiratoires. Pour cette raison, Mead74 a qualifié les voies
respiratoires conductrices périphériques de zone silencieuse des
poumons, dans laquelle la maladie pourrait s'accumuler pendant de
nombreuses années avec très peu d'effet.

De nombreuses études ont montré qu'il existait des anomalies
structurelles dans les petites voies respiratoires des fumeurs atteints
ou non de BPCO (79-80). Certains chercheurs ont suggéré que
l'accumulation de mucus obstruait les petites voies respiratoires des
patients atteints de BPCO, mais cette «obstruction du mucus» attribués à
un artefact post mortem lié aux événements ayant entraîné la mort.
Cependant, une étude récente entièrement basée sur du tissu pulmonaire
réséqué chirurgicalement provenant de patients à tous les stades de la
classification GOLD de la BPCO a montré une relation entre
l'accumulation d'exsudats inflammatoires contenant du mucus dans la
lumière des voies respiratoires et la sévérité de la BPCO (90). Une
partie de ce mucus pourrait être produite par les glandes des bronches
plus centrales et aspirée dans les voies respiratoires périphériques,
mais la plupart semble s'ajouter aux exsudats inflammatoires qui se
forment dans les lumières des petites voies respiratoires (figure 4).
Les cellules qui migrent à travers l'épithélium de la lumière des voies
respiratoires sont délivrées par les vaisseaux sous-épithéliaux, mais
les follicules lymphoïdes associés à la BALT sont centrés sur les
vaisseaux dans la couche adventitielle (figure 2). Toute augmentation de
tissu entre le muscle lisse des voies respiratoires et la surface de la
lumière empiétera sur le calibre de la lumière et augmentera la
résistance. De plus, cet effet peut être amplifié par la contraction des
muscles lisses pour expliquer l'hyperréactivité des voies respiratoires
observée dans la BPCO (91). Matsuba et Thurlbeck (10) ont montré un
dépôt préférentiel de tissu conjonctif dans le compartiment adventiciel
de la paroi des voies respiratoires dans un emphysème avancé. Nos
découvertes plus récentes (90) ont confirmé cette observation et suggéré
que cette fibrose péribronchiolaire pourrait contribuer à l'obstruction
des voies respiratoires fixes en limitant l'agrandissement du calibre
des voies aériennes provoqué par le gonflement des poumons (figure 4D).
Le processus inflammatoire dans le compartiment adventitiel des petites
voies respiratoires pourrait également détruire le support que les
petites voies respiratoires reçoivent des alvéoles attachées à leurs
murs extérieurs. Bien qu'une diminution du nombre et de la force de ces
attachements ait été impliquée dans une petite obstruction des voies
respiratoires (92,93) et soit en corrélation avec la diminution du VEMS,
(94) mesures directes de la résistance des voies aériennes périphériques
indiquent que cette perte de soutien des alvéoles est une cause moins
importante d'obstruction que la pathologie (de l'atteinte) de la lumière
et de la paroie des voies respiratoires.3 Des rapports antérieurs ont
également montré que les lymphocytes B et les lymphocytes CD4 et CD8
sont présents dans les tissus des voies respiratoires des patients
atteints de BPCO, et qu'une augmentation du nombre de lymphocytes CD8
(95 à 97) et de lymphocytes B (98) est associée à une diminution du
VEMS. Cette augmentation du nombre de lymphocytes est également associée
à une augmentation de la BALT qui est rare chez les non-fumeurs en bonne
santé, plus fréquente chez les fumeurs de cigarettes, (99) et montre une
nouvelle augmentation marquée chez les patients atteints de BPCO graves
(GOLD 3) et très graves (GOLD 4). ) (90). Cette augmentation des
sous-types de lymphocytes et l'apparition de la BALT à ce stade de la
BPCO suggèrent le développement d'une réponse immunitaire adaptative
pouvant être induite par la colonisation microbienne et l'infection.

Phénotypes d'obstruction des voies respiratoires et d'emphysème dans les
voies aériennes limitantes Les progrès vers des traitements spécifiques
de la BPCO pourraient être accélérés en allant au-delà des mesures de
limitation de débit d'air, au diagnostic précis des cibles spécifiques
responsables de la limitation de débit d'air. Cette étape nécessitera
des méthodes de diagnostic quantitatives précises, sûres et non
invasives, qui permettront aux phénotypes obstructif des voies
respiratoires et à l'emphysème de servir de critères d'évaluation
mesurables dans les essais cliniques. L'introduction du scanner a fourni
une méthode objective pour mesurer l'étendue et la gravité de
l'emphysème sur une base régionale.124--127 Cette approche a été
utilisée pour mesurer l'effet du traitement substitutif sur la
progression de l'emphysème dans le déficit en 1-antitrypsine. Des
rapports japonais indiquent également qu'il pourrait être possible de
séparer les phénotypes obstructifs de BPCO avec une résolution élevée de
la tomodensitométrie CT.130. L'imagerie par IRM d'un gaz hyperpolarisé
inhalé offre des perspectives similaires pour le diagnostic de
l'emphysème et présente l'avantage notable d'éliminer l'exposition
contre les rayonnements ionisants.131--135 Bien que ces procédures aient
une valeur limitée pour la médecine clinique pratique à court terme,
elles pourraient devenir extrêmement importantes pour déterminer les
résultats des essais cliniques de tout nouveau traitement pour l'un ou
l'autre des phénotypes de la BPCO.

{[}\protect\hyperlink{ref-hogg2004pathophysiology}{6}{]}

La bronchite chronique L'inflammation associée à la bronchite chronique
se situe dans l'épithélium des voies aériennes centrales (diamètre
interne supérieur à 4 mm), où elle s'étend le long des canaux des
glandes jusque dans les glandes productrices de mucus59,60. Ce processus
inflammatoire est associé à une production accrue de mucus, clairance
mucociliaire défectueuse et perturbation de la barrière épithéliale
fournie par le système de défense inné de l'hôte.20--22 Des cellules
inflammatoires de la réponse innée et de la réponse adaptative de l'hôte
participent à ce processus.60--62 Reid63 a remarqué que les glandes
muqueuses bronchiques étaient élargies dans la bronchite chronique
(figure 3) et a utilisé le rapport de l'épaisseur de la paroi des
bronches (appelée désormais indice de Reid) comme critère de diagnostic
du diagnostic pathologique de la bronchite chronique. Thurlbeck et
Angus64 ont rapporté que l'indice de Reid était normalement distribué
sans rupture nette entre patients avec et sans bronchite, mais la
plupart des études ont montré un chevauchement assez faible avec des
valeurs plus élevées chez les patients atteints de bronchite
chronique65,66. La bronchite chronique est également associée à un
épaississement des facteurs de croissance tels que TGF\_beta se sont
révélés présents dans les voies aériennes centrales 67, mais la
complexité de leur rôle dans ce remodelage dépasse le cadre de la
présente analyse. . La reconnaissance de la perte de la stérilité
bactérienne normale dans les voies respiratoires inférieures en présence
d'une bronchite chronique68 a conduit à l'hypothèse d'une progression
naturelle de l'hypersécrétion de mucus associée à la cigarette à une
hypersécrétion purulente et une bronchite obstructive (69). Cependant,
les études longitudinales entreprises par Peto et ses collègues 70 et le
groupe de Copenhague71 ont montré que les symptômes de la bronchite
chronique ne permettaient pas de prévoir le déclin rapide du VEMS qui
conduisait à la MPOC (figure 1). Cependant, les infections aiguës des
voies respiratoires inférieures chez les fumeurs actuels atteints de
BPCO pourraient entraîner une chute plus rapide du VEMS (72).

Petite obstruction des voies respiratoires Bien que les termes bronchite
chronique et obstruction des voies respiratoires soient souvent utilisés
de manière interchangeable, le principal site d'obstruction se trouve en
réalité dans les plus petites voies aériennes conductrices (diamètre
inférieur à 2 mm) .3--5 Ces voies respiratoires sont réparties entre la
quatrième et la quatorzième génération. ramification des voies
respiratoires, puisque les branches de l'arbre bronchique humain ne sont
pas dichotomiques73. L'augmentation du nombre de voies respiratoires à
ramification progressive augmente rapidement leur section totale et
diminue leur résistance. Dans les poumons sains, environ 75\% de la
résistance totale des voies respiratoires inférieures se situe dans les
voies respiratoires principales dont le diamètre interne est supérieur à
2 mm, contre 25\% dans les bronches plus petites et les bronchioles3,5.
le larynx ne représente que 50\% de la résistance totale mesurée à la
bouche; les voies respiratoires d'un diamètre interne inférieur à 2 mm
ne représentent que 10 à 15\% de la résistance totale des voies
respiratoires. Pour cette raison, Mead74 a qualifié les voies
respiratoires conductrices périphériques de zone silencieuse des
poumons, dans laquelle la maladie pourrait s'accumuler pendant de
nombreuses années avec très peu d'effet. De nombreuses études ont montré
qu'il existait des anomalies structurelles dans les petites voies
respiratoires des fumeurs atteints ou non de MPOC 75--90. Certains
chercheurs ont suggéré que l'accumulation de mucus obstruait les petites
voies respiratoires des patients atteints de BPCO, mais ce «bouchage du
mucus» attribués à un artefact post mortem lié aux événements ayant
entraîné la mort. Cependant, une étude récente entièrement basée sur du
tissu pulmonaire réséqué chirurgicalement provenant de patients à tous
les stades de la classification GOLD de la BPCO a montré une relation
entre l'accumulation d'exsudats inflammatoires contenant du mucus dans
la lumière des voies respiratoires et la gravité de la BPCO90. Bien
qu'une partie de ce mucus puisse être produite par les glandes des
bronches plus centrales et aspirée dans les voies respiratoires
périphériques, la plupart semble être ajoutée aux exsudats
inflammatoires qui se forment dans les lumières des petites voies
respiratoires (figure 4). Les cellules qui migrent à travers
l'épithélium de la lumière des voies respiratoires sont délivrées par
les vaisseaux sous-épithéliaux, mais les follicules lymphoïdes associés
à la BALT sont centrés sur les vaisseaux dans la couche adventitielle
(figure 2). Toute augmentation de tissu entre le muscle lisse des voies
respiratoires et la surface de la lumière empiétera sur le calibre de la
lumière et augmentera la résistance. De plus, cet effet peut être
amplifié par la contraction du muscle lisse pour expliquer
l'hyperréactivité des voies respiratoires relevée dans la BPCO91.
Matsuba et Thurlbeck (10) ont montré un dépôt préférentiel de tissu
conjonctif dans le compartiment adventiciel de la paroi des voies
respiratoires dans un emphysème avancé. Nos découvertes plus récentes90
ont confirmé cette observation et suggéré que cette fibrose
péribronchiolaire pourrait contribuer à l'obstruction des voies
respiratoires en restreignant l'élargissement du calibre des voies
aériennes provoqué par l'inflation des poumons (figure 4D). Le processus
inflammatoire dans le compartiment adventitiel des petites voies
respiratoires pourrait également détruire le support que les petites
voies respiratoires reçoivent des alvéoles attachées à leurs parois
extérieures. Bien qu'une diminution du nombre et de la force de ces
attachements ait été impliquée dans une petite obstruction des voies
respiratoires92,93 et soit corrélée à la diminution du VEMS, 94 mesures
directes de la résistance des voies aériennes périphériques indiquent
que cette perte de soutien des alvéoles est une cause moins importante
d'obstruction la pathologie dans le mur et la lumière des voies
respiratoires. 3 Des rapports antérieurs ont également montré que les
lymphocytes B et les lymphocytes CD4 et CD8 sont présents dans les
tissus des voies respiratoires des patients atteints de BPCO, et qu'une
augmentation du nombre de lymphocytes CD8 (95 à 97) et de lymphocytes B
(98) est associée à une diminution du VEMS. Cette augmentation du nombre
de lymphocytes est également associée à une augmentation de la BALT qui
est rare chez les non-fumeurs en bonne santé, plus fréquente chez les
fumeurs de cigarettes, 99 et montre une nouvelle augmentation marquée
chez les patients atteints de maladies graves (GOLD 3) et très graves
(GOLD 4). ) BPCO.90 Cette augmentation des sous-types de lymphocytes et
l'apparition de la BALT à ce stade de la MPOC suggèrent l'apparition
d'une réponse immunitaire adaptative pouvant être induite par la
colonisation et l'infection microbiennes.30

++++

{[}bpco\_arce{]} La bronchopneumopathie chronique obstructive (BPCO) est
une pathologie caractérisée par une atteinte des petites voies aériennes
et une destruction du parenchyme pulmonaire (emphysème). C'est une des
causes principales de morbidité et mortalité dans le monde et représente
un poids social et économique grandissant. La prévalence de la BPCO
augmente avec l'âge et le tabagisme et est plus fréquente chez les
hommes. En Suisse, 8\% des personnes \textgreater{} 70 ans ont une
obstruction aux fonctions pulmonaires (15\% des hommes et 3\% des
femmes).2 La BPCO est sous-diagnostiquée. Elle se situe à la 6ème place
des causes de mortalité et est responsable de 3\% des coûts liés aux
dépenses de la santé. Les manifestations extra-pulmonaires de la BPCO
incluent les maladies cardiovasculaires, la sarcopénie, l'ostéoporose,
le diabète et les troubles anxio- dépressifs. La BPCO est d'origine
multifactorielle. Gènes, âge, sexe, développement et maturation
pulmonaire, exposition aux particules, en particulier au tabac, statut
socioéconomique, asthme/hyperactivité bronchique, bronchite chronique et
infections sont autant d'éléments impliqués dans la survenue de la
BPCO.1 2. DEFINITION/CLASSIFICATION La BPCO se définit par une
obstruction irréversible des voies aériennes, le plus souvent
progressive. La BPCO n'est pas synonyme de l'emphysème (entité
pathologique ou radiologique, décrivant une destruction du parenchyme
pulmonaire), ni de la bronchite chronique (entité clinique définie par
une anamnèse de toux et d'expectorations durant au moins 3 mois et deux
années consécutives) Les stades de la BPCO sont décrits sous la rubrique
« diagnostic ».

{[}\protect\hyperlink{ref-wagner1977ventilation}{7}{]} Les modèles
observés d'inégalité V / Q et de shunt représentaient toute l'hypoxémie
au repos et pendant l'exercice. Il n'y avait donc pas de preuve d'une
hypoxémie causée par une altération de la diffusion. En conséquence, le
schéma d'inégalité V / Q ne peut pas nécessairement être déduit de la
PO2 artérielle et de la PCO2.

{[}\protect\hyperlink{ref-mcdonough2011small}{8}{]}

{[}\protect\hyperlink{ref-hogg2013small}{1}{]} L'augmentation de la
superficie totale de la section transversale dans les voies
respiratoires distales du poumon humain améliore le mélange de chaque
respiration de marée avec le volume de gaz expiratoire en ralentissant
le débit massique et en augmentant la diffusion de gaz. Cependant, cette
transition favorise également le dépôt de particules en suspension dans
l'air dans cette région, car elles diffusent 600 fois plus lentement que
les gaz. En outre, le dépôt persistant de particules toxiques en
suspension dans l'air stimule une infiltration inflammatoire chronique
des cellules immunitaires ainsi que des processus de réparation et de
remodelage des tissus, ce qui augmente la résistance dans les voies
respiratoires \textless{} 2 mm de diamètre 4 à 40 fois plus dans la
BPCO. Cette augmentation a été initialement attribuée à un
rétrécissement de la lumière car elle augmente la résistance
proportionnellement à la modification du rayon de la lumière portée à la
quatrième puissance. En revanche, l'élimination de la moitié du nombre
de tubes disposés en parallèle est nécessaire pour doubler leur
résistance, et environ 90\% doivent être supprimés pour expliquer
l'augmentation de la résistance mesurée dans la BPCO. Un réexamen récent
de ce problème basé sur la micro-tomodensitométrie indique que les
bronchioles terminales sont à la fois rétrécies et réduites à 10\% des
valeurs de contrôle dans l'emphysème centrolobulaire et à 25\% dans le
phénotype panlobulaire des BPCO très sévère de grade IV de la
classificaiton de GOLD. Ces nouvelles données indiquent que le
rétrécissement et la réduction du nombre de bronchioles terminales
contribuent au déclin rapide du VEMS, ce qui entraîne une obstruction
grave des voies respiratoires dans la BPCO. De plus, l'observation que
la perte bronchiolaire terminale précède l'apparition de la destruction
emphysémateuse suggère que cette destruction commence aux tout premiers
stades de la BPCO.

Un effet de l'augmentation rapide de la surface totale de la section
transversale à mesure que la génération de ramifications augmente (Figs
3A, 3B) ralentit la vitesse de l'écoulement globale et améliore la
diffusion du gaz vers la surface alvéolaire. Bien que cette transition
facilite le mélange entre l'air inspiré et le volume de gaz en fin
d'expiration, elle expose également la région de transition du poumon à
un excès de dépôt de particules en suspension dans l'air car elles
diffusent beaucoup plus lentement que le gaz. Par exemple, les
particules en suspension dans l'air de 0,5 mm de diamètre qui ont la
plus grande probabilité d'atteindre la zone de transition du poumon sont
environ 1 000 fois plus grandes et ont une constante de diffusion
environ 1 million de fois plus petite que les molécules de gaz. De plus,
étant donné que leur déplacement quadratique moyen par seconde par
diffusion est environ 600 fois plus petit que l'oxygène (c'est-à-dire 10
mm / s contre 6 000 mm / s), les particules peuvent se déposer à la
surface de la zone de transition de poumon entre les respirations. En
outre, étant donné que la surface des bronchioles prématernelle,
terminale et transitoire est relativement petite comparée à la surface
des espaces aériens plus distaux, ces voies respiratoires reçoivent la
plus grande dose de particules inhalées par unité de surface.

\begin{figure}[h!]

{\centering \includegraphics[width=500px]{figure/hogg2013small_fig01.1} 

}

\caption{B : Bronchogramme d'un poumon humain post mortem où un groupe de bronchioles terminales est entouré par un septa de tissu conjonctif fibreux pour former un lobule pulmonaire secondaire. C et A : lobule tel que défini par Reid10, où le motif de ramification passe de celui où les branches sont espacées de quelques centimètres à celui où les branches du carré ne sont séparées que de quelques millimètres..}\label{fig:unnamed-chunk-2}
\end{figure}

Figure 1 : B : Bronchogramme d'un poumon humain post mortem où un groupe
de bronchioles terminales est entouré par un septa de tissu conjonctif
fibreux pour former un lobule pulmonaire secondaire. C et A : lobule tel
que défini par Reid10, où le motif de ramification passe de celui où les
branches sont espacées de quelques centimètres à celui où les branches
du carré ne sont séparées que de quelques millimètres.

\begin{figure}[h!]

{\centering \includegraphics[width=500px]{figure/hogg2013small_fig02} 

}

\caption{Cross-section}\label{fig:unnamed-chunk-3}
\end{figure}

\begin{figure}[h!]

{\centering \includegraphics[width=500px]{figure/hogg2013small_fig03} 

}

\caption{Cross-section}\label{fig:unnamed-chunk-4}
\end{figure}

{[}\protect\hyperlink{ref-reid1958secondary}{9}{]}

\hypertarget{emphyseme}{%
\subsubsection{Emphysème}\label{emphyseme}}

++++++++ L'emphysème pulmonaire affecte les bronchioles respiratoires,
les canaux alvéolaires, les sacs alvéolaires et les alvéoles, qui
forment l'unité fonctionnelle «acinus». Il implique une destruction des
parois alvéolaires, ce qui se traduit non pas par une fibrose, comme
c'est le cas dans certaines pneumopathies interstitielles, mais par une
raréfac- tion des alvéoles et du lit vasculaire, et ainsi par une
réduction de la surface d'échange gazeux. Les différents sous-types
d'emphysème sont définis sur la base de la structure touchée du lobule
pulmonaire (= acinus). L'emphysème «centrolobulaire» (également appelé
«centro-acinaire) touche les bronchioles respiratoires, c.-à-d. la
partie proximale de l'acinus. L'emphysème centrolobulaire, qui est la
forme la plus fréquente d'emphysème pulmonaire, est le plus souvent
associé au tabagisme et il est fréquemment localisé dans les zones
apicales des poumons. En cas d'emphysème «panlobulaire» (également
appelé «panacinaire»), l'ensemble de l'acinus est touché. Cette forme
d'emphysème peut soit représenter la forme terminale d'un emphysème
associé au tabagisme, soit être par ex. causée par un déficit en
alpha-1-antitryp- sine (AAT). L'emphysème ``paraseptal'' (également
appelé «acinaire distal») affecte principalement les canaux
alvéolaires{[}\protect\hyperlink{ref-mcdonough2011small}{8}{]}. La
distribution de l'emphysème est également évaluée, avec une distinction
faite entre emphysème «homogène» et emphysème «hétérogène». En cas
d'emphy- sème homogène, les lobes pulmonaires présentent des altérations
«emphysémateuses» réparties de façon relativement uniforme, alors qu'en
cas d'emphysème hétérogène, les altérations emphysémateuses pré-
dominent nettement au niveau d'un lobe pulmonaire ou de segments d'un
lobe. Outre l'atteinte des voies aériennes terminales et des alvéoles,
il convient de noter que l'emphysème est souvent associé à des
pathologies des voies aériennes plus proximales, typique- ment à des
épaississements des parois bronchiques, à des bronchectasies, à une
impaction mucoïde ou à des bronchiolites. L'association d'altérations
des alvéoles et d'altérations des voies aériennes est responsable de la
variabilité d'un trouble ventilatoire obstructif associé. Cette relation
est illustrée dans le «diagramme de Venn» (fig. 1), qui décrit les liens
et chevauchements entre la bronchite chronique, l'emphysème, la BPCO,
l'asthme, la fibrose pulmonaire et d'autres affections impliquant une
limitation du flux
respiratoire{\textbf{???}}{[}\protect\hyperlink{ref-vogelmeier2017global}{10}{]}.

++++

Syndromes de chevauchement Comme préalablement évoqué, il existe
différentes formes mixtes associant d'autres entités pneumologiques et
l'emphysème pulmonaire. Il est par ex. question de «combined pulmonary
fibrosis and emphysema» (CPFE) en présence d'une association de fibrose
pulmonaire et d'emphysème pulmonaire. La forme mixte associant asthme et
BPCO (avec ou sans emphysème pulmonaire) est appelée
«asthma-COPD-overlap» (ACO).

++++

Conséquences physiopathologiques En raison de la réduction de la surface
disponible pour les échanges gazeux qui se produit dans l'emphysème
pulmonaire, la capacité de diffusion des poumons diminue au fur et à
mesure que l'emphysème pulmonaire progresse. Il peut en résulter des
hypoxémies. Il peut également y avoir une altération, à des degrés
variables, des volumes pulmonaires statiques et dynamiques, en fonction
du «phénotype global» (fig. 1). La diminution des forces de rappel
élastique («lung elastic recoil») du parenchyme pulmonaire favorise la
survenue d'une obstruction bronchique. En fonction de l'étiologie et de
l'ampleur des facteurs associés, différents mécanismes
physiopathologiques peuvent influencer le tableau clinique et avoir un
impact sur les options thérapeutiques sélectionnées: -- trouble
ventilatoire obstructif irréversible en cas de BPCO concomitante; --
trouble ventilatoire obstructif (partiellement) réversible en présence
d'un asthme bronchique ou d'un ACO; -- hyperinflation statique et/ou
dynamique; -- altération de la clairance mucociliaire : impaction
mucoïde; -- altération du microbiome pulmonaire : infections; --
«mismatch» ventilation-perfusion; -- hypertension pulmonaire (HP),
groupe 3 de l'OMS; -- hypoxémie -- insuffisance respiratoire de type 1;
-- hypercapnie -- insuffisance respiratoire de type 2.

++++

{[}\protect\hyperlink{ref-hogg2004pathophysiology}{6}{]} Emphysème La
destruction pulmonaire emphysémateuse réduit le débit expiratoire
maximum en diminuant la force de recul élastique disponible pour chasser
l'air hors du poumon.6 Les lésions provoquées par l'emphysème ont été
décrites pour la première fois par Laennec100 et sont définies par la
dilatation et la destruction du tissu pulmonaire au-delà de la
bronchiole terminale101,102. la pratique de l'examen du poumon post
mortem à l'état gonflé a conduit aux descriptions modernes des
différentes formes de destruction du poumon emphysémateux.103--110
L'unité de l'anatomie du poumon sur laquelle sont basées ces
descriptions est définie par les septa du tissu conjonctif qui
l'entourent (figure 5A). et est communément appelé le lobule secondaire
de Miller. Cette unité est visible sur la surface du poumon à l'œil nu
et contient plusieurs acini, définis comme l'unité de poumon fournie par
une bronchiole terminale unique (figure 5B). L'emphysème, de forme
centrolobulaire ou centriacineuse, résulte de la dilatation et de la
destruction des bronchioles respiratoires (figures 5C et D). Ce type
d'emphysème est le plus étroitement associé au tabagisme109 et se
rencontre le plus souvent dans les lobes supérieurs du poumon, où des
lésions séparées peuvent se fusionner pour produire de plus grandes
cavités (figure 6A). La forme panacineuse de l'emphysème est
généralement associée à un déficit en anti1 anti-trypsine (106,107,110)
et est plus fréquente dans les lobes inférieurs (figure 6B) où elle
provoque une dilatation et une destruction plus uniformes de tout
l'acin. Kim et ses collègues110 ont présenté des données suggérant que
l'un ou l'autre de ces types d'emphysème dominait généralement dans les
maladies avancées et que la dominance de la forme centriacineuse était
associée à une obstruction plus grave des petites voies respiratoires.
L'emphysème paraseptal se définit par la destruction de la partie
externe du lobule, près des septa, et par un emphysème irrégulier lié
aux cicatrices (109). La relation entre l'usage de la cigarette et la
présence d'emphysème (figure 7A) montre une courbe dose-réponse
approximative entre les années de conditionnement et la présence
d'emphysème, mais seulement environ 40\% des gros fumeurs développent
une destruction pulmonaire importante. même aux plus hauts niveaux de
tabagisme.111 Cette observation ne doit pas être confondue avec le fait
que seulement 15\% des personnes développent une BPCO 8, car on retrouve
parfois l'emphysème chez les personnes qui maintiennent une fonction
pulmonaire normale111. Ce type d'observation est devenu plus courant.
commun depuis l'introduction du scanner, mais l'hypothèse selon laquelle
cette forme précoce d'emphysème prédit un déclin rapide de la fonction
et le développement ultérieur de la MPOC n'a pas été testée. Le fait
qu'environ 40\% des gros fumeurs développent un emphysème et que
seulement 15\% développent une limitation du débit d'air reflète la
longue évolution subclinique de la BPCO. La figure 7B présente les
données d'une étude12 dans laquelle les fumeurs ayant développé un
emphysème sévère avaient décuplé leurs neutrophiles, macrophages,
lymphocytes T et éosinophiles présents dans leurs poumons, par rapport
aux personnes fumant des quantités similaires mais conservant une
fonction pulmonaire normale. 12 La comparaison des figures 7A et 7B
suggère fortement que les personnes développant un emphysème ont une
réponse amplifiée à la fumée de cigarette, mais le mécanisme de cette
amplification est inconnu.

Cinétique des leucocytes chez les fumeurs Une possibilité est que
l'effet du tabagisme sur la cinétique des leucocytes augmente le nombre
de ces cellules dans le tissu pulmonaire. Un débit cardiaque de 6 L /
min distribue environ 8640 L de sang au poumon dans la circulation
pulmonaire toutes les 24 h, et 86 L supplémentaires (environ 1\% du
débit cardiaque) sont délivrés par les vaisseaux bronchiques
systémiques. Chaque litre de sang contenant environ 109 leucocytes,
environ 8,7012 leucocytes traversent les poumons chaque jour. Les
observations directes de la surface pleurale chez les animaux et les
mesures indirectes chez les humains ont montré que les leucocytes sont
retardés par rapport aux érythrocytes lorsqu'ils traversent des
microvaisseaux pulmonaires.112 Les leucocytes et les érythrocytes sont
ralentis dans les microvaisseaux pulmonaires car leur diamètre maximal
est légèrement plus grand que ceux des capillaires pulmonaires. Mais la
forme discoïde de l'érythrocyte lui permet de se plier et de traverser
ces restrictions beaucoup plus rapidement que les leucocytes. La
disposition du lit capillaire de la paroi alvéolaire en courts segments
d'interconnexion fournit un grand nombre de voies parallèles permettant
aux érythrocytes se déplaçant plus rapidement de s'écouler autour de
leucocytes se déplaçant plus lentement (figure 8A). Cet effet concentre
les leucocytes par rapport aux érythrocytes, produisant un vaste pool de
leucocytes marginés dans les microvaisseaux pulmonaires. Ces leucocytes
peuvent être rapidement mobilisés dans le bassin circulant par le stress
et l'exercice.112 Le tabagisme est connu pour augmenter le nombre de
leucocytes en circulation58 et pour augmenter la taille du pool de
leucocytes marginalisé dans les capillaires pulmonaires en activant les
PMN et en les ralentissant.113.114 Lorsque la limitation du débit
devient plus sévère, la constante de temps pour la vidange du poumon
dépasse la paroi thoracique, d'abord pendant l'exercice puis au repos.
Ce changement produit une hyperinflation dynamique, dans laquelle une
augmentation de la pression alvéolaire par rapport à la pression
pleurale entraîne une compression capillaire. Des études chez des
patients ayant eu besoin du cathétérisme cardiaque pour d'autres raisons
ont montré que la compression capillaire au cours de la manœuvre de
valsalva augmente le pool de leucocytes marginés115. De plus, les PMN
ont tendance à être activés par leur déformation lorsqu'ils traversent
ce type de restriction.116 Chronic l'exposition à la fumée de cigarette
aussi stimule la moelle osseuse à libérer dans la circulation des
cellules plus immatures qui sont plus facilement retardées dans le pool
marginalisé des microvaisseaux pulmonaires.117 Tous ces facteurs
augmentent la population de leucocytes dans les capillaires pulmonaires.
Seule une faible proportion des cellules acheminées vers un site
inflammatoire aigu émigre hors de l'espace vasculaire dans le tissu
pulmonaire et les espaces aériens.118 Ce processus migratoire est
contrôlé par une série complexe d'événements moléculaires qui amorcent
tout d'abord une réaction progressive cellules en circulation. Cette
réponse commence par rigidifier les cellules pour les rendre moins
déformables et les ralentir, puis par la mobilisation de leur
cytosquelette pour leur permettre de se déplacer de manière ciblée le
long de la voie migratoire et par l'expression de protéines d'adhésion
leur permettant de adhérer aux cellules structurelles et aux cellules.
développer la traction dont ils ont besoin pour bouger.119.120 Une série
importante d'études menées par Walker et ses collaborateurs120-122 a
montré que les cellules inflammatoires débutent ce processus migratoire
en recherchant des zones d'endothélium où des lacunes se forment aux
coins où se croisent trois cellules endothéliales (figure 8B). Après
avoir migré à travers ces interstices, ils entrent en contact avec la
membrane basale endothéliale près du côté épais de la paroi capillaire
(figure 8C). Des reconstructions minutieuses en trois dimensions du mur
alvéolaire sur la base de micrographies électroniques en série ont
montré (figure 8D) que le PMN migre à travers des trous préformés de
cette membrane basale et entre en contact avec les fibroblastes lors de
leur entrée dans l'espace interstitiel. Ils utilisent ensuite la surface
du fibroblaste comme guide lorsqu'ils traversent l'espace interstitiel
et entrent en contact avec la membrane basale épithéliale.121,122
L'association très étroite entre les cellules inflammatoires et
interstitielles en migration suggère que le fibroblaste interstitiel
pourrait fonctionner comme le «quart arrière» dirigeant le flux de
cellules inflammatoires à travers le compartiment interstitiel de la
paroi des voies respiratoires. Lorsque le PMN arrive à la membrane
basale épithéliale, il traverse les pores existants, puis migre entre
les cellules alvéolaires de type 1 et de type 2 sur la surface
alvéolaire de l'espace aérien. Le concept que la pathogenèse de
l'emphysème est due à un déséquilibre entre les enzymes protéolytiques a
été introduit par la découverte d'un lien entre l'emphysème sévère et le
déficit en 1-antitrypsine chez l'homme et par des expériences sur
l'animal montrant que le dépôt d'enzymes puissantes produisait des
lésions similaires à celles de l'emphysème dans les poumons .123 Bien
que l'élastase de neutrophile soit l'enzyme la plus impliquée dans ce
processus, il est de plus en plus évident que d'autres cellules et
systèmes enzymatiques sont impliqués. Des découvertes assez récentes
suggèrent également que l'emphysème peut se développer avec une
inflammation faible ou nulle, en perturbant la synthèse des
protéoglycanes et en augmentant l'apoptose dans les tissus
pulmonaires.123 Ces observations importantes doivent être réconciliées
avec le corpus de preuves beaucoup plus large que l'inflammation
pulmonaire constitue le lien entre la cigarette et l'emphysème. Une
meilleure compréhension du comportement migratoire des cellules
inflammatoires à travers le tissu de la paroi alvéolaire, de leur
interaction avec les cellules structurelles et de leur séquence
d'activation lorsqu'elles rencontrent un matériau étranger dans le tissu
pourrait permettre de mieux comprendre la pathogénie de la disparition
de tissu dans l'emphysème.

++++

\hypertarget{dilatation-des-bronches---ddb}{%
\subsubsection{Dilatation Des Bronches -
DDB}\label{dilatation-des-bronches---ddb}}

\hypertarget{bronchiolite-chronique}{%
\subsubsection{Bronchiolite Chronique}\label{bronchiolite-chronique}}

\hypertarget{fibrose-pulmonaire}{%
\subsubsection{Fibrose Pulmonaire}\label{fibrose-pulmonaire}}

\hypertarget{sarcoidose}{%
\subsubsection{Sarcoidose}\label{sarcoidose}}

\hypertarget{lobectomie---pneumectomie}{%
\subsubsection{Lobectomie -
Pneumectomie}\label{lobectomie---pneumectomie}}

\hypertarget{htap}{%
\subsubsection{HTAP}\label{htap}}

\hypertarget{sdra}{%
\subsubsection{SDRA}\label{sdra}}

\hypertarget{oap}{%
\subsubsection{OAP}\label{oap}}

\hypertarget{phatologies-restrictives}{%
\paragraph{Phatologies Restrictives}\label{phatologies-restrictives}}

\hypertarget{pathologies-obstructives}{%
\paragraph{Pathologies Obstructives}\label{pathologies-obstructives}}

Les pathologies respiratoires obstructives comme l'Asthme et la
Bronchite Chronique Obstructive sont un enjeu de santé publique.
L'asthme est une inflammation chronique des bronches, entrainant leur
hyperréactivité à certaines substances. La maladie se manifeste par des
crises, sous forme de sifflements et de gênes respiratoires. Dans les
cas les plus graves, ces crises peuvent nécessiter une hospitalisation.
Les dernières enquêtes nationales montrent une prévalence cumulée de
l'asthme de plus de 10 \% chez l'enfant âgé d'au moins dix ans et une
prévalence de l'asthme actuel de 6 à 7 \% chez l'adulte. En 2006, sont
survenus 1038 décès par asthme en France. Après un pic observé dans les
années 1980, la mortalité par asthme a diminué. Cette diminution est
également observée chez les enfants et adultes jeunes. Selon les données
du PMSI, on dénombre, en 2007, 54 130 séjours pour asthme dont plus de
la moitié chez des moins de 15 ans. Entre 1998 et 2007, le taux annuel
standardisé d'hospitalisation pour asthme a diminué. Les données sur les
recours aux urgences permettent de documenter d'importantes variations
saisonnières des exacerbations d'asthme (1-3). L'Asthme Aigu Grave
(AAG), forme particulièrement grave d'épisodes aigus des crises
d'asthme, représente entre 8 000 et 16 000 hospitalisations par an en
France. Elle nécessite une ventilation dans 13\% des cas. La mortalité
est en diminution mais reste non négligeable avec 1 500 à 2 000 morts
par an soit 4 pour 100 000 habitants en 1990 contre 5,75 en 1970,
associé à des conséquences médico-économiques lourdes. (6). La Bronchite
Chronique Obstructive (BPCO) est une cause majeure de mortalité et
d'incapacité dans le monde entier. Deux aspects anatomiques sont
caractéristiques de la maladie : l'obstruction des bronches et les
destructions du parenchyme pulmonaire. Le diagnostic de BPCO ne peut
être posé qu'après mesure de la fonction respiratoire par des
Explorations Fonctionnelles Respiratoires (EFR) qui identifient un
rapport expiratoire maximal seconde/capacité vitale (VEMS/CV)
\textless{} 70\% ; ceci après inhalation de bronchodilatateur. A la
différence, la bronchite chronique est définie par la présence d'une
toux productive durant 3 mois, pendant 2 années consécutives, mais sans
retentissement sur la fonction respiratoire (8). En moyenne, 5 à 15\%
des adultes dans les pays industrialisés ont une BPCO définie par
spirométrie. En 1990, on considérait que la BPCO occupait la douzième
place mondiale en tant que cause de mortalité et d'invalidité combinées,
mais on s'attend à ce qu'elle devienne la cinquième cause d'ici 2020.
(7). Selon l'INSERM en 2000, on estimait à 1,7 million le nombre de
personnes atteintes de BPCO en France, soit 4,1\% de la population.
Cette proportion monte à 7,5\% chez les plus de 40 ans. La maladie est
plus fréquente chez les fumeurs, le tabac étant le principal facteur de
risque. De par le développement du tabagisme féminin, la BPCO concerne
aujourd'hui presque autant de femmes que d'hommes. En 2013, environ 145
000 personnes atteintes de formes sévères de la maladie bénéficiaient
d'une oxygénothérapie de longue durée (associée ou non à un traitement
par ventilation). La BPCO entraine en outre chaque année de nombreuses
hospitalisations et des décès liés aux exacerbations de la maladie
(poussées d'aggravation des symptômes habituels). En 2013, entre 95 000
et 145 000 hospitalisations liées à la maladie ont été comptabilisées en
France, et environ 16 000 décès par an, en moyenne, ont été enregistrés
sur la période 2000-2011. La mortalité associée à la maladie stagne
malgré les progrès de la prise en charge, probablement en raison d'un
sous-diagnostic de la maladie ou de diagnostics trop tardifs. (9-10).

\pagebreak

\begin{center}\rule{0.5\linewidth}{\linethickness}\end{center}

\hypertarget{deuxieme-partie---modele-de-poumon}{%
\section{Deuxième Partie - Modèle de
Poumon}\label{deuxieme-partie---modele-de-poumon}}

\hypertarget{physiologie-pulmonaire}{%
\subsection{Physiologie pulmonaire}\label{physiologie-pulmonaire}}

\hypertarget{physiologie-pulmonaire-1}{%
\subsubsection{Physiologie pulmonaire
:}\label{physiologie-pulmonaire-1}}

Les deux poumons font partie de l'organe de la respiration mais jouent
également un rôle important d'épuration et de protection vis-à-vis de
l'environnement avec lequel ils sont en contact aérien permanent.
Objectif du poumon est d'oxygèné le sang et d'évacuer le CO2. La
fonction principale des poumons est de permettre à de l'O2 contenu dans
l'air ambiant de pénétrer dans le sang veineux mêlé et de permettre à du
CO2 provenant du sang veineux mêlé d'être rejeté dans l'air ambiant.
Cette fonction d'hématose peut être divisée en deux mécanismes.
Premièrement, la mécanique ventilatoire correspond à l'ensemble des
structures et mécanismes qui permettent un acheminement de l'air
extérieur vers les alvéoles, ainsi que le renouvellement du gaz
alvéolaire. Deuxièmement, la diffusion alvéolocapillaire qui décrit
l'échange à travers la membrane alvéolocapillaire de l'oxygène et du
dioxyde de carbone. Il s'apparente à une membrane de transfert mais avec
de nombreux compartiements les alvéoles, au nombre de 500 milions chez
l'homme. Cette compartiementation crée un éloignement entre l'air
ambiant et la zone de transfert. Ainsi, le poumon génère sa grande
surface de diffusion en se divisant en un multitude d'unités. Le débit
sanguin et le consommation d'oxygène crée un pompe à oxygène (qui
``aspire'' l'O2). Cette poly-compartiementation du poumon permet
d'augmenter la surface déchange dans un espace réduit donc d'optimiser
les transferts des gaz. Cette surface d'échange est comprise entre 50 et
100 \(m^{2}\) soit 85 \(m^{2}\) en moyenne. Cette particularité du
poumon permet également de s'adapter au besoin en optimisant les approts
en fonction des besoins, mobilisation de réserve. L'hétérogénité de
foncitonnement des différents compartiement ou territoire entraine une
réduction de son efficacité. L'homogénéisation de fonctionnement
optimise l'efficacité du poumon, sa rentabilité. C'est clairement ce qui
ce passe pour le rapport ventilation-perfusion. La grande hétérogénéité
du rapport ventilation perfusion entraine une baisse des capacités
d'extraction de l'O2, avec un sang veineux melé au sortir du poumon plus
bas que si le poumon fonctionne de façon plus homogène. L'hétérogénéité
s'applique pour la pefusion pulmonaire, la ventilation, le rapport
ventilation-perfusion mais également aux propriétés de diffusion locale.
Le soufflet pulmonaire avec la ventilation de ce dernier permet
l'évacuation du CO2.

Inspiration active L'entrée de gaz dans les poumons, ou « inspiration »,
est un mécanisme actif qui résulte de la contraction des muscles des
voies aériennes supérieures et de muscles dits inspiratoires. Le
principal muscle inspiratoire est le diaphragme, innervé par les nerfs
phréniques. Des muscles accessoires sont stimulés conjointement,
notamment les muscles intercostaux externes et les scalènes. Leur
contraction permet de rigidifier la cage thoracique et d'éviter son
affaissement lors de la contraction du diaphragme. (3) La contraction de
ces différents muscles permet ainsi une distension de la cage thoracique
et une diminution de la pression qui règne dans la cavité pleurale. Cela
provoque l'augmentation du volume des poumons et diminue la pression
régnant dans les alvéoles ; de l'air est aspiré dans les alvéoles dès
que la pression alvéolaire devient inférieure à la pression
atmosphérique.

Expiration passive sauf lors de détresse respiratoire La sortie de gaz,
ou « expiration », est le plus souvent un phénomène passif, lié à la
restitution de l'énergie élastique accumulée par le système ventilatoire
au cours de l'inspiration. Dans certaines circonstances comme l'exercice
physique ou l'obstruction bronchique sévère, des groupes musculaires
tels que les muscles abdominaux ou intercostaux internes contribuent à
l'augmentation de la pression pleurale et, par conséquent, de la
pression alvéolaire ; l'expiration est alors dite active. Le poumon est
élastique et revient passivement à son volume pré-inspiratoire au cours
de la respiration de repos.

Le volume d'air insipiré (qui entre dans les poumons) est légèrement
plus élévé que le volume d'air expiré car la quantité d'oxygène absorbée
est légèrement supérieur à celle du CO2 élminié.

Les gaz gagnent un côté de l'interface sang/gaz par les voies aériennes
et le sang gagne l'autre côté par les vaisseaux sanguins.

\hypertarget{ventilation}{%
\subsubsection{Ventilation}\label{ventilation}}

Les voies aériennes : Les voies aériennes sont divisés en deux
groupes,une zone de condution et une zone transitionnelle et
repiratoire. La zone de condution débute par la traché suivi de bronche
souche puis de muliple division associé à une réduction de qualibre de
bronchioles au fur et à merue que l'on pénètre dans le poumon. La zone
de conduction n'étant pas en contacte avec les alvéoles, elle ne
participe pas au échange gazeux et forme lespace mort anatomique. Son
volume est d'environ 150 mL environ. Ce volume peut être calculé à
l'aide de la mesure de la concentration en CO2 dez gaz expiré. La région
où s'effectuent les échanges corresond à la zone respiratoire composé
des alvéoles mais également des bronchioles respiratoire des conduits
alvéolaires et des sacs alvéolaires, formant une unité anatomique
appelée acinus. L'air inspiré descend jusqu'aux bronchioles terminale
par un mouvement de masse, comme l'eau dans un tuyau. Au-delà, la
surface tansversale globale des voies aériennes atteint une telle
taille, en raison du grand nombre de ramifications, que la vélocité du
gaz vers l'avant diminue considérablement. La diffusion est le
principale mécanisme de ventilation dans la zone respiratoire.

C'est la ventilation avléolaire est un paramètre essentiel correspondant
à la quantité d'air frais disponible et en contacte pour les échange
gazeux. La distribution de la ventilation varie également entre la base
et le sommet en fonction de la gravité et donc de la position du
patient. Ce phénomène peut être expliqué par la compliance. Il est en
effet intéressant de rappeler qu'une alvéole pulmonaire déjà distendue
va être moins sensible aux variations de pression et pourra ainsi
difficilement accumuler plus d'air. Inversement, une alvéole moins
distendue pourra rapidement augmenter de volume pour une même variation
de pression. Lorsque le patient est debout, le poumon est entrainé vers
le bas par gravité, tirant ainsi sur la plèvre au niveau de l'apex. La
pression pleurale est alors plus négative à l'apex qu'à la base et les
alvéoles sont donc plus distendues. De ce fait, elles sont moins
compliantes à l'apex qu'à la base et sont moins bien ventilées. Ce
principe reste vrai si le patient est en décubitus latéral ou dorsal.
Ainsi, en décubitus latéral gauche par exemple, le poumon droit sera
moins bien ventilé. En décubitus dorsal, la partie postérieure des
poumons sera la mieux ventilée. (8) Les zones inférieurs du poumon sont
mieux ventilées que les zones supérieures. Lors d'un test par inhalation
qu Xénon-133, la ventilation par unité de volume est plus importante
vers le bas du poumon et diminue au fur et à mesure que l'on s'élève. On
retrouve les mêmes résultats quelque soil la position du sujet, mais les
inégalité et le gradient de ventilation change de position. Ainsi en
décubitus dorsal, la ventilation basale et apicale retrouve une
ventilation identique, mais le gradient de ventilation se fait du haut
vers le bas. De même, en décubitus latéral, le poumon situé vers le bas
est le mieux ventilé.

Perfusion pulmonaire - débit sanguin pulmonaire Hétérogénéité de
perfusion - de débit sanguin en fonction de la localisation dans le
poumon. Contôle actif de la circulation avec une vasoconstriction
pulmonaire hypoxique lors de la baisse de la PO2 alvéolaire (hypoxie
dans l'environnement des artères, avec de nombreux médiateur). Le
mécanisme n'est pas encore clairement identifié.

\hypertarget{diffusion}{%
\subsubsection{Diffusion}\label{diffusion}}

Rapport ventilation-perfusion La ventilation et la perfusion pulmonaire
ne sont pas homogènes mais varient principalement en fonction de forces
gravitationnelles et donc de la position du patient. L'augmentation de
la ventilation vers les bases est cependant moins importante que
l'augmentation de la perfusion, si bien que le rapport entre la
ventilation est la perfusion augmente de la base vers le sommet.

Lois de diffusion La diffusion d'un gaz à travers un tissu est régie par
la \emph{loi de Fick}: la diffusion est proportionnelle à la surface du
tissu (S: 50 à 100 m2 pour la barrière alvéolo-capillaire) et
inversement proportionnelle à l'épaisseur (E: 0,5 \(\mu m\) pour la
barrière alvéolo-capillaire) de ce tissu; elle est aussi proportionnelle
au gradient de concentration ( ou de pression: P1 - P2) de part et
d'autre du tissu; elle est enfin proportionnelle à une constante de
diffusion (D) qui tient compte des caractéristiques du gaz (cette
constante est proportionnelle à la solubilité du gaz (Sol) et
inversement proportionnelle à la racine carrée de son poids moléculaire
(PM). Cette constante de diffusion est 20 fois plus élevée pour CO2 que
pour O2. \[ \dot{V}_{gaz} \sim \frac{S}{E}.D.(P_{1}-P_{2})\]
\begin{equation}
  \dot{V}_{gaz} \sim \frac{S}{E}.D.(P_{1}-P_{2})
\label{eq:Loi de Fick}
\end{equation}

où \[ D \sim \frac{Sol}{\sqrt{PM}}\]

Caractéristique du poumon :

\begin{verbatim}
* Ventilation : Compliance et Résistance
* Pefusion : Débit sanguin pulmonaire
* Rapport $\.{Q}/\.{V}$
* Diffusion : Sufrace d'échange et Coefficient de Diffusion
\end{verbatim}

{[}\protect\hyperlink{ref-abid2015model}{11}{]} La compliance pulmonaire
(\(C_{l}\)) et la résistance (\(R_{l}\)) à l'écouelement de l'air sont
des mesures fondamentales de mécanique respiratoire permettant de
caractériser l'état du poumon.

La compliance pulmonaire est le rapport entre le changement de volume
des poumons d'un patient et la variation de la pression alvéolaire
{[}1{]}. Alors que la compliance pulmonaire peut changer aux cours du
cycle respiratoire, les valeurs moyennes de \(C_{l}\) sont de 0.12 à
0.20 L / cmH2O (200 mL/cmH2O) chez les adultes bonne santé {[}2{]} et
peuvent augmenter ou diminuer de manière significative en réponse aux
affections pulmonaires obstructives {[}3{]}, {[}4{]}. La resistance
pulmonaire est le rapport entre le débit d'air sous pression et le débit
d'air dans les voies respiratoires et est généralement de 3.6 à 4.2
cmH2O / L / s chez les adultes en bonne santé et beaucoup plus élevé
chez les sujets atteints de pneumopathie obstructive {[}5{]},{[}6{]}.
Comme mesures agrégées des propriétés mécaniques du poumons, \(C_{l}\)
et \(R_{l}\) jouent un rôle important dans la formule du capnogramme du
patient, la forme d'onde de la pression partielle du dioxyde de carbone
dans l'air expiré en fonction du temps (capnographie temporelle) ou en
fonction du volume expiré (capnographie volumétrique) {[}7{]}. Des
études antérieures ont montré une relation entre les caractéristiques du
capnogramme et les modifications de la compliance et de la résistance
pulmonaires, le déséquilibre ventilation-perfusion et l'inhomogénéité
pulmonaire {[}7{]}-{[}9{]}. De plus, les chercheurs ont utilisé les
caractéristiques du capnogramme pour distinguer les sujets sains des
sujets atteints de maladie pulmonaires {[}10{]},{[}11{]}.

{[}\protect\hyperlink{ref-abid2017model}{12}{]}

{[}\protect\hyperlink{ref-roy2007calculating}{13}{]} : La structure du
modèle présentée dans cet article est basée sur la physiologie et les
paramètres du modèle ont des significations physiques claires. Comme
tout modèle est basé sur des hypothèses, l'analyse du modèle doit être
considérée dans ces limites. Une hypothèse est que les alvéoles peuvent
être modélisées par un nombre limité de compartiments. Ceci est une
simplification évidente, car les voies respiratoires humaines se
composent de millions d'alvéoles. Celles-ci peuvent toutes avoir des
propriétés différentes en termes de vidange ainsi que des taux d'échange
de gaz et de ventilation-perfusion. Une autre hypothèse du modèle est
que les tensions de CO2 alvéolaires et artérielles sont équilibrées. Par
conséquent, toute limitation de la diffusion du CO2 sur la membrane
capillaire alvéolaire est négligée. Bien que ces hypothèses puissent
être justifiées dans des circonstances physiologiques normales, elles
peuvent être moins valables dans les états pathologiques.

{[}\protect\hyperlink{ref-west1988physiopathologie}{4}{]} : Inégalité
ventilatoire en série et en parallèle (page 16). La vitesse de vidanve
d'une region est détreminée par sa constante de temps qui est le produit
de la résistance de sa voie aérienne (R) par sa compliance (C). Plus
grande est la constante de temps (RC), plus longue sera la vidange. Ce
mécanisme est décrit comme une inégalité de ventilation en parallèle
dans le cardre d'une différecne de résistance entre deux régions
ventilées par obstruction partielle des voies aéreinnes (bronchioles).
Un autre mécanisme appelé inégalité de ventitation en série créer par
des dilatations des voies aériennes périphériques qui provoquent des
différences de ventilation le long des segments bronchiques desservant
les unités pulmonaires. Ces régions mal ventilé par diminution de la
diffusion des gaz inspirés compte tenu de l'atteinte des bronchioles
terminale, se viderons en dernier.

{[}\protect\hyperlink{ref-wagner1977ventilation}{7}{]} Une inégalité
ventilation-perfusion est inévitable dans les BPCO, et ceci entraine une
hypoxémie avec ou sans rétention de CO2. Typiquement, le patient de type
A a seulement une hypoxémie modérée (entre 60 et 70 mmHg) et une PaCO2
normale. Par opposition, le patient de type B a une hypoxémie souvent
sévère (entre 40 et 50 mmHg) avec une PaCO2 augmentée, surtout à un
stade avancé de la maladie. La différence alvéolo-artérielle pour la PO2
est toujours augmentée, en particulier chez les patients souffrant d'une
bronchite chronique sévère. Augmentation à la fois de l'espace mort
physiologique et du shunt physiologique. L'espace mort est plus
spécialement augmenté dans l'emphysème, alors que des valeurs élevèes de
shunt sont particulièrement courante au cours de la bronchite chornique.

A l'effort, la PO2 artérielle peut augmenter ou diminuer, selon les
variations de la ventilation et du débit cardiaque, et les modificaitons
dans la distribution de la ventilation et du débit sanguin. Chez
quelques patients, le facteur principal de la chute de la PaO2 est une
limitation du débit cardiaque ce qui, en présence d'une inégalité
ventilation-perfusion, majore l'hypoxémie. Les patients qui ont une
rétention de CO2 présentent souvent des valeurs de PaCO2 plus élevées à
l'effort, en raison de la limitation de leur réposne ventilatoire.

La lésion essentielle de l'emphysème centroacinaire est une dilatation
des bronchioles respiratoires au milieu de l'acinius, ce qui est
certainement une anomalie morphologique qui peut engendrer des
inégalités en série.

Les effets délétères de l'obstruction des voies aériennes sur les
échanges gazeux sont limités par la ventilation collatérale qui se
produit chez ces patients. Des conduits de communication existent
normalement entre les avléoles adjacentes et entre les petites voies
aériennes voisine, fait provué expérimentalement. Chez ces patients, le
faible débit sanguin allant aux unités non ventilées met en valeur
l'efficacité de la ventilation collatérale, puisque certaines voies
aériennes sont vraissemeblablement complétement fermées, en particulier
dans la bronchite sévère. La vasoconstriction hypoxique et un autre
facteur qui réduit l'importance de l'inégalité ventilation-perfusion.
Cette réponse locale à une PO2 alvéolaire basse diminue le débit sanguin
destiné aux régions faiblement ou non ventilées, minimisant l'hypoxémie
artérielle. Lorsque des patients ayant une BPCO reçoivent des
bronchodilatateurs, par un exemple du Sablutamol, une légère chute de la
PaO2 se produit parfois. Ceci est probablement dû à l'action
vasodilatatrice de ces médicaments qui augmente le débit sanguin des
alvéoles faiblement ventilés. Ces constations sont encore plus nettes
dans l'asthme.

{[}\protect\hyperlink{ref-wagner1978ventilation}{14}{]} : L'hypoxémie
artérielle est fréquente dans l'asthme. Elle est provoquée par une
inégalité du rapport venitation-pefrusion (VA/Q). L'inégalité de
ventilation est largement démontrée aussi bien par la méhtode en apnée
que par la méhtode du rinçage sur plusieurs cycles. De plus, les mesures
avec le xénon radioactif montrent des régions hypoventilées, en
particulier lorsqu'elles sont recherchées par l'enregistrement du débit
d'entrée ou sortie du gaz. Des inégalités marquées de distribution du
débit sanguin sont également observées, avec typiquement des réductions
transitoires concernant des zones différentes à des moments différents.
L'espace mort physiologique et le shunt physiologique sont généralement
tous les deux anormalement évevées. Un exemple de ditribution des
rapports veniltation-perfusion montre chez un asthmatique présentant des
symptômes modérés, présente une distribution franchement différente
d'une distribution ventilation-perfusion normale. Elle montre une
distribution bimodale avec une part très importante (environ 25\%) du
débit sanguin total qui va aux uintés à bas rapport VA/Q (environ à
0.1), ce qui rend compte de l'hypoxémie modérée du patient. Après avoir
bénéficié d'un aérosol bronchodilateur avec diminution du bronchospasme
associée à une diminution des résistances respiratoires, une
modification des rapport ventilation-perfusion apparait. Le débit
sanguin allant aux alvéoles à bas VA/Q augmente de 25 à 50\% du débit
total avec comme conséquence une baisse de la PaO2. Le mécanisme de
l'augmentation de l'hypoxémie semble être une levée de la
vasonconstriction dans les zones faiblement venitlées. La chute de la
PaO2 est associée à une augmentation de l'espace mort et du shunt
physiologique. En pratique, les effets favorables de ces médicaments sur
les voies aériennes excèdent de loin les desavantages d'une hypoxémie
modérée supplémentaire.

\hypertarget{modele-mathematique-de-poumon}{%
\subsection{Modèle mathématique de
poumon}\label{modele-mathematique-de-poumon}}

Modèle mathématique le plus simple fréquemment utilisé est le modèle
\emph{monocompartiemental}. Modèle connu et utilisé depuis longtemps, le
flux d'air est proportonnel à la différence de pression entre la
pression alvéolaire \(P_{A}\) et la pression à la bouche \(P_{atm}\). La
constante de proportionnnalité correspond à la résistance \(R\) de
l'arbre bronchique à l'écoulement. Cette résistance dépend de la
dimensions du tube (longueur et surface) et de la viscosité du fluide.
Plus le tube est long et fin plus il est difficile de faire passer de
l'air dans ce dernier. On peut exrpimé cela par une équation
correspondant à la Loi de Poiseuille :

\[ P_{atm}-P_{A}=R\dot{V} \]

Par ailleurs le volume du « ballon » dépend de l'élastance des tissus
(notée E) et des forces qui lui sont appliquées (ici la pression
alvéolaire \(P_{A}\) et la pression \(P\) exercée par le diaphragme).
Cela s'exprime par une autre équation traduisant l'équilibre élastique
du ballon :

\[ P_{A}-P=EV \]

On en déduit en combinant les deux équations, un modèle décrivant
l'évolution du volume du ballon au cours tu temps :

\[ P_{atm}-P=R \dot{V}+EV \]

Il s'agit d'une équation différentielle puisqu'elle fait intervenir à la
fois le volume \(V\) et sa variation \(\dot{V}\), linéaire, et on sait
en calculer explicitement la solution.

\pagebreak

\begin{center}\rule{0.5\linewidth}{\linethickness}\end{center}

\hypertarget{troisieme-partie---capnographie}{%
\section{Troisième Partie -
Capnographie}\label{troisieme-partie---capnographie}}

\hypertarget{histoire-de-la-capnographie}{%
\subsection{Histoire de la
capnographie}\label{histoire-de-la-capnographie}}

En 1620, le chimiste belge Van Helmont décrit pour la première fois la
molécule CO2. En 1754, le Dr Black publie dans sa thèse que le CO2 est
expiré dans l'air. Le CO2 est considéré comme toxique et mortel jusqu'en
1824. Il faudra attendre l'américain Henderson en 1925 pour démontrer
l'intérêt physiologique du dioxyde de carbone dans le métabolisme. En
1928, le CO2 est associé aux narcotiniques dans les anesthésies
humaines, mais les patients convulsaient au réveil, raison pour laquelle
cette méthode a été abandonnée. Luft développe le principe de la
capnométrie pendant la seconde guerre mondiale en partant du principe
que le CO2 est un gaz qui absorbe les rayons
infrarouges{[}\protect\hyperlink{ref-luft1943neue}{17}{]}{[}\protect\hyperlink{ref-bhavani1992capnometry}{18}{]}.
Il dépose deux brevets en 1943 pour des applications pratiques cliniques
: usine de caoutchouc et surveillance environnementale dans les
sous-marins. C'est en 1950 qu'Etan et Liston introduisent le concept de
monitorage du CO2 expiré pendant l'anesthésie par absorption des
infrarouges. Les rayons IR sont absorbés par tous les gaz composés de
plus de deux atomes par molécule, ou par deux atomes différents lorsque
la molécule est composée d'uniquement deux atomes. Cependant, à cause du
poids, du volume et du coût, ces appareils ne sont pas utilisés en
clinique courante. Puis une méthode de capnographie basée sur la
spectrographie photoacoustique est développée au Danemark. Elle repose
sur la détection au microphone des sons produits par la lumière pulsée
en fonction des longueurs d'onde des IR. Le spectromètre de masse, basé
sur la séparation des gaz en fonction de leur masse moléculaire, peut,
de façon séquentielle, échantillonner tous les gaz inspirés et expirés.
Au début du XXe siècle, les chercheurs ont exploré la nature de la
courbe expiratoire du dioxyde de carbone, y compris les composants
physiologiques comme l'espace mort et les gaz alvéolaires, autant par la
méthode de capnographie standard (temporelle) qu'avec la capnographie
volumétrique par échantillonnage séquentiel rapide du volume de
gaz(1){[}\protect\hyperlink{ref-aitken1928fluctuation}{19}{]}. Dans les
années 1950 et les années 1960, les premières études ont examiné la
forme de la courbe de capnographie normal et ont recherché aux niveaux
du CO2 alvéolaire un modèle utilisant des équations de dilution
dépendantes du temps
(2)(3){[}\protect\hyperlink{ref-chilton1952mathematical}{20}{]}{[}\protect\hyperlink{ref-yamamoto1960mathematical}{21}{]}.
Cependant, ce n'est qu'en attendant la disponibilité des systèmes
commerciaux des analyseurs infrarouges rapides que la détermination de
la pression partielle du dioxyde de carbone dans la respiration a pu
passer du laboratoire de recherche de physiologie au « chevet » du
patient(4){[}\protect\hyperlink{ref-jaffe2008infrared}{22}{]}. Cette
technologie a d'abord trouvé une application dans l'anesthésie et plus
tard aux soins intensifs pour un nombre limité d'indications. Ce n'est
que récemment que cette technologie a fait son apparition dans d'autre
spécialité comme les urgences s'ouvrant à de nouvelles applications
cliniques. C'est à la fin des années 1950 que les chercheurs ont
commencé à appliquer des approches tout d'abord manuelles puis
informatiques pour l'ajustement du dioxyde de carbone et l'étude des
courbes d'écoulement. Ces courbes de CO2 en fonction du temps encore
appelées capnographie standard et rapporté au volume expiré encore
appelées capnographie volumétrique ont permis l'estimation de paramètres
ventilatoires comme la PCO2 alvéolaire
(5){[}\protect\hyperlink{ref-bellville1959respiratory}{23}{]}. Ces
travaux ont été initialement entrepris avec des méthodes manuelles
consommant du temps
(6){[}\protect\hyperlink{ref-berengo1961single}{24}{]} puis ont été
rapidement suivis par des méthodes analogiques
(5){[}\protect\hyperlink{ref-bellville1959respiratory}{23}{]},
analogiques / numériques combinés
(7){[}\protect\hyperlink{ref-murphy1899analogue}{25}{]} et, plus tard,
avec des données numériques coûteuses et complexes
(8){[}\protect\hyperlink{ref-noe1963computer}{26}{]}. Les premiers
capnographes s'appuyaient sur des méthodes électroniques analogiques
pour déterminer les paramètres de la fréquence respiratoire (FR) et de
la pression partielle de CO2. L'introduction du microprocesseur au début
des années 1970 a conduit à des logiciels basés sur la mesure de la
fréquence respiratoire et de la PETCO2, avec un fabricant intégrant le
CO2 et le volume pour calculer la ventilation alvéolaire et
l'élimination du dioxyde de carbone. Cependant, malgré le développement
de technologie informatique pour identifier et classer les changements
en temps réel du capnogramme dès la fin des années 1980
(9)(10){[}\protect\hyperlink{ref-bao1992expert}{27}{]}{[}\protect\hyperlink{ref-ventzas1994capnex}{28}{]},
les algorithmes logiciels des dispositifs commerciaux ne permettent pas
à l'heure actuelle l'analyse de la courbe en clinique courante. Ces
dispositifs restent concentrés sur une estimation robuste du taux de CO2
en fin d'expiration (PETCO2) et de la fréquence respiratoire.

\hypertarget{technique-denregistrement-de-capnographie}{%
\subsection{Technique d'enregistrement de
capnographie}\label{technique-denregistrement-de-capnographie}}

\hypertarget{detection-du-co2-expire}{%
\subsubsection{Détection du CO2 expiré}\label{detection-du-co2-expire}}

Il existe différents systèmes pour détecter et analyser le CO2 expiré.

\hypertarget{la-spectrometrie-de-masse-11bhavani1992capnometry}{%
\paragraph{\texorpdfstring{La spectrométrie de masse
(11){[}\protect\hyperlink{ref-bhavani1992capnometry}{18}{]}}{La spectrométrie de masse (11){[}18{]}}}\label{la-spectrometrie-de-masse-11bhavani1992capnometry}}

Elle est basée sur le concept de séparation des ions en fonction de leur
poids moléculaire et de leur charge. Il existe deux types de
spectromètre de masse, celui à secteur magnétique et détecteurs fixés et
le quadripôle. Un champ électrique charge les molécules de gaz et les
transforme en cations qui seront accélérées et analysées. Cette
technique a pour avantages d'analyser plusieurs gaz en même temps, et
voir le CO2 expiré de plusieurs patients en même temps dans différentes
salles opératoires (maximum 31). Mais le spectromètre de masse est un
appareil volumineux et cher qui ne trouve pas son utilité au pied du lit
du malade.

\hypertarget{la-spectrographie-raman}{%
\paragraph{La spectrographie Raman}\label{la-spectrographie-raman}}

Elle a été inventée en 1928 par Mr Raman. Un rayon laser d'argon
monochromatique de haute intensité traverse un échantillon de gaz. Les
molécules de gaz dispersent la lumière avec des longueurs d'ondes
caractéristiques, qui sera collectée par des lentilles. Différents
filtres sélectionnent alors la lumière en fonction des gaz mesurés. Le
nombre « d'impacts » par 100 ms est utilisé pour déterminer la
concentration des gaz. Cette technique permet une analyse multi gaz,
rapide.

\hypertarget{methode-colorimetrique}{%
\paragraph{Méthode colorimétrique}\label{methode-colorimetrique}}

Il s'agit d'une méthode récente peu utilisée en raison de l'absence d'un
monitoring continu et d'affichage graphique des variations de CO2. C'est
un système qui détecte la présence de CO2 dans un gaz à l'aide d'un
papier filtre ou un autre révélateur sensible au pH.

\hypertarget{la-spectrophotometrie-dabsorption-infrarouge}{%
\paragraph{La spectrophotométrie d'absorption
infrarouge}\label{la-spectrophotometrie-dabsorption-infrarouge}}

La spectrophotométrie par rayonnement infrarouge (IR) est la technique
de capnographie actuellement la plus utilisée en médecine
(11){[}\protect\hyperlink{ref-bhavani1992capnometry}{18}{]}. Un faisceau
IR traverse un mélange gazeux, l'intensité de la lumière transmise est
fonction de la longueur de la cellule d'absorption, de la pression
partielle du gaz et de son coefficient d'absorption. Chaque gaz absorbe
plus spécifiquement une longueur d'onde donnée, il est donc possible de
mesurer l'absorption d'un gaz spécifique, qui sera proportionnelle à la
concentration. Le CO2 absorbe les IR à une longueur d'onde de 4.3 µm. En
comparant la différence d'absorption d'un rayonnement IR qui traverse
deux cellules (une dont la concentration en CO2 est connue, l'autre dont
la concentration reste à mesurer), il est possible de mesurer la
concentration en CO2 du mélange.

\hypertarget{systeme-danalyse-de-letco2}{%
\subsubsection{Système d'analyse de
l'ETCO2}\label{systeme-danalyse-de-letco2}}

On distingue deux types de système d'analyse de l'ETCO2
(12){[}\protect\hyperlink{ref-yosefy2004end}{29}{]}.

\hypertarget{systeme-non-aspiratif-mainstream}{%
\paragraph{Système non aspiratif : « Mainstream
»}\label{systeme-non-aspiratif-mainstream}}

Le système non aspiratif, mainstream, où la mesure est effectuée
directement sur le flux gazeux du circuit respiratoire du patient,
permet de fournir une analyse rapide et sans distorsion de la
concentration en CO2. La mesure se fait en continu, sans décalage
temporel. L'inconvénient de ce système réside dans le fait que le
capteur alourdit le dispositif et risque de modifier la position de la
sonde d'intubation chez les patients en ventilation invasive.

\hypertarget{systeme-aspiratrif-sidestream}{%
\paragraph{Système aspiratrif : « Sidestream
»}\label{systeme-aspiratrif-sidestream}}

Le système sidestream est un système aspiratif où l'échantillon de gaz
est aspiré à débit constant à travers un tube fin pour être analysé par
une cellule de mesure. Le débit aspiratif est de 150ml/min. Il permet
une analyse multi gaz en utilisant plusieurs longueurs d'ondes. Ce
capnographe a l'avantage d'être léger mais les sécrétions pulmonaires et
l'eau de condensation ont tendance à obstruer le tube d'analyse. Toute
fuite sur le circuit provoque une aspiration d'air ambiant et crée une
mesure anormalement basse du CO2 expiré. Par ailleurs, le délai
d'analyse de l'ETCO2 est allongé. Les progrès technologiques ont permis
de faire un système microstream, qui est un système aspiratif amélioré
où le CO2 transite par une fine tubulure avant d'atteindre la chambre de
mesure. Ce dispositif permet de diminuer le temps d'analyse et permet
ainsi une synchronisation entre flux expiré et courbe capnographique.

Actuellement les systèmes de mesure sont compacts, petits, légers et
faciles d'utilisation. Pour une utilisation clinique, les capnomètres
doivent pouvoir mesurer une concentration de CO2 dans une gamme de 0 à
10 \% (76 mmHg). La précision avec un système IR est d'environ 0.2\%, ce
qui est suffisant en clinique. Les résultats sont faussés par la
présence de protoxyde d'azote ou d'oxygène mais le système est conçu
pour annuler automatiquement ces biais. Par ailleurs, les variations de
température ambiante et de pression atmosphérique peuvent intervenir sur
les résultats de concentration en CO2 notamment sur les systèmes
sidestream.

\hypertarget{capnographie-et-pathologie-respiratoire}{%
\subsection{Capnographie et pathologie
respiratoire}\label{capnographie-et-pathologie-respiratoire}}

Il existe deux types de capnogrammes, soit il s'agit d'une mesure du CO2
expiré en fonction du temps (capnographie temporelle ou standard), soit
les mesures sont réalisées en fonction du volume expiré (courbes SBT-CO2
ou capnogramme volumétrique).

\hypertarget{la-capnographie-temporelle-ou-sandard}{%
\subsubsection{La capnographie temporelle ou
sandard}\label{la-capnographie-temporelle-ou-sandard}}

Le terme « capnographie standard », c'est-à-dire la capnographie basé
sur le temps, se réfère généralement au tracé du dioxyde de carbone en
fonction du temps par opposition à la capnographie en fonction du volume
expiré volumétrique appelé « capnographie volumétrique » où la phase
inspiratoire de la courbe n'est pas représentée. Les deux formes d'onde
sont subdivisées en trois phases expiratoires avec une quatrième
inspiratoire pour la capnographie standard)
(13)(14){[}\protect\hyperlink{ref-jabre2010place}{30}{]}{[}\protect\hyperlink{ref-howe2011use}{31}{]}
\emph{figure 1}:

\begin{itemize}
\tightlist
\item
  Phase I : appelée base inspiratoire, elle correspond à la vidange de
  l'espace mort anatomique, la pression partielle en CO2 est proche de 0
  puisque le mélange gazeux de cet espace n'a pas été en contact avec
  les échanges gazeux alvéolaires. Sa composition est comparable à celle
  de l'air inspiré avec peu ou pas de CO2.
\item
  Phase II : de montée expiratoire, on observe une augmentation rapide
  du CO2 expiré, sous forme d'une courbe sigmoïde, correpsondant à la
  transition entre l'espace mort anatomique et l'arrivée des gaz
  alvéolaires. Durant cette étape, le gaz de l'espace mort anatomique
  est remplacé par le gaz alvéolaire. La verticalité de la phase II est
  en faveur d'une distinction nette entre l'espace mort anatomique et
  l'air alvéolaire. L'élimination de l'air alvéolaire est normalement
  synchrone, ce qui est prouvée par l'ascension soudaine de cette phase
  (14){[}\protect\hyperlink{ref-howe2011use}{31}{]}.
\item
  Phase III : on parle de plateau expiratoire, provenant de la vidange
  séquentielle des volumes alvéolaires. Elle correspond à l'expiration
  du CO2 provenant uniquement des alvéoles. Cette phase est en plateau
  voir légèrement ascendant et varie en fonction des pathologies
  respiratoires. Une majoration de la pente correspond à une vidange
  asynchrone des alvéoles lors de l'expiration, ce qui est visible dans
  les pathologies obstructives comme l'asthme et la BPCO. La fin du
  plateau correspond à la PETCO2, affichée sur les écrans de monitoring.
  La PETCO2 est un reflet de la pression artérielle en CO2 chez un sujet
  sain avec un métabolisme stable.
\item
  Phase 0 : c'est l'inspiration d'un gaz dépourvu de CO2 qui provoque
  une chute brutale du CO2 jusqu'à la ligne de base (c'est-à-dire 0
  mmHg). A la phase 0 succède la phase I. Néanmoins, il est parfois
  difficile de faire la distinction entre fin d'expiration et début
  d'inspiration.
\end{itemize}

Les phases II et III sont séparées par un angle alpha, décrit dans la
littérature, dont la norme est comprise entre 100° et 110°. Cet angle
augmente lorsque la pente de la phase III s'accentue et que la phase II
présente une croissance plus progressive. L'angle beta sépare les phases
III et 0, il est proche de 90°. Il permet d'estimer l'importance du
phénomène de réinhalation. C'est ainsi que les différentes phases se
succèdent sur le capnogramme, au fil des mouvements respiratoires.
L'analyse du capnogramme est un élément important dans la surveillance
du patient, aussi bien pour les valeurs numériques que pour l'analyse
des quatre phases et de la modification de l'aspect des courbes
(14){[}\protect\hyperlink{ref-howe2011use}{31}{]}. Il existe deux
vitesses de balayage des courbes. Une lente, avec une vitesse de 25
mm/min, qui montre les tendances générales de la concentration en CO2
expiré. Une rapide, qui défile à 12.5 mm/sec soit 750 mm/min, et qui
permet l'analyse des détails d'un cycle respiratoire.

\begin{figure}[h!]

{\centering \includegraphics[width=250px]{figure/courbe_capno} 

}

\caption{Courbe de Capnographie Standard et ses différentes phases.}\label{fig:unnamed-chunk-5}
\end{figure}

\hypertarget{capnographie-volumetrique}{%
\subsubsection{Capnographie
Volumétrique}\label{capnographie-volumetrique}}

La capnographie volumétrique correspond à la PCO2 mesurée et exprimée en
fonction du volume lors de la phase d'expiration. La terminologie
utilisée pour les différentes phases, les transitions de phase, les
angles et les pentes sont similaires pour la capnographie standard et la
capnographie volumétrique. Le capnogramme volumétrique a été subdivisé
en volume pour chaque phase
(15){[}\protect\hyperlink{ref-rayburn1994neural}{32}{]} et associé à des
fonctions (16){[}\protect\hyperlink{ref-tusman2011capnography}{33}{]}
pour mieux caractériser et de manière plus fiable la forme d'onde ainsi
que produire des estimations dérivées avec une plus grande
reproductibilité. L'analyse de la courbe de capnographie volumétrique
permet d'estimer un certain nombre de mesures physiologiques et
physiquement interprétables de la fonction respiratoire, telles que
l'espace mort anatomique et l'espace mort physiologique ainsi que le
rapport des deux, l'élimination du CO2 et le flux sanguin capillaire
pulmonaire. Ces mesures permettent de découvrir de nombreux troubles
cardio-respiratoires comme le syndrome de détresse respiratoire aiguë,
la BPCO, l'asthme et l'embolie pulmonaire. Ces paramètres
morphologiques, qui divisent le capnogramme volumétrique en phases I-III
(17){[}\protect\hyperlink{ref-fletcher1981concept}{34}{]}, permet une
caractérisation et une interprétation accrues du capnogramme
volumétrique à l'aide d'un certain nombre de caractéristiques dérivées.
L'un des intérêts de cette analyse est de permettre une mesure
substitutive non invasive pour l'espace mort anatomique, l'espace mort
des voies aériennes, estimé par la méthode de Fowler
(18){[}\protect\hyperlink{ref-fowler1948lung}{35}{]} qui utilise les
parties centrales '' linéaires '' ajustées en phase III (par exemple, 40
à 80\% du volume expiré) et la phase II. Cette mesure permet de calculer
l'espace mort des voies respiratoires sur une base respiratoire pour
chaque patient plutôt que de compter sur la règle plus approximative de
1 mL/kg d'espace mort en fonction du poids. Brewer et al.
(19){[}\protect\hyperlink{ref-brewer2008anatomic}{36}{]} a constaté que
cette règle, dérivée des moyennes de population, par rapport à l'espace
mort mesuré de la voie aérienne, ne présentait aucune corrélation (r2 =
0,0002). Au fil du temps, les chercheurs s'inquiètent de l'effet des
variations de mesure sur les estimations du volume de l'espace mort de
la méthode Fowler et de la variation résultante de ces paramètres
calculés. Cela a conduit au développement de méthodes à faible
dispersion, qui comprenaient l'ajustement du volume expiratoire de CO2
par rapport à la courbe de volume expiratoire entre 40 et 80\% du volume
expiré en utilisant des polynômes de première, seconde et troisième
ordre (20){[}\protect\hyperlink{ref-tang2007systematic}{37}{]}. Plus
récemment, un ajustement rapide du capnogramme volumétrique à l'aide
d'une approximation fonctionnelle déterminée avec un algorithme non
linéaire de moindres carrés a été utilisé
(16){[}\protect\hyperlink{ref-tusman2011capnography}{33}{]}, ce qui a
entraîné une diminution de la variabilité respiratoire intra-patient
ainsi que la dispersion des paramètres calculés. Les caractéristiques
discutées dans cette section sont destinées à être interprétées en
utilisant des concepts physiologiques ou cliniques connus.

++++++ (Corrections of Enghoff's dead space formula for shunt effects
still overestimate Bohr's dead space,
{[}\protect\hyperlink{ref-suarez2013corrections}{38}{]})
Conceptuellement, l'inefficacité de la ventilation chez les patients
sous ventilation mécanique est évaluée par l'espace mort. Elle est
définie comme la partie de la ventilation qui ne participe pas aux
échanges gazeux (Bohr, 1891; Fletcher et Jonson, 1981; Bartels et al.,
1954; Fletcher, 1985). Toute augmentation de l'espace mort réduit la
clairance du dioxyde de carbone (CO2), ce qui rend les patients
prédisposés à l'hypercapnie, ce qui doit être compensé par une
augmentation de la ventilation totale. En 1891, Bohr propose une formule
pour calculer le rapport espace mort / volume courant (Vd / Vt) en
utilisant du CO2 expiré mixte (Bohr, 1891). Il a postulé que le degré de
dilution du CO2 dans les poumons, déterminé par la différence entre les
pressions partielles alvéolaires moyennes (PACO2) et mixtes expirées
(PEtCO2), était proportionnel à la quantité d'espace mort. En raison des
difficultés pratiques rencontrées pour mesurer la pression partielle
alvéolaire de CO2 (PACO2), Enghoff a modifié la formule originale de
Bohr en utilisant artériel comme substitut de la PCO2 alvéolaire
(PaCO2), en se référant au concept de Riley d'un poumon idéal (Enghoff,
1938; Riley et Cournand. , 1949, 1951). Cependant, l'utilisation de
PaCO2 nécessite non seulement un prélèvement de gaz de sang artériel
invasif, mais est également affectée par la présence d'un shunt de
droite à gauche. Lorsque les concentrations élevées de CO2 dans le sang
veineux atteignent le côté artériel via les voies de dérivation
intrapulmonaires, PaCO2 augmente au-dessus de PACO2. Cela fait de PaCO2
un paramètre inapproprié pour calculer l'espace mort, en particulier à
des niveaux de shunt élevés (Fletcher et Jonson, 1981; Suter et al.,
1975; Wagner, 2008). Pour la surveillance des patients ventilés
mécaniquement, les sacs de Douglas traditionnels sont désormais
remplacés par la capnographie volumétrique (VCap), la courbe du CO2
expiré sur le volume courant (Fig. 1) (Fletcher et Jonson, 1981; Sinha
et Soni, 2012). Bien que l'on puisse facilement estimer PEtCO2 à partir
de VCap, on pensait qu'il n'était pas possible d'en déduire des
estimations précises de PACO2 (Fletcher et Jonson, 1981; Enghoff, 1938).
Cependant, nous avons récemment introduit et validé une méthode basée
sur VCap pour estimer PACO2. Il permet une évaluation non invasive,
souffle après souffle, de l'espace mort de Bohr (Tusman et al., 2011a,
b). Plusieurs auteurs ont décrit des approches mathématiques pour
corriger l'équation de shunt d'Enghoff (Kuwabara et Duncalf, 1969;
Mecikalski et al., 1984; Torda et Duncalf, 1974; Niklason et al., 2008).
Cependant, ces corrections ont été développées sur des bases théoriques
uniquement. À notre connaissance, la validation de tels algorithmes à
l'aide de données biologiques réelles fait encore défaut. Nous avons
émis l'hypothèse que si les espaces morts d'Enghoff étaient corrigés du
shunt, ils devraient donner des valeurs similaires à celles obtenues par
l'équation originale de Bohr. Le but de ce travail était de vérifier
cette hypothèse dans une cohorte d'animaux à ventilation mécanique, dans
laquelle différents degrés de shunts et d'espaces morts étaient induits
par des lavages pulmonaires et une large gamme de pressions expiratoires
aux extrémités positives (PEEP).

++++++

Espace Mort Physiologique - \(V_{D_{phys}}\)
\[V_{D_{phys}}=V_{D_{aw}}+V_{D_{alv}}\]

Equation de BOHR : Christian Bohr a été le premier à décrire l'espace
mort en 1891. Correspond au rapport du Volume de l'Espace Mort
Physiologique (\(V_{D_{phys}}\)) par rapport au volume courant \(V_{T}\)
\[\frac{V_{D}}{V_{T}}=\frac{(P_{A}CO2-P\overline{E}CO2)}{P_{A}CO2}\]

Equation de ENGHOFF : Enghoff a remplacé la \(P_{A}CO2\) par la
\(P_{a}CO2\), cette simplification pratique et brillante est devenu la
méthode de référence au lit du malade. Néamoins, elle reste une
technique invasive avec la nécessité d'une gazométrie artérielle pour
recueillir la valeur de \(P_{a}CO2\). En considéréant que la
\(P_{A}CO2\) est égale à la \(P_{a}CO2\) on en déduit :
\[\frac{V_{D}}{V_{T}}=\frac{(P_{a}CO2-P\overline{E}CO2)}{P_{a}CO2}\]

Equation de FLETCHER : L'introduction de VCap par Fletcher et al.~ont
encore simplifié la mesure clinique de l'espace mort. Fletcher et al.
{[}16{]} {[}\protect\hyperlink{ref-fletcher1981concept}{34}{]} suggèrent
que PACO2 pourrait théoriquement être obtenu à partir de la partie
médiane de la phase III du capnogramme volumétrique, une idée confirmée
par la suite par Breen et al.~Heureusement, le concept d'obtention de
PACO2 à partir de la moitié de la phase III a récemment été validé par
rapport à la technique d'élimination multiple des gaz inertes (MIGET)
{[}19{]}{[}\protect\hyperlink{ref-tusman2011validation}{39}{]}.
Maintenant que cette limitation majeure a été surmontée, il est possible
d'obtenir des mesures fiables et de manière non-invasive, de l'espace
mort physiologique (VDPhys) par Capnographie Volumétrique (VCap). Bien
que la plupart des méthodes actuelles s'appuient sur l'approche
géométrique de Fowler, la forme du capnogramme peut grandement affecter
sa précision. Par conséquent, de nouvelles approches mathématiques ont
été proposées pour améliorer la robustesse de la détermination de VDaw
{[}35{]}{[}\protect\hyperlink{ref-tusman2009model}{40}{]}. Une fois que
VDphys et VDaw sont connus, VDalv se calcule facilement en soustrayant
simplement VDaw de VDphys.

Principales Caractéristiques de la Capnographie Volumétrique :

\begin{itemize}
\tightlist
\item
  \(P_{ET}CO2\) : pression partielle de CO2 en fin d'expiration
\item
  \(P_{A}CO2\) : pression partielle de CO2 alvéolaire
\item
  \(P\overline{E}CO2\) : pression partielle moyenne de CO2 expiré
\item
  \(V_{D_{phys}}\) : volume de l'espace mort physiologique
\item
  \(V_{D_{aw}}\) : volume de l'espace mort anatomique
\item
  \(V_{D_{alv}}\) : volume de l'espace mort alvéolaire
\item
  \(V_{T}CO2\) : aire sous la courbe correspond à
\item
  \(V_{Talv}\) ou \(V_{T_{alv}}\) : volume alvéolaire (ou volume courant
  alvéolaire)
\item
  \(S_{II}\) :
\item
  \(S_{III}\) :
\item
  \(V_{D_{alv}}/V_{T_{alv}}\) : fraction efficace du volume courant
  alvéolaire (ou plutôt fraction inefficace) correspond à un indice
  d'efficacité du compartiement alvéolaire
\end{itemize}

\hypertarget{pathologies-respiratoires-et-courbes-de-capnographie}{%
\subsubsection{Pathologies respiratoires et courbes de
capnographie}\label{pathologies-respiratoires-et-courbes-de-capnographie}}

L'analyse des données fournies par la capnographie doit être réalisée de
manière systématique. Une variation de la PETCO2 doit tenir compte des
modifications de la forme du capnogramme et des trois éléments qui
interviennent sur la PETCO2, à savoir la production métabolique de CO2,
son transport en fonction du débit sanguin et son élimination par la
ventilation.

\hypertarget{intubation-oesophagienne}{%
\paragraph{Intubation oesophagienne}\label{intubation-oesophagienne}}

Affirmer que la sonde d'intubation se trouve dans la trachée est souvent
difficile, et l'auscultation pulmonaire ne permet pas toujours de faire
la part des choses, notamment chez les sujets obèses. C'est pourquoi la
vérification de la position de la sonde par le suivi de la courbe de
capnographie est devenue une pratique courante en pré-hospitalier et en
anesthésie(21){[}\protect\hyperlink{ref-jung2008modalites}{41}{]}. En
cas d'intubation oesophagienne, on observera soit l'absence de courbe de
capnographie, soit quelques courbes dont l'amplitude va décroître
progressivement. Quand celles-ci sont visibles, elles reflètent le gaz
entré dans l'estomac lors de la ventilation non invasive au BAVU, ou
l'ingestion au préalable de traitements anti-acides, carbonatés ou
effervescents créant une réaction chimique avec production de CO2
intra-gastrique. Mais ce CO2 est rapidement évacué par la ventilation,
provoquant une chute de l'EtCO2 et un aplatissement des courbes. Il est
donc recommandé d'attendre 6 cycles respiratoires avant de confirmer
l'intubation endotrachéale.

\hypertarget{intubation-selective}{%
\paragraph{Intubation sélective}\label{intubation-selective}}

La bronche souche droite forme un angle de 25° avec la trachée, elle est
donc pratiquement dans son axe, ce qui explique que dans la majorité des
cas d'intubation sélective accidentelle, la sonde se trouve dans la
bronche droite. Lors d'une intubation sélective, il existe deux
phénomènes. Le premier est lié à une augmentation de l'espace mort, un
élargissement du gradient alvéolo-artériel en CO2 et une baisse de la
PETCO2. Le second est une diminution de la ventilation alvéolaire, une
augmentation de la PaCO2 et donc de la PETCO2 au niveau de la bronche
non intubée(21){[}\protect\hyperlink{ref-jung2008modalites}{41}{]}.

\hypertarget{embolie-pulmonaire-2223kline1998preliminarywiegand2000effectiveness}{%
\paragraph{\texorpdfstring{Embolie pulmonaire
(22)(23){[}\protect\hyperlink{ref-kline1998preliminary}{42}{]}{[}\protect\hyperlink{ref-wiegand2000effectiveness}{43}{]}}{Embolie pulmonaire (22)(23){[}42{]}{[}43{]}}}\label{embolie-pulmonaire-2223kline1998preliminarywiegand2000effectiveness}}

L'embolie pulmonaire augmente l'espace mort alvéolaire, entraînant une
diminution du CO2 expiré par dilution et par conséquent une PETCO2
inférieure à des valeurs normales. Le gradient alvéolo-artériel et par
conséquent la P(a-ET)CO2 augmente. Plusieurs études ont été menées pour
discuter de l'intérêt de la mesure du CO2 expiré dans le diagnostic de
l'embolie
pulmonaire(13)(22){[}\protect\hyperlink{ref-jabre2010place}{30}{]}{[}\protect\hyperlink{ref-kline1998preliminary}{42}{]}.
Une des pistes de travail est d'exclure ce diagnostic quand le patient a
une probabilité clinique faible et un gradient P(a-ET)CO2 normal malgré
des D-dimères positifs. Mais plus récemment, c'est l'aire sous la courbe
de capnographie qui est utilisée pour diagnostiquer l'EP. Celle-ci
diminue quand le diagnostic est confirmé. L'aire moyenne sous la courbe
de patients ayant une EP est d'environ la moitié moindre de celle des
patients sains. Ces données suggèrent qu'un patient présentant des
symptômes pouvant évoquer une EP avec courbe de capnographie ayant une
faible aire sous la courbe doit être considéré comme étant à haut
risque.

\hypertarget{trouble-ventilatoire-obstructif-asthme-et-bpco}{%
\paragraph{Trouble ventilatoire obstructif : Asthme et
BPCO}\label{trouble-ventilatoire-obstructif-asthme-et-bpco}}

L'asthme se caractérise par une hyperréactivité bronchique, une
inflammation et un remodelage bronchique
(24){[}\protect\hyperlink{ref-ozier2011pivotal}{44}{]}. On observe
essentiellement une diminution du calibre des voies aériennes appelée
obstruction bronchique. Le bronchospasme modifie l'allure du capnogramme
avec une augmentation de la pente en phase III. Le bronchospasme
entraîne une diminution de la ventilation alvéolaire sans diminution de
la perfusion pulmonaire ce qui diminue le rapport ventilation-perfusion
(V/Q). Or, la réduction de diamètre des bronches n'est pas homogène dans
l'ensemble poumon, certaines zones étant plus spastiques que d'autres,
de ce fait on assiste également à une diminution du rapport V/Q
alvéolaire local. On parle d'hétérogénéité en parallèle du rapport V/Q
alvéolaire(25){[}\protect\hyperlink{ref-hisamuddin2009correlations}{45}{]}.
Cela cause une déformation du capnogramme marquée par une
(14){[}\protect\hyperlink{ref-howe2011use}{31}{]} \emph{figure 2}:

\begin{itemize}
\tightlist
\item
  Diminution de la verticalité en phase II
\item
  Ouverture de l'angle alpha
\item
  Augmentation de la pente en phase III
\end{itemize}

Dans les cas sévères, cette courbe prend une forme triangulaire. La
pente en phase III et l'angle alpha sont associés au degré de sévérité
de la crise d'asthme, plus ceux-ci sont importants, plus le
bronchospasme est majeur
(26){[}\protect\hyperlink{ref-langhan2008quantitative}{46}{]}. Des
études ont montré que l'ETCO2 des patients faisant une crise d'asthme
est plus bas que les sujets sains
(35)(26){[}\protect\hyperlink{ref-den2006bayesian}{47}{]}{[}\protect\hyperlink{ref-langhan2008quantitative}{46}{]}.
Dans ce contexte la PETCO2 n'est plus le reflet de la pression
alvéolaire en CO2 du fait d'un trapping alvéolaire important. On observe
une augmentation du temps expiratoire, ce qui entraîne une augmentation
du gradient P(a-ET)CO2. Une étude américaine a montré que le rapport
dCO2/dt, qui reflète la pente du plateau alvéolaire du capnogramme,
permet de détecter l'obstruction
bronchique(27){[}\protect\hyperlink{ref-yaron1996utility}{48}{]}.
L'auscultation pulmonaire avec la recherche de sibilants ne permet pas
toujours de prédire la présence ou non d'une obstruction bronchique, et
la sévérité des sibilants n'est pas forcément corrélée au diamètre des
voies aériennes, d'où l'intérêt d'une mesure objective
(28){[}\protect\hyperlink{ref-egleston1997capnography}{49}{]}. Krauss et
al.(29){[}\protect\hyperlink{ref-krauss2005capnogram}{50}{]} ont montré
dans les troubles obstructifs de type asthme et BPCO, une déformation de
la courbe de capnographie caractéristique en forme d'« aileron de requin
» ou « shark's fin » en anglais. Ces modifications chez les malades de
pathologie pulmonaire obstructive (BPCO et asthme) sont marquées par une
courbure importante
{[}5{]}{[}\protect\hyperlink{ref-you1994expiratory}{51}{]},
{[}10{]}{[}\protect\hyperlink{ref-smalhout1975atlas}{52}{]},
{[}11{]}{[}\protect\hyperlink{ref-kelsey1962expiratory}{53}{]} et
résulte de l'arrivée décalée du gaz alvéolaire
{[}3{]}{[}\protect\hyperlink{ref-dubois1952alveolar}{54}{]}. Ces
modifications de courbe chez les patients obstructifs étaient corrélées
aux altérations retrouvées lors des mesures de spirométrie.
Comparativement, il y a peu de données sur la morphologie du capnogramme
chez les patients présentant une insuffisance cardiaque aiguë ou
d'autres maladies pulmonaires restrictives.

\begin{figure}[h!]

{\centering \includegraphics[width=250px]{figure/courbe_capno_patho} 

}

\caption{Courbe de Capnographie Standard normale (en noir) et pahtologique (en rose).}\label{fig:unnamed-chunk-6}
\end{figure}

\pagebreak

\hypertarget{analyse-darticle---bibliographie-sur-la-capnographie}{%
\subsection{Analyse d'article - Bibliographie sur la
Capnographie}\label{analyse-darticle---bibliographie-sur-la-capnographie}}

\hypertarget{utility-of-the-expiratory-capnogram-in-the-assessment-of-bronchospasm-yaron1996utility}{%
\subsubsection{\texorpdfstring{Utility of the expiratory capnogram in
the assessment of bronchospasm
{[}\protect\hyperlink{ref-yaron1996utility}{48}{]}}{Utility of the expiratory capnogram in the assessment of bronchospasm {[}48{]}}}\label{utility-of-the-expiratory-capnogram-in-the-assessment-of-bronchospasm-yaron1996utility}}

En 1996, Yaron et al. (27) ont conduit une étude prospective au service
des urgences en vue d'étudier l'intérêt de l'utilisation de la courbe de
capnographie pour l'évaluation du bronchospasme chez des patients
asthmatique. Cette étude incluait 20 patients asthmatiques et 28
patients sains. L'objectif de ce travail était de voir si la pente du
plateau alvéolaire (pente de phase III ou dCO2 / dt) du capnogramme
pouvait servir comme mesure non invasive du bronchospasme et était
indépendant de l'effort contrairement aux mesures de spirométries. Les
valeurs de Débit Expiratoire de Pointe (DEP), approche standard pour
évaluer l'obstruction des voies respiratoires aux urgences, et dCO2 / dt
ont été mesurées pour chaque sujet ainsi qu'avant et après traitement
bronchodilatateur. La mesure de la pente du plateau alvéolaire dCO2 / dt
était réalisée sur un cycle de cinq expirations régulières consécutives
avec un moyennage des valeurs pour chaque patient. Ils ont conclu que la
dCO2 / dt était une mesure indépendante de l'effort, rapide et non
invasive permettant une évaluation du bronchospasme chez les patients
souffrant d'asthme sévère. Ce paramètre était fortement corrélé au DEP
(r = 0,84, p \textless{} 0,001). Les résultats montraient également une
modification du DEP et de la dCO2 / dt après traitement par
bronchodilatateur. Le DEP passait de 58 à 74\% (p \textless{} 0.001) et
de 0.27\% à 0.19\% (p \textless{} 0.005) pour la dCO2/dt. Druck et al.
(31), travaillant avec Yaron, ont cherché à prolonger ce travail en
examinant la relation entre les modifications de DEP et le changement de
pente de la phase III en capnographie volumétrique pour des mesures
enregistrées avant et après bronchodilatateur. Le pourcentage de
changements entre le DEP et son substitut était associé à une forte
corrélation (r = 0.7, p \textless{} 0.2, n = 13). Comme pour le travail
de You et al., cette recherche définit un nouveau paramètre pour
analyser le capnogramme pour des patients asthmatiques. Cependant, la
mesure de cet indice a été réalisée manuellement avec un risque d'erreur
important et une difficulté pour son application en clinque courante.

\hypertarget{expiratory-capnography-in-asthma---evaluation-of-various-shape-indices-you1994expiratory}{%
\subsubsection{\texorpdfstring{Expiratory capnography in asthma -
evaluation of various shape indices
{[}\protect\hyperlink{ref-you1994expiratory}{51}{]}}{Expiratory capnography in asthma - evaluation of various shape indices {[}51{]}}}\label{expiratory-capnography-in-asthma---evaluation-of-various-shape-indices-you1994expiratory}}

Une des premières études fondamentales dans l'analyse de la capnographie
a été conduite par You et al (30). Ils ont examiné de nouveaux
paramètres en fonction des pentes et des ratios de surface pour mieux
caractériser la nature de la transition de la phase II à la phase III en
capnographie standard. Les corrélations entre indices capnographiques et
mesures spirométriques ont été calculées chez 10 sujets sains et 30
patients asthmatiques de sévérités différentes. Huit indices descriptifs
ont été évaluée en mesurant leur reproductibilité et leur sensibilité
sur l'évaluation de l'obstruction des voies aériennes. Trois indices
mesurais la pente de la courbe lors de la phase d'expiration (S1, S2 et
S3). Un index composite SR correspondant au ratio entre les pente de la
partie 1 et 2 de l'ascention du CO2 (S2/S1)x100 a été évalué. Un ratio
de surface AR correspondant au rapport des surface A1 et A2, surface
au-dessus d'un seuil de 2.5 \% de CO2 entre 0.2 et 1 seconde (AR :
(A1/A2) x100), a été calculé. Trois derniers indices ont été utilisé
correspondant à la deuxième dérivée de la courbe de capnographie au
niveau des segment S1, S2 et S3 (SD1, SD2 et SD3). Dans cette étude, une
mesure de la reproductibilité des 8 indices a été réalisé sur 14
patients (7 femmes et 7 hommes). Les 8 indices ont été mesurés et
comparés après 2 enregistrements successifs à 10 minutes d'intervalle
pour chaque patient. Les variabilités ``intra-individu'' (\(Vi\)) et
``inter-individu'' (\(VI\)) ont été calculées comme suit :

\[Vi=\frac{2(m1-m2)}{m1+m2}\]

\begin{equation}
  Vi=\frac{2(m1-m2)}{m1+m2}
\label{eq:Vi}
\end{equation}

et

\[VI=\frac{\frac{\sigma1}{m1}+\frac{\sigma2}{m2}}{2}\]

\begin{equation}
  VI=\frac{\frac{\sigma1}{m1}+\frac{\sigma2}{m2}}{2}
\label{eq:VI}
\end{equation}

Le rapport signal sur bruit de chaque indice a ensuite été caractérisé
en réalisant le rapport de la moyenne Vi sur la moyenne de VI sur
l'ensemble des patients, où \(\sigma1\) et \(\sigma2\) sont les
écarts-types des premières et secondes mesures du groupe.

Les résultats ont montré qu'une grande différence de corrélation parmi
l'ensembles des indices. Selon leur recherche, une haute sensibilité à
l'obstruction de voie aérienne a été vue pour des pentes intermédiaires
et terminales (S2 et S3) ainsi que pour le radio des pentes (SR), suivie
par les indices associés à la dérivées secondes de la courbes (SD1, SD2,
SD3). La corrélation la plus forte a été retrouvée entre le VEMS\%-pred
et le pic de dérivée seconde SD2 associé à une corrélation r = 0.93 (p
\textless{} 0.01). Les corrélations les plus faibles ont été retrouvé
pour les indices S1 et AR. Cette recherche a permis de mettre en
évidence des paramètres simples et utilisables, ainsi qu'une corrélation
forte entre indice de capnogrpahie et données spirométriques. Mais ces
paramètres restent du domaine temporel sans anlayse dans le domaine
fréquentiel. Il est à noter également que les mesures ont été réalisées
manuellement associé à un facteur d'erreur non négligeable, aussi
consommateur de temps avec une reproductibilité médiocre. Ce travail
reste néanmoins malgré quelques points discutables, une référence dans
l'analyse de la capnographie.

++++++++++++++++++++++++ De grandes différences ont été observées entre
les indices (tableau 1). Les variabilités intra-individuelles étaient
faibles (Vi \textless{}10\%), les indices décrivant la partie initiale
du capnogramme (S1, AR, SD1, SD2). Les résultats étaient médiocres pour
les indices intermédiaires (SD3, SR et S2) et les moins bons pour S3.
Une variabilité importante entre les sujets, suggérant une sensibilité
élevée à l'obstruction des voies respiratoires, a été observée pour les
pentes intermédiaires et terminales (S2, S3, SR), suivie de SD1, SD2,
SD3 et les plages les plus basses ont été observées avec S1 et SR. .
Pour Vi / VI qui est un indice du rapport bruit / signal, les meilleurs
résultats, c'est-à-dire les valeurs les plus basses, ont été obtenus
avec SR, AR et SD2; de bons résultats ont été trouvés avec S1, S2 et
SD1; et des résultats médiocres avec SD3 et S3.

Comparaison des sujets contrôles et asthmatiques. Les deux populations
de sujets ont pu être clairement distinguées sur la base des résultats
spirométriques (tableau 2). Alors que les sujets témoins avaient des
valeurs étroitement groupées autour de la moyenne (écart-type/moyenne,
\(\sigma\)/m = 7,4--16\%), les sujets asthmatiques présentaient des
différences interindividuelles marquées, reflétant une grande diversité
dans la sévérité du bronchospasme (\(\sigma\)/m = 42,5 à 61,2\%). Dix
patients asthmatiques présentaient des valeurs spirométriques normales
ou sous-normales, avec un VEMS prédictif compris entre 80 et 100\%, dix
autres modérés (VEMS = 40--79\% prédétés) et 10 signes d'obstruction
marqués (VEMS = 16--39\% prédictifs). L'analyse de la variance a montré
que tous les indices capnographiques étaient significativement
différents (p \textless{}0,001 dans tous les cas) entre les groupes
contrôle et asthmatique. Malgré la taille réduite des sous-groupes, la
plupart des indices étaient également différents entre les trois
sous-groupes, notamment entre A2 et A3.

Corrélations entre capnographie et spirométrie. En général, les indices
capnographiques étaient fortement corrélés aux paramètres spirométriques
(p \textless{}0,001 dans tous les cas) (tableau 1 et figure 2). Dans
certains cas, des corrélations légèrement meilleures ont été trouvées en
utilisant un ajustement semi-log. Les corrélations les plus élevées ont
généralement été observées avec FEV1\% pred plutôt qu'avec PEF\% pred.
Ceux avec FEV25--75\% pred étaient systématiquement inférieurs à ceux
des autres indices spirométriques. Les indices intermédiaires et
terminaux étaient souvent plus corrélés aux flux maximaux que les
indices initiaux (S1, AR, SD1). Les meilleurs résultats ont été obtenus
avec les indices moyennant un partie de la dérivée seconde, notamment
SD2. Pour les pentes, aucune amélioration n'a été constatée entre S2 et
S1.

DISCUSSION: KELSEY et OLDHAM {[}18{]} ont décrit quatre types de formes
dans les capnogrammes normaux et obstructifs. Depuis leur rapport,
diverses approches théoriques ont été utilisées pour quantifier ces
déformations {[}2, 7--9, 19, 20{]} et plusieurs indices capnographiques
ont été évalués dans le traitement de la bronchopneumopathie chronique
obstructive (COLD): 1) la pente du plateau alvéolaire qui peut être liée
à la PETCO2 {[}1, 10, 11{]}; 2) le rayon de courbure minimale de l'angle
Q {[}7--9, 12, 21{]}; 3) le temps nécessaire pour passer de 25 à 75\% de
la PETCO2 {[}4, 13{]}; et 4) l'angle entre ``E2'' et ``E3'' mesuré
manuellement ou calculé à partir du rapport de leurs pentes {[}14, 22,
23{]}. Dans toutes les études de validation {[}10--12, 14{]}, des
corrélations significatives ont été trouvées entre les mesures
spirométriques habituelles et les valeurs des indices capnographiques
utilisés. Quatre études ont porté sur le capnogramme chez les
asthmatiques. Les trois premiers {[}23--25{]} ont été menés chez des
enfants et ont montré des modifications des pentes E2 et E3 après une
provocation bronchique. La quatrième {[}16{]}, réalisée par nos soins, a
été réalisée chez des asthmatiques adultes. La valeur de la pente de fin
de marée (ETS), mesurée manuellement, a été comparée au VEMS\% préd. Une
bonne corrélation a été observée entre ces deux paramètres, mais cette
étude préliminaire a suggéré une informatisation de l'analyse par
capnogramme, puis l'évaluation de nouveaux indices ne présentant pas les
inconvénients rencontrés avec l'utilisation de l'ETS (voir ci-dessous).

Sélection de cycles D'un point de vue méthodologique, les irrégularités
physiologiques de la respiration nécessitent la sélection de cycles de
bonne qualité en fonction de critères de durée, d'amplitude et, si
possible, de régularité de la courbe. Dans cette étude, nous avons
systématiquement éliminé les cycles qui ne répondaient pas aux critères
suivants: 1) expiration durant entre 0,8 et 3 s; 2) amplitude maximale
supérieure à 3,5\%; et 3) bonne régularité de E2 et E3. De plus, dans
ces limites, certains indices exigent, selon leurs définitions, qu'un
cycle réponde à des exigences spécifiques de durée et de forme. Ces
exigences particulières sont responsables d'un taux de rejet
automatique, spécifique à chaque indice, et augmentant avec la sévérité
du bronchospasme (cycles courts empêchant la mesure des derniers
indices).

Facteurs influençant les données Certains indices peuvent être altérés
par des facteurs induisant une déformation du capnogramme non liée à une
obstruction bronchique. La forme du capnogramme peut dépendre en premier
lieu de facteurs méthodologiques, à savoir les caractéristiques
dynamiques de l'analyseur. À cet égard, il convient de distinguer
\emph{le temps de réponse du délai pur introduit par le cathéter de
prélèvement. Ce dernier ne modifie pas la forme du signal, mais un temps
de réponse long filtre ses composants rapides. Dans cette étude, le
temps de réponse à 90\% de l'instrument était de 250 ms, avec un débit
d'échantillonnage de 150 ml.min-1. En supposant un comportement de
premier ordre, cela correspond à une constante de temps d'environ 100
ms. Nous avons essayé de corriger numériquement les données de cette
réponse instrumentale et avons étudié l'influence de doubler le temps de
réponse: il en résultait une diminution de tous les indices qui
restaient toutefois corrélés aux paramètres spirométriques (p
\textless{}0,01 pour tous indices, sauf p \textless{}0,05 pour S1 et p
\textless{}0,02 pour SR). Nous concluons qu'un un temps de réponse plus
long modifie les indices capnographiques, mais n'efface pas leur valeur
diagnostique}. En capnographie en fonction du temps, le débit
expiratoire détermine également la forme générale du capnogramme, car le
PECO2 est davantage lié au volume expiré qu'au temps expiratoire. Une
élévation émotionnelle du débit expiratoire peut être responsable de la
sous-estimation du bronchospasme. En revanche, pendant le sommeil, des
diminutions du débit de ventilation (phases de mouvement oculaire
rapide) liées au stade peuvent induire des déformations
pseudo-obstructives du capnogramme. C'est le prix à payer lorsqu'un
spirogramme n'est pas enregistré simultanément. De la même façon, la
tension artérielle en dioxyde de carbone (PaCO2) détermine l'amplitude
globale du capnogramme et influence légèrement les indices. La durée de
la phase expiratoire influence profondément les derniers indices en
raccourcissant ou en allongeant la partie terminale du capnogramme dont
la pente diminue. Dans certaines conditions (expiration \textless{}1,5 s
ou\textgreater{} 2,5 s), un indice tardif (S3) peut donc
considérablement ou sous-estimer le bronchospasme. Cela peut expliquer
la grande variabilité intellectuelle de ces indices. Enfin, le plateau
alvéolaire peut être altéré par des artefacts dérivés des voies
respiratoires supérieures (obstruction nasale, ondes pulstiles d'origine
carotidienne) qui empêchent l'analyse de 5\% des enregistrements. Ces
différents pièges nécessitent des critères d'utilisation et
d'interprétation adéquates de la capnographie par aspirations; les plus
importants, en utilisation clinique, sont les conditions de repos
strictes (ce qui exclut les épreuves d'effort).

Variabilité et relation aux indices spirométriques L'évaluation des
indices capnographiques doit prendre en compte d'une part leur rapport
bruit / signal (Vi / VI), d'autre part leur corrélation avec les indices
spirométriques. En termes de Vi / VI, les indices initiaux ont le bruit
le plus faible (Vi \textless{}10\%) et montrent des signes de
robustesse, tandis que les derniers indices (S3 et SD3) ont un rapport
bruit / signal élevé (Vi / VI\textgreater{} 50\%) qui interdit leur
utilisation clinique. Inversement, les indices intermédiaires et
terminaux indiquent les meilleures corrélations avec les paramètres
spirométriques. Par conséquent, au cours de l'expérience, le capnogramme
semble être de plus en plus sensible à l'obstruction bronchique mais de
moins en moins reproductible. Les bons résultats de l'étude de
corrélation sont confirmés par des mesures effectuées lors d'un
challenge bronchique. Dans la littérature, deux études conduites chez
l'enfant ont montré des modifications de différents indices
capnographiques lors de défis bronchiques par inhalation {[}23, 24{]}.
Dans l'un de nos travaux précédents {[}16{]}, une corrélation élevée (p
\textless{}0,001) a été constatée entre les variations d'un indice
capnographique (ETS) et d'un des VEMS produits par le salbutamol. Dans
une autre étude (non publiée) menée auprès de six enfants asthmatiques
(âge moyen ± DS ± 5,0 ± 2,2 ans), les valeurs de SD2 ont été comparées à
celles de l'impédance respiratoire obtenue par oscillations forcées. Les
mesures effectuées lors d'un test à l'acétylcholine puis après le
salbutamol ont montré une bonne corrélation (p \textless{}0,01) entre
les variations des deux indices. Ces résultats nous ont incités à
utiliser le suivi en ligne des indices capnographiques, dont les courbes
de tendance donnent une vision dynamique, liée au temps, de la réponse
bronchique à tout médicament {[}26{]}.

La surveillance de l'asthme ne fait pas encore partie de la pratique
clinique; plusieurs méthodes fonctionnelles ont été utilisées
{[}27--35{]} en raison de l'absence d'une méthode universellement
reconnue, sensible, fiable et non invasive. L'absence de contraintes
(non-invasive et coopération-indépendance), associée à une sensibilité
satisfaisante (corrélation élevée avec les paramètres spirométriques),
permet de proposer une capnographie pour cette application. La mesure
informatisée des indices, leur mémorisation et leur visualisation sous
forme de courbes de tendance pourraient constituer un outil utile pour
la surveillance de l'asthme. La capnographie peut être utilisée chez des
sujets éveillés et endormis et pourrait permettre d'envisager de
nouvelles applications: surveillance de l'état asthmatique, détection
des crises nocturnes, évaluation de la durée d'action des
bronchodilatateurs, surveillance intra- et postopératoire des
asthmatiques. patients, et des tests dynamiques de provocation
bronchique, en particulier chez les enfants. ++++++++++++++++++++++++

\hypertarget{capnography-for-monitoring-non-intubated-spontaneously-breathing-patients-in-an-emergency-room-setting-egleston1997capnography}{%
\subsubsection{\texorpdfstring{Capnography for monitoring non-intubated
spontaneously breathing patients in an emergency room setting
{[}\protect\hyperlink{ref-egleston1997capnography}{49}{]}}{Capnography for monitoring non-intubated spontaneously breathing patients in an emergency room setting {[}49{]}}}\label{capnography-for-monitoring-non-intubated-spontaneously-breathing-patients-in-an-emergency-room-setting-egleston1997capnography}}

Egleston C.V. et al.~ont réalisé un travail chez des patients admis en
salle d'urgences vitales en service d'urgence, présetants une détresse
respiratoire aiguë. Deux groupes de patients étaient différentiés le
premier avec sifflement expiratoire faisant évoquer une obstruction des
voies aériennes inférieurs, le deuxième sans sifflement expiratoire.
Lors de la prise en charge habituelle standard des patients, ces
derniers ont bénéficié d'un enregistrement de capnographie à l'aide
d'une canule nasale. Un enregistrement de capnographie de 30 secondes a
été obtenue à l'aide d'un capnographe : Datex Normocap 200,
Instrumentation Corp, Helsinki, Finlande. La morphologie des traces a
été évaluée dans les groupe avec ou sans sibilitant expiratoire pour
déterminer si elle avait un profil normal ou obstructif (c'est-à-dire
Shark'sfin). Les pentes de la phase ascendante (S1) et la phase de
plateau (S2) du capnogramme ont été analysées (fig4). S1 a été mesurée
entre 0 et 0,2 secondes a priori du début du capnogramme, et S2 a été
mesurée entre 0,8 et 1,2 secondes. Le rapport de pente (SR) a été obtenu
par la formule SR = S2 / S1 x 100. Les capnogrammes étaient exclus de
l'analyse du rapport de pente si leurs durées étaient inférieurs à 0,8
secondes ou supérieurs à 3 secondes. Les tracés étaient également
considérés comme non valide en cas d'artéfacts entrainant une
déformation importante de la courbe et si le CO2 final était inférieur à
3,72 kPa. Des travaux antérieurs ont montré que la mesure du rapport de
pente n'est pas fiable dans ces circonstances. Le rapport de pente moyen
a été obtenu pour les deux groupes de patients considérée cliniquement
comme obstructif et non obstructif. Un test t de Student de comparaison
de moyenne indépendante a été utilisé pour comparer les moyennes des
deux groupes. Trente-huit patients ont été inclus dont 26 présentaient
des sibilants expiratoires cliniquement en faveur d'un syndrome
obstructif. L'âge médian du groupe normale était de 32 ans (entre 20 et
62 ans) et de 37 ans (entre 16 et 90 ans) dans le groupe obstructif. Le
rapport homme-femme était de 8/4 dans le groupe normal et de 11/15 dans
le groupe obstructif. Après selection des courbes de capnographie, 11
tracés du groupe de morphologie normale et 14 du groupe de morphologie
obstructif ont été retenus pour la détermination du rapport de pente. Le
rapport moyen de la pente pour les patient sans obstruction était de
7,57 (+/- 0,18) et de 31,9 (+/- 4,46). Les valeurs des pentes S1, S2 et
du rapport de pente pour les deux groupes sont présentées dans les
tableaux 3 et 4. La différence entre les moyennes du rapport de pente
entre les deux groupes était statistiquement significative (P
\textless{}\textless{} 0,001). Cette étude analyse qu'un seul paramètre
calculé, le rapport des pentes S1/S2. Ce dernier a été calculé après
mesure manuel des pentes. Un des arguments avancé pour le choix de ce
paramètre est qu'il est facillement mesurable et ne nécessite pas de
traitement ni de matériel sophistiqué. Que cette mesure peut-être évalué
par mesure manuelle. Les auteurs se basent aussi sur les résultats et la
bonne corrélation de ce paramètre avec le VEMS retrouvé par You B. et
al.~De nombreux tracés n'ont pu être utilisés du fait d'une fréquence
respiratoire importante avec des cycles expiratoires court ou
présentaient des artéfacts importants, excluant ces données de
l'analyse. Ce travaille portant sur un faible effectif avec une
comparaison de moyennes entre deux groupes selectionnés sur des critéres
cliniques qui manquent fréquement de sensibilité et de spécificité.

\hypertarget{capnogram-shape-in-obstructive-lung-disease-krauss2005capnogram}{%
\subsubsection{\texorpdfstring{Capnogram Shape in Obstructive Lung
Disease
{[}\protect\hyperlink{ref-krauss2005capnogram}{50}{]}}{Capnogram Shape in Obstructive Lung Disease {[}50{]}}}\label{capnogram-shape-in-obstructive-lung-disease-krauss2005capnogram}}

Krauss et al., du \emph{service des urgences du ``Children's Hospital''
à Boston}, ont inclus deux cent soixante-un patients sur 15 mois dans
cette étude. Cents dix patients présentaient un syndrome obstructif dont
49 étaient considérés comme sévère, 60 un syndrome restrictif et 91
patients étaient sans pathologie pulmonaire. Ce travail prospectif
incluaient des rectutés via le service d'exploration foncitonnelle, chez
des patients bénéficients d'une exploration par spirométrie avec un état
pulmonaire stable. Les enregistrement de capnographie ont été réalisé à
l'aide d'une canule orale-nasale (prélèvement du CO2 du nez et de la
bouche) à un moniteur multiparamètre portable (Medtronic LIFEPAK®12,
Redmond, WA) avec captage de flux latéral à faible débit (Microstream®).
Cette technique de capnographie recueille un échantillon continu d'air à
un débit de 50 mL / min et enregistre la concentration de CO2
instantanée toutes les 40 ms soit une fréquence de 25 Hertz. Un
échantillon de 90 s, contenant au moins 8 respirations spontanées, a été
prélevé pour chaque patient. Les mesures spirométriques standard, y
compris le VEMS et la capacité vitale forcée, ont été réalisées
immédiatement après les enregistrements de la capnographie. La
catégorisation diagnotique des patients ainsi que le degré de sévérité
de l'atération de la fonction pulmonaire a été confirmé par un
pneumologue indépendant après lectures des données spirométriques.
L'analyse des courbes de capnographie a été réalisé à l'aide d'un
algorithme informatique, capable d'identifier automatiquement et de
mesurer systématiquement les principales caractéristiques du
capnogramme. Les données brutes ont été utilisées et analysées selon les
étapes suivantes. L'analyse des courbes de capnographie a été ralisé
après cyclage de la coubres, respiration par respiration. Le repérage
des cycles respiratoires a été réalisé à l'aide de la pente et de
l'amplitude de la concentration en CO2 permettant de repérer la fin de
l'expiration et le début de l'inspiration. Les différentes phases du
capnogramme ont été identifiés : les phases d'élévation expiratoire
initiale et le plateau alvéolaire. Leur pentes mesurées en mmHg/s ont
été calculées à l'aide d'une regression linéaire et les angles associés
à l'horizontale ont été calculés. La concentration d'ETco2, la fréquence
respiratoire et les temps inspiratoires et expiratoires ont également
été déterminées pour donner un ensemble de six valeurs caractéristiques
pour chaque respiration (valeur ETco2, fréquence respiratoire, angle de
décollage, angle d'élévation, temps inspiratoire et temps expiratoire).
Les valeurs médianes de chaque paramètre ont été utilisées pour
l'analyse réduisant l'influence des valeurs aberrantes. Les valeurs
médianes, à leur tour, ont été moyennées sur tous les sujets appartenant
au même groupe ventilatoire. Les moyennes de chaque groupe ont été
comparées par paire pour en évaluer les différences statistiquement
significatives à l'aide de la méthode de comparaison multiple de
Tukey-Kramer. Les résultats montre une différence entre les moyennes des
paramètres des populations des patients obstructifs vis à vis des
patients normaux et/ou restrictifs. Ces différences sont plus marqués
pour les angles de montée expiratoire initiale et du plateau alvéolaire
(l'angle de décollage et l'angle d'élévation). Ces différences étaient
progressives, augmentant avec la gravité du syndrome obstructif
(\emph{figure n° 3 \& 4}). Pour l'angle de montée expiratoire initiale
une différence statistiquement significative a été retrouvé seulement
entre les patients présentant un syndrome obstructif sévère et les
patients présentant une syndrome obstructif modéré, les patients normaux
et restrictifs. Le groupe des patients présentant un syndrome obstructif
modéré ne présentait pas de différence avec les groupes de patients
normaux ainsi que des groupes des patients restrictifs. Pour l'angle du
plateau alvéolaire, l'ensemble des comparaisons entre groupe, hors mis
celle des normaux versus restrictif, étaient significativement
différent. L'angle moyen de décollage de la phase ascendante du
capnogramme pour le groupe des patients obstructifs sévères était de 7,2
degrés de moins (IC 95\% : 4,0-10,4) par rapport à l'angle de décollage
pour le groupe des patients normaux. L'angle d'élévation moyen du
plateau alvéolaire était de 0,8 degré de plus (IC 95\%: 0,14-1,4) pour
le groupe des patients obstructifs modérés par rapport au sujets avec
spirométrie normale, tandis que l'angle d'élévation moyen était de 3,6
degrés de plus (IC à 95\%: 2,9-4,3) pour le groupe des patients
obstructifs sévère que pour les sujets avec spirométrie normale. L'angle
du plateau alvéolaire semble plus sensible pour distinguer les syndromes
obstructifs. Aucune comparaison de groupe associant les deux angles n'a
été réalisé.

\begin{figure}[h!]

{\centering \includegraphics[width=250px]{figure/KRAUSS_take-off_angle} 

}

\caption{Angle d'élévation expiratoire intiale pour chaque groupe avec intervalles de confiance. A, angle d'élévation par rapport à l'état ventilatoire. B, différences ventilatoires appariées, angle d'élévation. sO = maladie pulmonaire obstructive sévère (DO), mO = OD modérée, n = normale, r = maladie pulmonaire restrictive.}\label{fig:unnamed-chunk-7}
\end{figure}

\begin{figure}[h!]

{\centering \includegraphics[width=250px]{figure/KRAUSS_elevation_angle} 

}

\caption{Angle du plateau alvéolaire pour chaque groupe avec intervalles de confiance. A, angle d'élévation par rapport à l'état ventilatoire. B, différences ventilatoires appariées, angle d'élévation. sO = maladie pulmonaire obstructive sévère (DO), mO = OD modérée, n = normale, r = maladie pulmonaire restrictive.}\label{fig:unnamed-chunk-8}
\end{figure}

\hypertarget{correlations-between-capnographic-waveforms-and-peak-flow-meter-measurement-in-emergency-department-management-of-asthma-hisamuddin2009correlations}{%
\subsubsection{\texorpdfstring{Correlations between capnographic
waveforms and peak flow meter measurement in emergency department
management of asthma
{[}\protect\hyperlink{ref-hisamuddin2009correlations}{45}{]}}{Correlations between capnographic waveforms and peak flow meter measurement in emergency department management of asthma {[}45{]}}}\label{correlations-between-capnographic-waveforms-and-peak-flow-meter-measurement-in-emergency-department-management-of-asthma-hisamuddin2009correlations}}

hisamuddin et al.~ont étudié l'intérêt de la capnographie chez des
patients en crise d'asthme aiguë se présentant aux services des urgences
du centre hospitalier uinversitaire de Sains en Malaysie. Ce travail
portait sur la comparaison et la corrélation entre la mesure
spirométrique de Débit Expiratoire de Pointe (DEP) réalisé à l'arrivée
aux urgences et des paramètres de la courbe de capnographie. Les
critères d'inclusions étaient une dyspnée associée à des antécédents
d'asthme connu chez des patients adultes. Tous les patients présentant
un doute diagnostic ont été exclus. Les patients ont tous bénéficiés
d'un examen clinique et d'une spirométrie par DEP pour évaluer la
sévarité de la crise, ainsi que d'une prise en charge standard de la
pathologie asthmatique. Les patients présentant un état de gravité
instable nécessitant une prise en charge immédiate ainsi que les
patients BPCO ont été exclus également. Pour chaque patient trois
mesures du DEP ont été enregistrées à l'aide du Wright's Mini Peak Flow
Meter®. La mesure la plus élevée des trois a été pris en compte pour
l'analyse. Elle a été exprimer en pourcentage de la théorique attendue
basé sur l'âge, la taille et le sexe. Les courbes de capnographie ont
été enregistré à chaque phase de l'étude avant et après traitement par
un capnographe Novametrix Capnogard®, fabriqué aux États-Unis, avec une
fréquence de mesure de 48 fois par seconde soit 48 Hz, par intervale de
0.02s. Les formes d'onde ont été jugées adéquates/conformes lorsqu'au
moins trois formes d'onde de morphologie régulière, sans artefacts
significatifs, ont été observées. Les enregistrements de DEP et
capnographie ont été ralisés avant et après traitement pour chaque
patient. Les paramètres d'analyse étudiés sur la courbe de capnographie
étaient : la pente de phase 2 (appelé ``pente''), la pente de phase 3
(appeler ``plateau'') et l'angle \(\alpha\) corresondant à l'angle formé
par les pentes de phase 2 et 3. La pente de la phase 2 a été mesurée
pendant 0,25 s à partir du premier point où le CO2 mesuré dépassait 4
mmHg. La pente de la phase 3 a été mesurée entre la 0,75s et la 0,25 s
(soit un temps total 0,5 s ) à partir du point de fin d'expiration. Tous
les capnographes des patients avaient trois lectures permettant de
recueillir trois valeurs pour la pente et le plateau, et trois valeurs
de \(\alpha\) ont été calculées. La moyennes des trois valeurs a été
pris en compte pour l'analyse. Une analyse statistique a été effectué à
l'aide d'un test t appariés pour la comparaison des moyennes, une
corrélations simples et canoniques pour déterminer les corrélations et
un test des rangs signés Wilcoxon pour l'analyse des données
catégoriques. Au total 128 patients ont été inclus dont 8 non pu être
retenu compte tenu d'une mauvais qualité de courbe de capnographie et 16
n'ayant pu ralisé un DEP post traitement. Seulement 100 patients ont été
analysés, agée en moyenne de 35.2 ans (12-71 ans) en majorité des homme
(56\%). Trente pourcent ont bénéficié d'un aérosol de Sablutamole
simple, 53\% ont eu un aérosole d'Ipratropium associé au premier
traitement et 17\% ont eu en plus de ces deux première mesure un bolus
de 200 mg d'Hydrocortisone. Sur les 100 patients inclus dans l'étude,
63\% (n = 63) avaient un DEP \textless{}50\% de la valeur théorique
attendu à l'admission; 26\% (n = 26) un DEP entre 50 et 80\% et
seulement 11\% (n = 11) avait un DEP à plus de 80\% de la théorique.
Après traitement 87 patients ont pu sortir du services des urgences et
13 ont nécessité encore 24h de surveillance. Aucun n'a nécessité de
prise en charge en soins intensifs. Une différence significative entre
les DEP avant et après traitement a été retrouvé (p \textless{}0,001).
Concernant l'EtCO2 aucune différence significative n'a été retrouvé
entre le pré et le post traitement (p=0.871). La pente de phase 3
avaient une moyenne de 0,44 (IC 95\%: 0,33-0,47) avant traitement et une
moyenne de 0,23 (IC 95\%: 0,21-0,28) après le traitement. Cette
réduction était nettement significative (p \textless{} 0,001). L'angle
\(\alpha\) a été calculé avait une moyenne en prétraitement de 134,36
(IC 95\%: 129,53-138,74) était significativement plus élevée que la
moyenne en post-traitement de 123,27 (IC 95\%: 121,25-127,96 ) (p
\textless{} 0,001). Concernant la pente de phase 2 aucune différence
significative n'a été retrouvée entre le pré et le post traitement. De
faible correlation ont été retrouvé entre le DEP et les indices
graphiques des capnogrammes quelques soit la période avant ou après
traitement avec des coefficients de corréltation \textless{} 0.6. Aucune
comparaison des indices graphiques de capnographe, entre les différents
groupes de sévérité d'asthme n'a été ralisé. Ce travail a permis une
évaluation de la capnographie en situation réel aux urgences chez des
patients en crise d'asthme aiguë, ce qui n'est pas le cas de bon nombre
d'étude sur le sujet. Plus souvent des patients stables sont recrutés
pour ces analyses. Une des qualités de cette études, en plus de
travailler sur des patients situation instable, est de réalisé une
études avant après traitement. Les patients sont leurs propres témoins
ce qui démontre une forte preuve de concepte. C'est la première étude de
ce type. L'analyse de forme d'onde capnographique est relativement
nouvelle et n'a pas encore prouvé sa capacité à détecter les changements
dans les voies aériennes du patient asthmatique. Théoriquement, les
débits de pointe sont plus associé au fonction des voies respiratoires
de plus grands diamètres, tandis que le volume expiratoire maximal en 1
s (VEMS) et le VEMS / capacité vitale forcée (CVF) reflètes les voies
respiratoires plus petites. Le FEV1 et le FEV1 / FVC sont techniquement
difficiles à réaliser aux urgences et impliquent des équipements
encombrants contrairement au DEP.

\hypertarget{segmented-wavelet-decomposition-for-capnogram-feature-extraction-in-asthma-classification-betancourt2014segmented}{%
\subsubsection{\texorpdfstring{Segmented Wavelet Decomposition for
Capnogram Feature Extraction in Asthma Classification
{[}\protect\hyperlink{ref-betancourt2014segmented}{55}{]}}{Segmented Wavelet Decomposition for Capnogram Feature Extraction in Asthma Classification {[}55{]}}}\label{segmented-wavelet-decomposition-for-capnogram-feature-extraction-in-asthma-classification-betancourt2014segmented}}

Pomares et al. (40) décrivent une méthode pour évaluer la gravité de
l'asthme en utilisant l'extraction de caractéristiques à partir de la
décomposition des ondelettes du tracé de capnographie (41). Ces
caractéristiques ont été déterminées par la décomposition en segments de
la forme d'onde (le versant ascendant, le plateau et le versant
descendant). Une fenêtre de 3 pixels est utilisée pour approximer la
courbe par une droite par interpolation linéaire. Les pentes des 4
segmenter de la courbe en A-B, C-D, E-F, G-H ont été analysés. Une fois
cette première étape réalisée, 4 signaux ont été extraits permettant une
analyse par transformé en ondelette à l'aide successivement de filtre
passe haut et passe bas associé à une échantillonnage descendant. Dans
le même temps une analyse utilisant une ondelette mère de Haar " de
frome carré " a été réalisé permettant la détermination des coefficients
d'ondelettes. Cette dernière était associé à une analyse
multi-résolution par segments en fonction de la transformation en
ondelettes. En utilisant un classificateur de machine vectorielle de
support (SVM) avec un noyau de fonction Gaussian Radial Basis, les
coefficients d'entrée pour chaque capnogramme ont été classés dans l'une
des six classes de gravité. Le vecteur caractéristique le plus
performant comprenait l'association des segments ascendant et descendant
du capnogramme avec une sensibilité et une spécificité respectivement de
100\% et 91,43\%, en utilisant l'ensemble de données complet pour les
tests, polarisant ainsi les résultats rapportés. Les meilleurs
sensibilité spécificité retrouvé lors de l'analyse d'un seul segment
était de 55,7\% et 99,4\%, et correspondait au segment G-H du plateau de
la courbe. L'analyse, à l'aide de l'ondelette mère de Haar, retrouvait
une sensibilité et spécificité de respectivement 88,6\% et 100\% très
proche de l'analyse associant deux segments de la courbe. L'utilisation
de la transformation en ondelettes permet de sélectionner les
caractéristiques permettant d'optimiser l'analyse de la courbe et la
classification des niveaux de sévérité. Cette étude a été mené à Cuba
sur 23 patients présentant un asthme de modéré à sévère.

Segmentatoin de la courbe : Pour réaliser la segmentation, le signal de
capnogramme est converti en une fonction de pentes. Tout d'abord, une
fenêtre glissante est définie, ayant une largeur égale à 3 échantillons,
ce qui correspond au nombre minimal d'échantillons de segment de signaux
utilisés dans les expériences. Deuxièmement, à l'aide de la fonction
Matlab poly fit, les échantillons de l'intervalle de la fenêtre sont
ajustés sur un segment de ligne et sa pente est calculée. La fenêtre est
décalée d'un échantillon à chaque fois jusqu'à ce que tout le signal ait
été traité. Troisièmement, une courbe de pente en fonction de l'indice
de pente est construite. À partir de cette courbe, les cycles
respiratoires sont d'abord déterminés en identifiant les points de pente
négative minimale correspondant à la fin de l'expiration. Les segments
A-B, E-F et G-H sont ensuite identifiés. Le segment A-B est constitué de
points sur la même pente partant du point de pente minimum. Le segment
E-F correspond à des points de la même pente partant du point de la
pente maximale. Le segment G-H est identifié comme la ligne allant de la
fin de l'expiration au point où la valeur de la pente change au-dessus
de 0,01. Une fois que les segments sont identifiés, les intersegments
sont identifiés le long d'ensembles d'échantillons entre les segments.

Transformée en ondelettes : Ensuite, pour chaque type de segment, le
nombre d'échantillons a été normalisé à une moyenne calculée sans tenir
compte du nombre minimal ou maximal d'échantillons pour ce segment dans
le signal. La transformation en ondelettes a ensuite été appliquée aux
segments. De cette manière, le même nombre de coefficients d'ondelettes
a été obtenu par type de segment. Après la procédure précédente, les
valeurs moyennes des coefficients d'approximation et de détail du
troisième niveau de décomposition - CA3, CD3, CD2 et CD1 - ont été
sélectionnées comme caractéristiques, CA3 étant les coefficients
d'approximation du troisième niveau de décomposition et CD3, CD2 et CD1.
détailler les coefficients de décomposition des niveaux 3, 2 et 1.
L'expérience a également été menée à l'aide de la transformée en
ondelettes de Haar. Le nombre de coefficients d'ondelettes a été
considérablement réduit lorsque la décomposition en ondelettes a été
appliquée (restreinte) aux segments d'intérêt. Pour 6803 échantillons de
capnogramme, par exemple, les coefficients détaillés du premier niveau
de décomposition ont été réduits entre 3402 à 2110, ce qui équivaut à
une réduction de 40\%. Le pourcentage de réduction était similaire pour
les coefficients restants.

\[X_{W}(a,b)=a^{-\frac{1}{2}}\int_{-\infty}^{+\infty} f(t)\varphi(\frac{t-b}{a}) \mathrm{d}t\]

\begin{equation}
    X_{W}(a,b)=a^{-\frac{1}{2}}\int_{-\infty}^{+\infty} f(t)\varphi(\frac{t-b}{a}) \mathrm{d}t
\label{WDT}
\end{equation}

\[X_{W}(a,b)=a_{0}^{-\frac{m}{2}}\int_{-\infty}^{+\infty} f(t)\varphi(a_{0}^{-m}t-nb_{0}) \mathrm{d}t\]

\begin{equation}
    X_{W}(a,b)=a_{0}^{-\frac{m}{2}}\int_{-\infty}^{+\infty} f(t)\varphi(a_{0}^{-m}t-nb_{0}) \mathrm{d}t
\label{WDT_pente}
\end{equation}

Methode de Classification : La méthode utilisée pour sélectionner les
caractéristiques permettant la bonne classification des patients était
un classifieur à vecteurs de support ou ``Support Vector Machines''
(SVMs) en anglais. Seize capnogrammes d'asthme de degré 0 et 7 d'asthme
de degré 1 sont utilisés dans des expériences. Les coefficients
d'ondelettes de Haar, c'est-à-dire CA3, CD3, CD2 et CD1, constituent le
vecteur d'entrée du classificateur. La sortie est le degré de sévérité
de l'asthme, qui a 2 classes, c'est-à-dire 0 correspondant à aucune
crise et 1 correspondant à une oppression thoracique. Le nombre de
coefficients d'ondelettes par segment varie selon les signaux de
l'ensemble de données. Pour construire le vecteur d'entrée, les premiers
coefficients m0 d'ondelettes par segment sont sélectionnés, où m0 est le
nombre minimal de coefficients d'ondelette par segment parmi les signaux
des deux classes. Dans la procédure de vérification, les données
d'apprentissage sont d'abord sélectionnées pour 90\% de l'ensemble de
données. L'entrainement du classifieur SVM avec comme fonction de base
un noyau gaussienne (sigma = 1) est réalisé, puis l'ensemble du jeu de
données est testé. Enfin, les mesures de sensibilité (Se), de
spécificité (Sp), du taux de réponses correctes (CR) et du taux d'erreur
(ER) sont calculées à l'aide des équations. La procédure est répétée 10
fois et la moyenne des mesures d'évaluation des performances est
calculée: où TP - Vrai Positif - correspond au degré 1 des cas d'asthme
correctement classé comme tel, FN - Faux négatif - correspond au degré 1
des cas d'asthme mal classés au degré 0, TN - Vrai négatif - est le
degré 0 des cas d'asthme correctement classé comme tel, un FP - Faux
positif - est le degré 0 des cas d'asthme incorrectement classé comme
degré 1. La classification est appliquée pour différents
vecteurs/groupes/ensembles de caractéristiques - séparément pour les
coefficients d'ondelettes de chaque segment d'intérêt, combinant deux
segments, combinant trois segments d'intérêt et utilisant les
coefficients résultant de l'application de la transformation
d'ondelettes du signal sans segmentation (ondelettte de Haar).

\hypertarget{feature-extraction-of-capnogram-for-asthmatic-patient-kean2010feature}{%
\subsubsection{\texorpdfstring{Feature extraction of capnogram for
asthmatic patient
{[}\protect\hyperlink{ref-kean2010feature}{56}{]}}{Feature extraction of capnogram for asthmatic patient {[}56{]}}}\label{feature-extraction-of-capnogram-for-asthmatic-patient-kean2010feature}}

Kean et al.~ont étudié les coubres de capnographie de patients
asthmatiques et non asthmatique présentant une détresse respiratoire
consultant au service des urgences du Penang Hospital, pour tanter de
les différencier. L'enregistrement des courbes a été réalisé à l'aide
d'une canule naso-buccale par monitorage continu. Le matériel utilisé
pour ces enregistrement étaient un capnographe \emph{Nihon Kohden
Bedside Monitor BSM-2301K} permettant une extraction numérique des
données à partir d'un PC. Au total 34 patientes ont pu être analysés et
enregistrés, dont 18 étaient des patients non-asthmatiques et 16 étaient
asthmatiques. Un prétraitement a été effectué par lissage de la courbe
pour réduir le bruit à l'aide de la fonction \emph{``Curve Fitting
Tool''} du logiciel MATLAB. L'analyse a été réalisé respiration par
respiration ou encore capnogramme par capnogramme. Les paramètres
analysés étaient à la fois des paramétres proposés par des études
précédentes et de nouveaux paramètres proposés. Les paramètres déjà
connus étaient ceux proposés par You B. et al
{[}\protect\hyperlink{ref-you1994expiratory}{51}{]} : S1, S2, S3, SR,
A1, A2 et AR (comme décrits plus haut). Ils ont été recueillis et
calculés exactement de la même manière que décrits dans l'étude de You
B. et al.

Un autre paramètre nouvellement introduit utilisé dans l'étude est les
paramètres de Hjorth
{[}14{]}{[}\protect\hyperlink{ref-hjorth1970eeg}{57}{]}. Dans les
paramètres Hjorth, il y a 3 paramètres impliqués, qui sont l'activité,
la mobilité et la complexité. La figure 5 montre le calcul des
paramètres pour un signal typique.

\begin{verbatim}
  * L'Activité est l'écart type quadratique de l'amplitude de la forme d'onde. 
  * La Mobilité est le calcul de l'écart-type de la pente par rapport à l'écart-type de l'amplitude. 
  * La Complexité, donnant une mesure de détails excessifs en référence à la forme de courbe "la plus douce" possible, l'onde sinusoïdale, correspondant à l'unité.
\end{verbatim}

Les paramètres Hjorth ont été extraits du capnogramme sur deux parties
du capnogramme. Tout d'abord, ils ont été extraits sur tout le cycle du
capnogramme, nommé HP1. Deuxièmement, les paramètres de Hjorth ont été
extraits sur une partie du capnogramme allant du T1 (ou le CO2 est égal
à la moitié de la valeur de l'EtCO2) jusqu'au temps correspondant à
l'EtCO2, nommé HP2.

\hypertarget{investigation-of-capnogram-signal-characteristics-using-statistical-methods-kazemi2012investigation}{%
\subsubsection{\texorpdfstring{Investigation of capnogram signal
characteristics using statistical methods
{[}\protect\hyperlink{ref-kazemi2012investigation}{58}{]}}{Investigation of capnogram signal characteristics using statistical methods {[}58{]}}}\label{investigation-of-capnogram-signal-characteristics-using-statistical-methods-kazemi2012investigation}}

Les modifications du capnogramme de différentes maladies ont amené les
chercheurs à analyser ce signal pour différencier les maladies des voies
aériennes, tout particulièrement pour distinguer les asthmatiques et des
non asthmatiques {[}6-9{]}. Cependant, de nombreuses études antérieures
réalisées manuellement se sont basées sur l'hypothèse que le capnogramme
est un signal stationnaire. Ces analyses basées sur cette hypothèse sont
limitées à l'utilisation de certaines méthodes classiques dans le
domaine temporel, qui supposent que le contenu du signal est constant
dans des intervalles de temps différents. Cependant, la plupart des
signaux physiologiques sont considérés comme étant non stationnaires, ce
qui signifie que la distribution de probabilité conjointe de ces signaux
change lorsqu'ils sont décalés dans le temps. En conséquence, des
paramètres tels que la moyenne et la variance changent avec le temps ou
la position. Ainsi, les informations variables dans le temps sont
utiles. Jusqu'à présent, aucune tentative n'avait été faite pour
analyser les propriétés du capnogramme. Par conséquent, une enquête sur
les caractéristiques du signal semblait nécessaire pour comprendre la
nature des signaux de capnogramme et de les utiliser pour distinguer
avec précision différents types de maladies des voies respiratoires.
Dans cette étude, Kazemi et al.~nous présentent une méthode statistique
pour étudier les propriétés du signal capnogramme. Dans cette méthode,
la distribution de probabilité de premier ordre et toutes les fonctions
de probabilité conjointe sont calculées pour trouver leur dépendance à
l'égard du temps. Les 3 phases de l'étude qui consistent en
l'acquisition des données, le prétraitement et l'analyse des propriétés
des signaux de capnogramme sont présenté dans la partie méthode. Dans
cette étude Kazemi et al.~ont enregistré sur 51 patients consultants
pour dyspnée aux urgences des courbes de capnographies. Trente et un
patients avaient un diagnostic d'asthme et 20 présentaient une
pathologie autres. Ces enregistrement étaient réalisée par voie nazale
ou buccale à l'aide d'un capnographe \emph{CapnostreamTM20 Model
CS08798}. Acquisition des données Pour chaque patients la durée
d'enregistrement étaient d'environ 5 minutes avec une fréquence
d'échantillonnage de 200Hz. L'analyse a été effectuée sur les 5 cycles
respiratoire en continue présentant le moins d'artéfact soit environ une
durée de 20 secondes. Prétraitement Après cette première étape de
sélection des données, un prétraitement des données a été effectué pour
éliminer le bruit inutile. Une méthode de filtrage par moyenne mobile a
été utilisée pour lisser la courbe en raison de sa simplicité et de son
efficacité, en particulier pour éliminer les bruits de haute fréquence
dans les signaux. Cette méthode lisse les données en remplaçant chaque
point de données par la moyenne des points de données voisins définis
dans une plage spécifique. Ce processus équivaut à un filtrage
passe-bas. Une grande plage augmente le lissage mais avec une diminution
de la résolution. Inversement une petite plage diminue l'effet de
lissage mis préserve la résolution. La valeur optimal de la fenetre de
filtrage dépend de l'ensembles des données et nécessité évaluation. Ici
une fenetre de filtrage de 13 a été retenu car elle correspodait au
meilleur compromis entre lissage et résolution. De plus ce choix a été
justifié par le calcul des coefficient de corrélation entre courbe
native et courbe filtré avec un coefficient égal ou supérieur à 0.99.

Analyse de propriétés des signaux de capnographie Un processus aléatoire
stationnaire est caractérisé par une condition d'équilibre dans laquelle
les propriétés statistiques sont invariantes dans le temps {[}15{]}.
Cela signifie que la distribution de probabilité du premier ordre est
indépendante du temps. De même, toutes les fonctions de probabilité
conjointe sont également invariantes par rapport à un décalage d'origine
temporelle; c'est-à-dire que les distributions de probabilité conjointes
du second ordre ne dépendent que de la différence de temps (m-n). Un
processus aléatoire stationnaire est un processus dans lequel les
statistiques dérivées des observations sur les membres de l'ensemble à
deux instants distincts sont les mêmes. Par conséquent, un processus
stationnaire pourrait être dans l'une de ces catégories {[}17{]}: *
processus aléatoire stationnaire de premier ordre; que les statistiques
de premier ordre, moyenne (\(m_{x}\)) et variance (\(\sigma_{x}^{2}\)),
sont invariantes dans le temps. * processus aléatoire stationnaire de
second ordre; que la statistique du second ordre, l'autocorrélation
(\(\varphi_{xx}\)), dépend tout au plus des différences de temps (m-n)
et non des deux valeurs m et n. * processus aléatoire stationnaire au
sens large; que la moyenne des statistiques du premier ordre et
l'autocorrélation des statistiques du second ordre sont invariantes dans
le temps. Par conséquent, un processus considéré comme étant
stationnaire de second ordre est également stationnaire au sens large. *
processus aléatoire strictement stationnaire; que tous les moments et
moments articulaires possibles sont invariants dans le temps. Donc, si
un processus aléatoire est strictement stationnaire, il doit être
stationnaire pour toutes les minuscules.

Pour l'application de cette méthode, les signaux de capnogramme ont été
séparés en segments égaux pour calculer leur moyenne et leur variance.
C'est la première étape pour analyser les propriétés des signaux de
capnogramme. Selon la fréquence respiratoire de chaque patient, chaque
échantillon a été divisé en 5 segments égaux, en s'efforçant que chaque
segment contienne un cycle respiratoire complet. Ensuite, la moyenne et
la variance de chaque cycle ont été calculées. Les Fig.9 et Fig.10
montrent respectivement les CNP2 et CAP7 avec leurs segments séparés.
Comme le montrent les Fig.9 et Fig.10, chaque échantillon est séparé en
5 cycles complets de respiration qui constituent la première étape pour
étudier les caractéristiques du signal de capnogramme. La fonction
d'autocorrélation normalisée de chaque échantillon a également été
calculée.

Résultats

\hypertarget{frequency-analysis-of-capnogram-signals-to-differentiate-asthmatic-and-non-asthmatic-conditions-using-radial-basis-function-neural-networks-kazemi2013frequency}{%
\subsubsection{\texorpdfstring{Frequency analysis of capnogram signals
to differentiate asthmatic and non-asthmatic conditions using radial
basis function neural networks
{[}\protect\hyperlink{ref-kazemi2013frequency}{59}{]}}{Frequency analysis of capnogram signals to differentiate asthmatic and non-asthmatic conditions using radial basis function neural networks {[}59{]}}}\label{frequency-analysis-of-capnogram-signals-to-differentiate-asthmatic-and-non-asthmatic-conditions-using-radial-basis-function-neural-networks-kazemi2013frequency}}

Analyse Fréquentielle par Transformée de Fourier Discrète ou
\emph{Discrete Fourier Transform (DFT)} en anglais. La transformée de
Fourier discrète (DFT) d'une séquence de \(N\) points \(x(n)\) est
définie sous la forme suivante :

\[X(k)=-\sum_{n=0}^{N-1} x(n)\mathrm{e}^{-j2\pi nk/N}\]

\begin{equation}
    X(k)=-\sum_{n=0}^{N-1} x(n)\mathrm{e}^{-j2\pi nk/N}
\label{DFT}
\end{equation}

Modéle Autogressif ou \emph{Autoregressive models (AR)} en anglais. Le
modèle AR d'une série chronologique ou temporelle est représenté sous la
forme suivante :

\[x(n)=-\sum_{m=1}^{P} a(m)x(n-m)+e(n)\]

\begin{equation}
    x(n)=-\sum_{m=1}^{P} a(m)x(n-m)+e(n)
\label{autoregresive model}
\end{equation}

où \(x(n)\) est la série temporelle, \(a(m)\) sont des paramètres de
l'AR, \(P\) est l'ordre du modèle et \(e(n)\) est l'erreur de
prédiction.

\hypertarget{new-prognostic-index-to-detect-the-severity-of-asthma-automatically-using-signal-processing-techniques-of-capnogram-kazemi2016new}{%
\subsubsection{\texorpdfstring{New Prognostic Index to Detect the
Severity of Asthma Automatically Using Signal Processing Techniques of
Capnogram
{[}\protect\hyperlink{ref-kazemi2016new}{60}{]}}{New Prognostic Index to Detect the Severity of Asthma Automatically Using Signal Processing Techniques of Capnogram {[}60{]}}}\label{new-prognostic-index-to-detect-the-severity-of-asthma-automatically-using-signal-processing-techniques-of-capnogram-kazemi2016new}}

Kazemi et al. (38) (39) ont recueilli des enregistrements de
capnographie sur canule nasale pour 73 patients asthmatiques et 23 non
asthmatiques provenant du service des urgences. Les enregistrements
comprenaient une durée de 5 minutes chacune avec une fréquence
d'échantillonnage de 200Hz. Dans un second temps une sélection de 5
cycles respiratoires consécutifs sans artéfact correspondant à environ
20 secondes a été réalisé pour chaque patient. Une variabilité
inter-respiration significative était statistiquement retrouvé sur la
moyenne et la variance conduisant l'auteur à caractériser le signal de
capnographie comme non stationnaire. À partir du capnogramme, les
caractéristiques ont été extraites en utilisant différentes façons de
représenter des signaux non-stationnaire : dans le domaine temporel,
avec un codage prévisionnel linéaire (LPC) et dans le domaine
fréquentiel, avec un modèle autorégressif associé à une transformé
rapide de Fourier pour représenter la densité spectrale de puissance
(PSD). La capnographie étant un signal biomédical lié à la respiration
et donc de la catégorie des signaux basse fréquence, les auteurs ont
choisis une fenêtre de Blackman de longueur M = 256 associé à la
transformé rapide de Fourier (FFT). Ce choix n'a pas affecté la
résolution temporelle du signal. À partir des caractéristiques
extraites, cinq caractéristiques (toutes avec des AUC individuels
\textgreater{} 0,7) ont été sélectionnées pour la classification, y
compris 2 coefficients LPC, et, à partir du PSD, la première composante
de fréquence ainsi que l'amplitude et nombre total de composants de la
bande de fréquence ont été sélectionnés. En utilisant les cinq
caractéristiques des capnogrammes de patient asthmatiques et non
asthmatiques, la classification par une ANN avec comme sortie unique la
gravité de l'asthme, a montré que la détection rapportée et le taux
d'erreur étaient respectivement de 90,15 et 9,85\%. Dans cette études
les auteurs ont rapporté une différence significative du nombre de
composante spectrale et de leur amplitude de la FFT entre les patients
asthmatiques et non asthmatiques. Le profil spectral des patients
asthmatiques présentait deux pics principaux contre un seul pour les
patients non asthmatiques. La fréquence de ces pics principaux ainsi que
leur bande passante étaient différente entre ces deux populations. Le
premier des pics principaux avait une fréquence associée à une bande
passante pour les patients asthmatiques resperctivement de 0.078 Hz et
0.23 Hz contre 0.04 Hz et 0.08 Hz respectivement pour les non
asthmatiques. L'ensemble des paramètres (magnitude, fréquence et bande
passante) du premier des pics principaux présentaient des performances
diagnostic importante avec une Aire sous la courbe de la courbe ROC
(AUC) supérieur à 0.8. De la même façon l'analyse en PSD met en évidence
des différences similaires avec deux pics de composante spectrale pour
les patients asthmatique contre un pic pour les patients non
asthmatique. La performance diagnostic la plus intéressante était
représenté par la fréquence du premier pic en PSD avec respectivement
une sensibilité, spécificité et un AUC de respectivement 98\%, 95\% et
0.996 (p \textless{} 0.0001). L'étude des composantes spectrales par
transformé de Fourier du signal de capnographie n'a quasiment pas été
étudié. Cette étude de Kazemi et al.~montre des résultats prometteurs
avec des performances diagnostics intéressantes pour dépister les
patients asthmatiques et fait référence en la matière. Le DEP nécessite
une expiration forcée avec participation du patient dans un contexte
d'anxiété et de stress. Cette manoeuvre entraine parfois un spasme
sévère ou une toux importante entrainant une apréhension à la
réalisation de ce test expliquant un biais dans le receuils des cette
données aux urgences. En l'absence d'autre technique fiable et bien
qu'imparfraite car sujet à de nombreux biais, le DEP reste la référence
et la plus largement utilisée pour l'évaluation des patients
asthmatiques en crises aiguës aux urgences.

Codate Prédictif Linéaire ou \emph{Linear Predictive Coding (LPC)} en
anglais. Pour une échantillon de signal \(x(n)\) :
\[x(n)=\sum_{k=1}^{p} \alpha_{k}x(n-k)\]

\begin{equation}
    x(n)=\sum_{k=1}^{p} \alpha_{k}x(n-k)
\label{Linear Predictive Coding}
\end{equation}

où \(\alpha_{k}\) et \(p\) sont respectivement les coefficients et
l'ordre du LPC, et \(e(n)\) est l'erreur de prédiction. Dans cette étude
une analyse de la meilleur valeur de \(p\) (d'odre) a été réalisé
montrant que l'ordre 8 montre la meilleur corrélation entre signal
original et signal estimé. Cela s'explique par le fait que la fréquence
respiratoire est comprise entre 12 et 50 cycles par minutes
correspondant à une fréquence basse. Les coefficients \(\alpha_{k}\)
pour \(k=1\) et \(k=4\) montre les meilleurs sensibilité/spéficificité
ainsi qu'AUC avec respectivement 93.5/94.7 (0.70) et 96.8/95.6 (0.72).
Ces deux coefficients montrent la meilleur distinction entre les deux
groupes.

Modéle Autogressif ou \emph{Autoregressive models (AR)} en anglais. Le
modèle AR d'une série chronologique ou temporelle est représenté sous la
forme suivante :

\[x(n)=-\sum_{m=1}^{P} a(m)x(n-m)+e(n)\]

\begin{equation}
    x(n)=-\sum_{m=1}^{P} a(m)x(n-m)+e(n)
\label{autoregresive model}
\end{equation}

où \(x(n)\) est la série temporelle, \(a(m)\) sont des paramètres de
l'AR, \(P\) est l'ordre du modèle et \(e(n)\) est l'erreur de
prédiction.

Reseau de Neurone Artificiel Sur la base des résultats obtenus, le
vecteur de caractéristiques d'entrée pour le RNA est constitué de six
éléments. Il s'agit des valeurs \(\alpha_{1}\) et \(\alpha_{4}\) de
l'analyse LPC, du nombre de composantes fréquentielles, de l'amplitude
de la première composante et de la fréquence des première et deuxième
composantes de l'estimation de la densité PSD par la méthode de
modélisation AR de Burg. L'objectif du RNA était de classifier les
patients en fonction du degré de sévérité de leur asthme de 0 à 10. Zéro
pour les patients ne présentant pas d'asthme jusqu'à 10 représentant les
crises d'asthme sévère.

\hypertarget{automated-quantitative-analysis-of-capnogram-shape-for-copdnormal-and-copdchf-classification-mieloszyk2014automated}{%
\subsubsection{\texorpdfstring{Automated Quantitative Analysis of
Capnogram Shape for COPD--Normal and COPD--CHF Classification
{[}\protect\hyperlink{ref-mieloszyk2014automated}{61}{]}}{Automated Quantitative Analysis of Capnogram Shape for COPD--Normal and COPD--CHF Classification ,{[}61{]}}}\label{automated-quantitative-analysis-of-capnogram-shape-for-copdnormal-and-copdchf-classification-mieloszyk2014automated}}

Cette article a été écrit par le \emph{Mieloszyk R. et al.} du
\emph{Massachusetts Institute of Technology Cambridge USA} et pupblié en
2014. L'objectif principal de ce travail était de distinguer en les
classifiant de façon automatique à l'aide de la courbe de capnographie,
les patients indemnes de pathologies pulmonaires, les patients
présentant une exacerbation de BPCO et les patients en insuffisance
cardiaque aiguë. L'objectif principal était de distinguer les patients
en exacerbation de BPCO par rapport au patient en insuffisance cardiaque
aiguë.

La section III traite du prétraitement du capnogramme et de la
conception du classificateur (algorithme d'apprentissage) à l'aide de
l'ensemble d'apprentissage.

Les données ont été recueillies prospectivement sur des échantillon de
patients provenant de trois centres d'inclusion et sur une période de
sept ans. Les patients sains indemne de toutes pathologie pulmonaire ont
été inclus sur deux sites. Le recrutement des patients présentant une
insuffisance cardiaque aiguë ou une décompensation de BPCO ont été
réalisées sur le troisièmes site dans le service des urgences chez des
patients présentant une détresse respiratoire aiguë.

Les enregistrements des patients des urgences étaient réalisés en
position assise à l'aide d'une canule nasale à un capnographe portable
\emph{(Capnostream 20, Oridion Medical, Needham, MA, USA)}. Le
capnographe recueille un échantillon continu à un débit de 50 mL / min
et enregistre le PeCO2 instantané toutes les 50 ms. Pour ces patients
les enregistrements étaient effectués sur une durée 10 à 30 min.
Concernant les sujets normaux, après avoir obtenu leur consentement
éclairé , l'enregistrement s'est fait dans les même conditions en
position assis avec une canule nasale avec le même type de matériel
(Capnostream 20). Les durées d'enregistrement étaient de 3 min ou 15 min
en fonction des sites pour les sujets normaux. L'analyse a été réalisé
après partitionnement des données en deux groupes. Un premier groupe de
données permettant de développer et d'entrainer l'algorithme de
classification. Le second groupe correspond à groupe test permettant
d'évaluer et de valider le modèle à l'aide de test (d'analyse) de
perfomance en comparant les résultats de classification à l'aide de
l'algorithme de classification avec les résultats établis par les
cliniciens. Au total 143 patients ont été inclus. Quatre patients ont
été exclus car ils présentaient un tableau mixte de décompensation
cardiaque et d'exacerbation de BPCO. Le groupe d'entrainement presentait
84 patients distribué en : 20 patients normaux, 31 patients en ICA et 33
patients en exacerbation de BPCO. Le groupe test quand à lui comprenait
55 patients : 10 patients normaux, 22 patients en ICA et 23 patients en
exacerbation de BPCO. Le prétraitement des capnogrammes comprennait :
l'isolement des capnogrammes un par un en identifiant et marquant le
début et la fin de chaque expiration; un capnogramme type était produit
en moyennant les profils expiratoire pour chaque patient et en éliminant
les profils abérants correspondant une déviation exagérer par rapport au
profil moyen; et pour finir l'extraction des caractéristiques
physiologiques sélectionnées de chaque expiration. Le début et la fin de
chaque segment d'expiration pour chaque enregistrement ont été
identifiés à l'aide d'un algorithme reconnaissant le début des pentes
positive et négative. Un profil de capnogramme moyen a ensuite été
construit en superposant les expirations en les centrants sur une valeur
prédéterminée de 15 mmHg de PeCO2 (condidérée comme une valeur précoce
et reproductible pour chaque expiration), et en moyennant toutes les
valeurs à chaque pas de temps (ou chaque instant). L'amplitude de
variation par rapport au modèle a été déterminée en calculant, à chaque
instant, l'écart-type des expirations superposées. Un exemple d'analyse
et montré dans la \emph{figure 3}.

\begin{figure}[h!]

{\centering \includegraphics[width=250px]{figure/mieloszyk2014_fig1} 

}

\caption{Profil expiratoire d'un capnogramme de patient BPCO. Les exhalaisons de capnogrammes (en bleu) sont alignées à 15 mmHg de PeCO2 puis moyennées verticalement pour construire le profil moyen du capnogramme (en rouge).}\label{fig:unnamed-chunk-9}
\end{figure}

Les expirations atypiques ont été reconnues et exclues en utilisant le
profil moyen comme référence de capnogrammele modèle de capnogramme
comme exemple de respiration. Chaque expiration a été notée sur la base
d'une mesure globale des déviations par rapport au profil moyen
standard, en se référant à l'écart-type approprié (associé). Les
expirations, dont l'écart global dépassaient un certain seuil, étaient
exclues de l'analyse et de la classiffication. Après le prétraitement
automatisé de chaque enregistrement pour éliminer les exhalaisons des
valeurs aberrantes, les 80 premières expirations valides de chaque
enregistrement ont été utilisées pour l'analyse.

Les quatre caractéristiques physiologiques utilisées pour la conception
de l'électeur ont été sélectionnées à l'aide de l'analyse de validation
croisée décrite plus loin et comprenaient les éléments suivants
\emph{figure 4}:

\begin{itemize}
\tightlist
\item
  durée de l'expiration;
\item
  PeCO2 maximum ou PeCO2 de fin d'expiration (ETCO2);
\item
  temps passé à ETCO2;
\item
  pente de fin d'expiration.
\end{itemize}

La durée d'expiration est la durée entre le début de l'expiration et la
fin de celle-ci. PeCO2 maximum est la valeur PeCO2 à la fin de
l'expiration. Le temps passé à ETCO2 est la durée pendant laquelle PeCO2
reste à sa valeur maximale. La pente d'expiration finale a été calculée
comme la pente d'une droite correspondant aux cinq dernières valeurs de
PeCO2 de l'exhalation.

\begin{figure}[h!]

{\centering \includegraphics[width=450px]{figure/mieloszyk2014_fig2} 

}

\caption{Quatre caractéristiques extraites du capnogramme et utilisées pour la classification. Celles-ci comprennent : la durée de l'expiration, le CO2 en fin d'expiration (ETCO2), la pente de l'expiration finale et le temps passé à l'ETCO2.}\label{fig:unnamed-chunk-10}
\end{figure}

L'entraînement et le vote de chaque électeur se déroulent dans l'espace
à quatre dimensions défini par les caractéristiques. Deux approches pour
la constitution et le vote de chaque électeur ont été suivi. Dans une
première approche, respiration par respiration, les caractéristiques de
chaque expiration dans l'ensemble des respirations constituent un point
de l'espace \(\mathbb R^{4}\) des caractéristiques, de sorte qu'il y a
autant de points que d'expiration. Dans la seconde approche par moyenne
des caractéristiques, les valeurs de ces caractéristiques respectives
sont moyennées sur l'ensemble des expirations pour un enregistrement
donné permet de définir un point unique dans l'espace \(\mathbb R^{4}\)
des caractéristiques.

Dans cette article les auteurs fond par des besoins informatiques, du
coût qu'on nécessité les calculs de ces analyses. Les calculs pour le
prétraitement, l'extraction de caractéristiques, la validation croisée
et la construction ROC sur l'ensemble de l'ensemble ont duré moins de
2,7 minutes sur un ordinateur portable MacBook Pro 2012 (Apple,
Cupertino, Californie) avec 4 Go de RAM et un processeur Intel Core i7
de 2,2 GHz. processeur exécutant MATLAB 2013a (MathWorks, Natick,
Massachusetts).

Un seuil de 60\%, correspondant à une sensibilité et à une spécificité
d'environ 0,78, a été choisi pour la classification entre ICA et BPCO.
Ce seuil correspond au seuil minimal de votant en faveur d'une ICA pour
le calsser comme tel par rapport à une exacerbationd de BPCO. De la même
manière un seuil de 50\% de votant en faveur d'une exacerbation de BPCO
pour le classifier comme tel par rapport à des patients normaux indemnes
de pathologie pulmonaire. Ce seuil correspond à une sensibilité et une
spécificité d'environ 0.88.

L'exécution des classificateurs en utilisant l'approche respiration par
respiration avec 80 expirations retrouve des performance diagnostique
pour la classificaiton des patients ICA et BPCO avec une AUC de 0,89 (IC
à 95\%: 0,72-0,96). La courbe ROC pour la classification entre patient
BPCO et normaux retrouvait une AUC de 0,98 (IC à 95\%: 0,82-1,0).
L'approche respiration par respiration présentait de meilleurs
performance diagnostique avec une AUC supérieure sur l'ensembles de test
diagnostique aussi bien pour les ICA/BPCO que pour les BPCO/Normaux, par
rapport à l'approche moyennant les caractéristiques.

\hypertarget{model-based-estimation-of-pulmonary-compliance-and-resistance-parameters-from-time-based-capnography-abid2015model}{%
\subsubsection{\texorpdfstring{Model-Based Estimation of Pulmonary
Compliance and Resistance Parameters from Time-Based Capnography
{[}\protect\hyperlink{ref-abid2015model}{11}{]}}{Model-Based Estimation of Pulmonary Compliance and Resistance Parameters from Time-Based Capnography {[}11{]}}}\label{model-based-estimation-of-pulmonary-compliance-and-resistance-parameters-from-time-based-capnography-abid2015model}}

Cette article a été écrit par le \emph{Abid A. et al.} du
\emph{Massachusetts Institute of Technology Cambridge USA} et pupblié en
2015. Elle porte sur l'analyse de la courbe de capnographie par un
modèle mathématique de type mécanistique. Pour réaliser cette analyse 15
patients asthmatiques ont été inclus avec un âge médian de 48 ans en
majorité des femmes (73\%). L'enregistrement des courbes de capnographie
associé à la mesure du VEMS (Volume Expiratoire Maximal Seconde) a été
fait avant, pendant et après un test à la Metacholine. La réponse à la
Métacholine était considérée comme positive pour une diminution de plus
de 20\% du VEMS. Des Bêta2-mimétiques étaient administrées comme
bronchodilateur après le test à la Metacholine pour permettre un retrour
à la normale. Les mesures de capnographie de longueur variable (entre 2
et 5 minutes) ont été enregistrés en utilisant un capnographe portable
\emph{(Capnostream 20, Covidien, Mansfield, Massachusetts)} à l'aide
d'une canule nasale avec une fréquence de mesure à 20 Hz. L'analyse de
capnographie a été réalisée après un post-traitement, capnogramme par
capnogramme, en éliminant les segments de la phase inspiratoire et en
gardant uniquement les segments de la phase d'expiration. L'analyse a
consisté en un ajustement de la courbe, des segments expiratoires
mesurés, à un modèle mathématique de type mécanistique \emph{(fitting)}.
Ce modèle est basé sur l'hypothèse de deux sous-systèmes interconnectés.
Le premier est le sous-système de flux d'air, qui régit le débit d'air
total, mobilisé par les différences de pression dans différents
compartiments pulmonaires. Le second est le sous-système de mélange des
gaz, qui s'appuie sur le sous-système de flux d'air, et régit
l'évolution du taux de CO2 lors du mélange des zones riches en CO2 des
poumons avec l'air pauvre en CO2 correspondant à l'espace mort.
Concernant le sous-système de flux d'air : Un modèle simple est utilisé,
celui d'un compartiment alvéolaire unique, qui modélise la région des
poumons qui s'étend (en grande partie l'espace alvéolaire) comme une
paire de compartiments coulissants, comme le montre la figure 1. Ces
compartiments sont reliés les uns aux autres avec un ressort de
constante de \(C_{l}\), représentant la compliance du tissu alvéolaire,
et reliés à l'atmosphère par un tuyau de résistance \(R_{l}\),
représentant les résistances des voies aériennes supérieures. Concernant
le sous système de mélange : Lors de l'inspiration, l'espace mort, qui
est la région du poumons ne participant pas aux échanges gazeux, est
rincée avec de l'air atmosphérique contenant une fraction de CO2
négligeable. L'air dans la région alvéolaire, cependant, a des quantités
substantielles de CO2 (à pression partielle \(p_{A}\)) en raison des
échanges gazeux avec les capillaires pulmonaires. Lors de l'expiration,
l'air riche en CO2 des alvéoles se mélange à l'air pauvre en CO2 de
l'espace mort. Nous modélisons ce processus comme un mélange instantané
d'air: une quantité infinitésimale d'air pénètre dans l'espace mort et
se mélange instantanément pour créer un mélange homogène d'air dans
l'espace mort, de concentration \(p(t)\) au temps \(t\). Une quantité
infinitésimale de ceci est expulsée par la bouche et le nez pour
maintenir le volume de l'espace mort, \(V_{D}\), fixé. Ce modèle est
basé sur le principe qu'un mélange important se produit dans l'espace
mort par la turbulence et la ramification des voies aériennes dans les
voies respiratoires supérieures. En se basant sur ces hypothèses et en
considérant \(p_{D}(t)\) au temps zéro, égale à zéro soit
\(p_{D}(0)=0\), les auteurs retouve une équation du modèle mécanistique
du type : \[
p_{D}(t)=p_{A} - p_{A} \mathrm{e}^{\alpha} \mathrm{e}^{\alpha\mathrm{e}^{-\frac{t}{\tau}}} 
\tag{**}
\]

\begin{equation}
  p_{D}(t)=p_{A} - p_{A} \mathrm{e}^{\alpha} \mathrm{e}^{\alpha\mathrm{e}^{-\frac{t}{\tau}}}
\label{eq:modèle mécanique 1}
\end{equation}

avec \(\alpha= \frac{-C_{l}\Delta P}{V_{D}}\) et \(\tau=R_{l}C_{l}\).

\begin{itemize}
\tightlist
\item
  \(C_{l}\) : la compliance pulmonaire
\item
  \(R_{l}\) : la résistance des voies aériennes
\item
  \(\Delta P\) : gradient de pression moteur en expiration
\item
  \(p_{A}\) : pression partielle de CO2 alvéolaire
\item
  \(V_{D}\) : volume de l'espace mort anatomique
\end{itemize}

Tableau 1: Paramètres

\begin{longtable}[]{@{}cl@{}}
\toprule
paramétres & description\tabularnewline
\midrule
\endhead
\(C_{l}\) & la compliance pulmonaire\tabularnewline
\(R_{l}\) & la résistance des voies aériennes\tabularnewline
\(\Delta P\) & gradient de pression moteur en expiration\tabularnewline
\(p_{A}\) & pression partielle de CO2 alvéolaire\tabularnewline
\(V_{D}\) & volume de l'espace mort anatomique\tabularnewline
\bottomrule
\end{longtable}

Les expirations inférieures à 1,25 seconde ont été exclus. Les 2 200
expirations résultantes étaient chacune ajustées à la forme de
l'équation en choisissant des valeurs pour \(\alpha\) et \(\tau\)
(alternativement, \(R_{l}\) et \(C_{l}\) pour une fraction fixe
\(\frac{\Delta P}{V_{D}}\)) qui minimisant l'erreur quadratique moyenne
entre le CO2 calculé et le profil du CO2 mesuré. La valeur de \(p_{A}\)
a été fixée par les auteurs à 40 mmHg en considérant l'hypothèse que
cette valeur ne variait pas pendant la provocation à la méthacholine. Un
premier résultat globale non apparié montre des valeurs Pour une
majorités de patient, après le test à la méthacholine, une tendance
semble se dégager avec une augmentation du paramètre \(\tau\) ou une
diminution du pramètre \(\alpha\), voir les deux. Ces résultats
s'expliquent par le fait que la métacholine agissant comme un
bronchoconstricteur augmente les résistances pulmonaires \(R_{l}\) et
ainsi la valeur \(\tau\). Des études suggèrent également que
l'obstruction des voies aériennes augmente l'espace mort physiologique
et ainsi une diminution du paramètre \(\alpha\). Bien qu'une
correspondance entre les paramètres estimés par capnographie et le VEMS
ne soit pas retrouvé chez tous les sujets, la plupart des sujets
affichent une tendance avec une augmentation de \(\tau\) et une
diminution de \(\alpha\) lors de l'administration de la méthacholine. Ce
modèle mécanique simple mono-alvéolaire de poumon semble suffisament
complexe pour extraire des paramètres permettant de décrir et de
discriminer différents état du poumons. Ce modèle reste assez simple
pour estimer les paramètres de résistance et de compliance pulmonaire,
ainsi que les relations existant entre ces deux propriétés
physiologiques. Dans ce travail aucune données n'est communiqués sur les
résidus et l'erreur standard entre les données mesurées et les
estmiations du modèle. Les auteurs nottent également le fait de ne pas
avoir pris en compte les variablités de concentration du CO2 alvéolaire,
l'hétérogénéité de concentration entre les alvéoles et le bruit dû au
capteur.

\hypertarget{model-based-estimation-of-respiratory-parameters-from-capnography-with-application-to-diagnosing-obstructive-lung-disease-abid2017model}{%
\subsubsection{\texorpdfstring{Model-Based Estimation of Respiratory
Parameters from Capnography, with Application to Diagnosing Obstructive
Lung Disease
{[}\protect\hyperlink{ref-abid2017model}{12}{]}}{Model-Based Estimation of Respiratory Parameters from Capnography, with Application to Diagnosing Obstructive Lung Disease {[}12{]}}}\label{model-based-estimation-of-respiratory-parameters-from-capnography-with-application-to-diagnosing-obstructive-lung-disease-abid2017model}}

Ce nouvelle article de \emph{Abid A. et al.} s'inscrit dans les suites
du précédent publié en 2015
{[}\protect\hyperlink{ref-abid2015model}{11}{]} se basant sur un modèle
mathématique de type mécanistique pour analyser la courbe. Ce travail
comporte deux phases d'analyse. La première phase a pour objectif
d'estimer les paramètres respiratoires \(\tau\) et \(\alpha\), décrivant
la courbe de capnographie résultant de l'équation se basant sur le
modèle mécanistique. La seconde phase évalue le modèle sur les
différentes phases d'un test à la métacholine chez des patients
asthmatiques.

Les enregsitrements de capnographie ont été réalisés en service d'EFR en
position assise portant à l'aide d'une canule nasale reliée à un
capnographe portable \emph{(Capnostream 20, Covidien, Mansfield,
Massachusetts)}. La durée des enregistrements étaient de longueur
variable entre 2 et 20 minutes avec un fréquence d'échanillonnage de
20Hz avec une résolution d'amplitude de 1mmHg. Ce protocol a été suivi
pour l'ensemble des enregistrements concernant les deux phases de
l'étude.

Les données des enregistrements ont été importées et analysées à l'aide
du logiciel \emph{MATLAB} permettant l'extraction de chaque expiration
de façon individuelle après prétraitement.Les étapes de prétraitement
ont été effectuées pour atténuer les effets du bruit et des artefacts,
tels que les respirations partielles et la toux. Ce prétraitement
comportait une phase de reconnaissance du début et de la fin de
l'expiration, associé à un processus de selection permettant d'exclusion
les expirations atypiques. Le début de l'expiration était
considérée(reconnue/détectée) lorsque l'association d'un taux de CO2
égale ou inférieur à 1 mmHg immédiatement suivi d'un taux de 2 mmHg ou
plus pour la seconde mesure. La fin d'une expiration était indiquée par
une chute de 1 mmHg ou plus de pression partielle de CO2 mesurée entre
deux mesures consécutives. Les expirations de moins de 1,25 seconde ou
de plus de 4 secondes ont été exclues pour minimiser les erreurs
résultant des formes atypiques. Les expirations ayant donné de mauvais
ajustements, définis comme des cas où la somme des carrés des résidus
dépassait 4,8 mmHg2 (correspondant au 75e percentile des exhalations
normales), ont été également rejetées. Pour finir ce processus de
selection, un petit nombre d'expiration ayant donnée des valeurs
atypiques particulièrement élevées (supérieures à 9) après un bonne
ajustement ont-elles aussi été exclues. Cette processus de selection
correspond à un processus de validation des capnogrammes permettant
secondairement de leur inclusion pour l'analyse. La valeur de \(p_{A}\)
de l'équation (x) a été défini pour chaque patient comme la valeur la
plus élévée des peCO2 mesurée sur l'ensemble de l'enregistrement pour
chaque individu. Ce paramètre a été fixé par les auteurs de cette
manière car les données expérimentales n'ont pas permis d'estimer
simultanément les trois paramètres de l'équation (x). La \(p_{A}\) est
la pression partielle en CO2 au niveau alvéolaire. La PeCO2
correspondant à la pression partielle en CO2 expiré n'est pas stricement
égale à la \(p_{A}\). Elle est fréquement inférieure et ne s'en
rapproche que lorsque le volume courant est plus important que le volume
de l'espace mort {[}ref. 19{]}. Ces éléments expliquent le choix des
auteurs étayé par des études antérieurs suggérant que l'EtCO2 peut être
utilisé comme une estimation acceptable de la \(p_{A}\) pour le calcul
de certains paramètres respiratoires {[}ref. 19{]}. Après avoir fixé la
valeur de \(p_{A}\) pour chaque patient, un ajustement entre les valeurs
expérimentales enregistrées et l'équation (l'expression analytique) (x)
de la courbe individuellement pour chaque segment d'expiration. Cette
ajustement a été réalisée en utilisant un ajustement non linéaire basée
sur la méthode des moindres carrés, lsqcurvefit de MATLAB. En
recherchant les valeur de \(\tau\) et de \(\alpha\) permettant de
minimisé l'erreur quadratique entre l'expression analytique et les
valeurs mesurées du capnogramme. Les valeurs des paramètres \(\tau\) et
de \(\alpha\) ont été limité entre 0.01 et 50 (\(\tau\) est en secondes
et \(\alpha\) sans unité) pour ne pas dépasser des seuils physiologiques
acceptables.

Première partie : L'objectif de cette première partie de l'étude était
de détreminer (d'évaluer) la possibilité de distinguer de manière
fiable, à l'aide des paramètres \(\tau\) et de \(\alpha\), les sujets
normaux des patients atteints d'une BPCO. \emph{ou} L'objectif de cette
première partie de l'étude était de détreminer si les différences des
paramètres \(\tau\) et de \(\alpha\) entre les deux populations
permettaient de distinguer de manière fiable les sujets normaux des
patient atteints d'une BPCO. La population étudié était composée de 46
sujets dont 22 présentaient une BPCO et 24 considéré comme normaux.
Après le prétraitement des données, 71\% des expirations normales et
51\% des expirations de patients BPCO ont été retenues, permettant
l'extraction(l'estimation) des paramètres \(\tau\) et de \(\alpha\) pour
chaque expirations valides. Les moyennes de \(\tau\) et de \(\alpha\)
pour les deux populations de sujets normaux et BPCO ont été calculées.
Ces dernières étaient en accord avec les valeurs retrouvées dans la
littérature aussi bien pour les sujets normaux que BPCO. Aucune analyse
statistiques sur la comparaisons des moyennes des deux paramètres entre
les populations n'apparaient dans l'article. A ce stade, les auteurs
considèrent que les différences retrouvées suggèrent qu'il est possible
de concevoir un classifieur permettant de distinguer les deux
populations. Un groupe d'apprentissage a été constitué en selectionnant
aléatoirement 5 sujets normaux et 5 patients BPCO permettant d'analyser
1346 expirations valides et d'en extraire les paramètres \(\tau\) et de
\(\alpha\). Compte tenu que la distribution des paramètres présentaient
une distribution proche d'une loi log-normale, une analyse discriminante
linéaire {[}ref. 24{]} a été réalisée dans le plan
\(\ln(\tau)-\ln(\alpha)\). Cette dernière a permis d'établir un
classifieur linaire représenté par une droite permettant de séparer les
deux populations. La fonction \emph{fitcdiscr} de \emph{MATLAB} a été
employée à cette effet (dans ce but). Le classifieur résultant a obtenu
une précision de 89,5\% sur les 1346 expirations de l'ensemble
d'apprentissage. Par la suite le classifieur a été testé sur les 36
sujets restants, classant chaque expirations comme normale ou BPCO. Par
la suite pour chaque sujet a été établi la proportions d'expirations
classifés comme normale et comme BPCO. Cette proportion comparer à un
seuil a permis d'étiquetter (de ranger/ d'assigner/ de considérer/ de
classer) les patients dans l'un des deux groupes. Une courbe ROC, pour
estimer les pefromances du classifieurs, a été établi en faisant varier
la valeur du seuil. Les perfomances retrouvées étaient très bonne avec
une aire sous la courbe (AUC) de 0.99 (IC95\% 0.96-1.00) avec des taux
de vrais positifs très élevés tout en maintenant des taux de faux
positifs inférieur à 20\%.

Seconde partie : L'objectif de la seconde partie de cette étude était
d'appliqué le modèle à une maladie obstructive réversible. Pour cela, 22
patients présentants un asthme avec un test à la Métacholine positif ont
été recruté. Ce dernier était considéré comme positif lorsqu'il
induisait un bronchospasme mesuré par une chute inférieure à 80\% de la
valeur de référence du volume expiratoire maximal par seconde (VEMS) en
spirométrie. Dans ce test la métacholine, irritant pulmonaire, était
administrée par inhalation à dose croissante. Tous les sujets ont été
traité par de l'albuterol, un bronchodilatateur, à la fin du test à la
métacholine pour mettre en évidence la réversibilité du bronchospame. Un
enregistrement a été ralisé durant chacun des trois stades de l'examen :
avant et après l'administration de métacholine, puis après
administration d'un bronchodilatateur. Dans cette partie la valeur de la
\(p_{A}\) a été définit pour chaque patient comme la valeur maximale
d'EtCO2 mesurée au décours des trois stades. Ainsi, la valeur de la
\(p_{A}\) variait d'un patients à l'autre mais pas chez un même patient.
Les paramètres \(\tau\) et \(\alpha\) de chaque expiration, de chaque
stade et de chaque patient ont été extrait comme vu précédement. Les
médianes de \(\tau\) et de \(\alpha\) ont été calculées, observées et
comparées au cours des trois étapes. On observe que la valeur médiane de
\(\alpha\) après administration de métacholine est plus basse
comparativement au stade avant métacholine ou après albutérol. A
l'inverse, pour \(\tau\), la médiane augmente. Pour observer si ces
tendances se généralisaient à tous les patients, les proportions des
différences (des variations) entre deux médianes
(\(\Delta\tau/\tau_{0}\) et \(\Delta\alpha/\alpha_{0}\)) entre deux
stades de l'examen à la métacholine ( \emph{avant/après métacholine} et
\emph{après métacholine/après albutérol} ) ont été calculées. Ces
variations entre les médianes, représentés par des vecteurs en 2D dans
le plan \(\tau-\alpha\), ont été agrégés sur l'ensemble des 22 patients
et représentés sur la figure 11(7).

\begin{figure}[h!]

{\centering \includegraphics[width=250px]{figure/abid2017model_fig6} 

}

\caption{Proportions des variation entre les stades des valeurs médianes des 22 patients représentés sous forme de vecteur.}\label{fig:unnamed-chunk-11}
\end{figure}

Les changements de paramètres variaient d'un patient à l'autre,
peut-être parce que les patients ont commencé à des niveaux de référence
différents, ont répondu à la méthacholine à des degrés divers et n'ont
pas toujours répondu immédiatement. Cependant, de façon générale, on
observe que les modifications après adminitration de méthacholine ont
tendance à diminuer pour le paramètre \(\alpha\) et à augmenter pour le
paramètre \(\tau\), tandis que l'administration de bronchodilatateur a
eu l'effet inverse. Comme attendu par les auteurs, la direction des
changements de paramètres dus à la méthacholine était largement dans la
direction opposée aux déplacements dus à l'albutérol. Ces directions de
modifications vont dans le sens des données de la littérature avec une
augmentation de résistance des poumons lorsque \(\tau\) augmente
correspondant à une diminution du calibre des voies aériennes, et une
augmentaiton du rapport entre espace mort physiologique et volume
courant entrainant une diminution de \(\alpha\).

\hypertarget{justification-du-modele}{%
\paragraph{Justification du modèle}\label{justification-du-modele}}

Airflow Process : De nombreux modèles mécanistiques ont été proposés
dans la littérature pour décrire le flux d'air expiratoire (sortant des
poumons) en fonction de la pression motrice{[}13{]}. Un modèle a été
utilisé. Il correspond à considérer le poumon comme un partiment unique
correspondant à une alvéole, de volume variable associé à un volume
d'espace mort anatomique de volume constant.Le compartiment alvéolaire
étant élastique emmagansinant de l'énergie potentiel lors de
l'inspiration, est caractérisé par sa compliance \(C_{l}\). Ce dernier
est connecté au compartiement de l'espace mort anatomique qui correspond
au voies de conductions (bronches et trachées) et l'oropharynx. Il se
caractérise pas sa résistance à l'air \(R_{l}\). L'expiration se fait de
façon passive à l'état stable. Lors de l'inspiration, le compartiement
alvéolaire est distendu entrainant une expension de volume grace au
muscle respiratoire. Cela a pour effet d'emmagasiner de l'energie
potentiel qui sera restituée pour fournir la pression motrice lors de
l'expiration. Il est a noté que la paroi thoracique comprenant les côtes
et les muscles intercostaux ont une tendance à générer une force en
faveur d'une distention thoracique (augmentantion de volume).
Inversement le tissu pulmonaire par ces propriétés elastiques a tendance
à générer une force de rétraction en faveur d'un affaissement du poumon.
Le système paroie thoracique - tissu pulmonaire trouve son équilibre à
la Capacité Résiduelle Fonctionnelle \emph{CRF}. La mise en série des
deux compartiement forme un système décrivant le flux expiratoire qui
est comparable à un circuit \_LRC- électrique constitué d'une résitance,
d'une inductance et d'un condensateur. Ici il n'y a pas d'inductance
donc plutôt du type \emph{RC}. L'équation différentielle de ce système,
du flux d'air résultant \(\dot{V}(t)\), s'écrit :

\[ 
\frac{\mathrm{d}\dot{V}(t)}{\mathrm{d}t}=-\frac{\dot{V}(t)}{R_{l}C_{l}}+\frac{\dot{P}(t)}{R_{l}}
\]

\begin{equation}
  \frac{\mathrm{d}\dot{V}(t)}{\mathrm{d}t}=-\frac{\dot{V}(t)}{R_{l}C_{l}}+\frac{\dot{P}(t)}{R_{l}}
\label{AirflowProcess}
\end{equation}

Gas-mixing Processe : Les voies de conduction ne présentent pas de CO2,
seul le compartiment alvéolaire est générateur de CO2 en quantité
importante en raison des échanges gazeux avec les capillaires
pulmonaires. Au cours de l'expiration, le gaz riche en CO2 des alvéoles
se mélange au gaz pauvre en CO2 de l'espace mort. Nous supposons que ce
mélange est instantané car il se produit un mélange important dans
l'espace mort par turbulence et ramification des voies respiratoires
{[}10, 13{]}. Les auteurs de l'articles ont écrit une équation de
transport de masse pour la pression partielle de CO2 dans l'espace mort
en considérant que l'espace mort initialement ne comporte pas de CO2
comme dit précédement. Le mélange du CO2 provenant du compartiment
alvéolaire \(p_{A}\) se fait instantanément dans le compartiment
alvéolaire et augmente au fur à mesure du temps lors de l'expiration et
de la vidange des alvéoles. L'équation différentielle de la pression
partielle de CO2 dans l'espace mort en fonction du temps lors de
l'expiration \(p_{D}\) est la suivante :

\[ 
\frac{\mathrm{d}p_{D}(t)}{\mathrm{d}t}=\frac{-p_{D}(t)+p_{A}}{V_{D}} \ \dot{V}(t) 
\]

\begin{equation}
  \frac{\mathrm{d}p_{D}(t)}{\mathrm{d}t}=\frac{-p_{D}(t)+p_{A}}{V_{D}} \ \dot{V}(t)
\label{GasMixingProcess}
\end{equation}

Où \(p_{A}\) représente la pression partielle en CO2 du compartiement
alvéolaire et qui est considéré comme constante par les auteurs avec un
valeur prise à 40 mmHg.

Solution for a Step-Function Pressure Drive : Les équations
différentielles \eqref{AirflowProcess} et \eqref{GasMixingProcess}
peuvent être résolues pour donner les la fonction de \(P(t)\)

\[ 
p_{D}(t)=p_{A} (1-\mathrm{e}^{-\alpha} \mathrm{e}^{\alpha\mathrm{e}^{-\frac{t}{\tau}}})  
\tag{2}
\]

\begin{equation}
  p_{D}(t)=p_{A} (1-\mathrm{e}^{-\alpha} \mathrm{e}^{\alpha\mathrm{e}^{-\frac{t}{\tau}}})
\label{eq:modèle mécanique 2}
\end{equation}

Le paramètre \(\tau=R_{l}C_{l}\) représente la constante de temps
pulmonaire. Le parmètre \(\alpha\) définit par
\(\alpha= \frac{C_{l}\Delta P}{V_{d}} =\frac{V_{T}}{V_{D}}\) où la
quantité \(C_{l}\Delta P=V_{T}\) représente le volume courant (en
supposant une compliance pulmonaire linéaire), donc \(\alpha\) est
approximativement l'inverse de la fraction de l'espace mort pulmonaire

\hypertarget{bayesian-tracking-of-a-nonlinear-model-of-the-capnogram-den2006bayesian}{%
\subsubsection{\texorpdfstring{Bayesian Tracking of a Nonlinear Model of
the Capnogram
{[}\protect\hyperlink{ref-den2006bayesian}{47}{]}}{Bayesian Tracking of a Nonlinear Model of the Capnogram {[}47{]}}}\label{bayesian-tracking-of-a-nonlinear-model-of-the-capnogram-den2006bayesian}}

En 2006, Buijs et al.~ont présenté une méthode Bayésienne pour
l'identification de capnogramme volumétrique respiration après
respiration (35). La méthode intègre un modèle d'échange de CO2 dans les
poumons, en raison de la nature non-linéaire de la respiration humain.
Ce modèle représente le flux respiratoire et intègre un compartiment de
réinhalation, un compartiment espace mort des voies aériennes et un
compartiment représentant les alvéoles. En utilisant ce modèle une
prédiction pour la tension de CO2 alvéolaire non mesurée pourrait être
faite. Basé sur des modèles de simulations, il est montré que
l'algorithme adaptatif est capable de suivre rapidement les fluctuations
de mesures du CO2. La structure du modèle présentée cette recherche est
basée sur la connaissance physiologique a priori, où les paramètres du
modèle ont des significations physiques claires. Ceci a l'avantage que
le modèle peut être utilisé pour aider dans l'interprétation du
capnogramme. Cependant, si la méthode actuelle veut aider le clinicien
dans l'interprétation des données de capnographie, il est nécessaire de
fournir en directe des paramètres physiologiquement pertinents qui
seront déterminant pour l'analyse. Montre la simulation d'un
capnogramme, utilisant des paramètres modèles par défaut. Cependant, des
modèles physiques précis contiennent souvent des non-linéarités, comme
des variabilités dans le temps du flux d'air et de volume alvéolaire,
aussi bien que l'alternance entre l'inspiration et l'expiration dans le
modèle respiration de flux. À cause de ces non-linéarités,
l'identification des paramètres du modèle exige un certain type
d'algorithme d'optimisation(36). Ces algorithmes sont souvent hors
connexion (en mode autonome) et exigent la supposition que les
paramètres du modèle soient constants au cours de la période d'analyse.
De plus, la division en compartiments simples de l'espace mort des voies
aériennes et du volume alvéolaire semble être une schématisation
grossier, compte tenu que les voies aériennes humaines sont constitués
de nombreuses bronches et de millions d'alvéoles, avec probablement des
propriétés différentes en termes d'échange gazeux. En outre, le taux
métabolique de CO2 est un paramètre condensé ou composite, qui reflète
les modifications du flux sanguin par exemple pulmonaire, la production
de CO2 des tissus et l'hétérogénéité des rapports ventilation perfusion
alvéolaires. Le modèle ne permet pas de distinguer l'importance de
l'influence de ces différents phénomènes physiologiques. Il est donc
nécessaire, pour envisager une utilisation clinique future, de réaliser
de plus amples études sur des données tant simulées qu'expérimentales
qui permettront mieux définir avantage et les limites de ce méthode.

\hypertarget{online-classification-of-capnographic-curves-using-artificial-neural-networks-bleil2009online}{%
\subsubsection{\texorpdfstring{Online-Classification of Capnographic
Curves Using Artificial Neural Networks
{[}\protect\hyperlink{ref-bleil2009online}{62}{]}}{Online-Classification of Capnographic Curves Using Artificial Neural Networks {[}62{]}}}\label{online-classification-of-capnographic-curves-using-artificial-neural-networks-bleil2009online}}

Le but de l'algorithme de classification par corrélation est d'isoler
les capnogrammes analysables des échantillons de bruit. Un algorithme de
seuil simple sélectionne des capnogrammes individuels à partir de la
courbe de CO2 et les normalise à une longueur uniforme de 25 points de
données. Ensuite, les similitudes des capnogrammes sont testées en les
corrélant les unes avec les autres. Deux capnogrammes sont considérés
comme suffisamment similaires avec un coefficient de corrélation
supérieur à 0,9. Des capnogrammes similaires sont stockés dans une
catégorie, appelée cluster. Chaque grappe est représentée par un
capnogramme modèle, résultant de la moyenne de tous les capnogrammes de
cette catégorie. Même si le nombre de grappes peut en principe varier,
ces travaux limitent le nombre de grappes à cinq. Il y a toujours un
cluster ``Corbeille'', qui contient tous les capnogrammes qui n'ont pas
ou très peu de ressemblance avec un autre cluster. Après un premier
traitement, les modèles sont inspectés visuellement. Les valeurs
aberrantes évidentes dans tout groupe, comme les courbes sans structure,
sont immédiatement affectées au groupe ``Corbeille''. Au cours de
l'exécution de l'algorithme de cluster, les modèles ont tendance à
changer constamment car ils ne représentent qu'une moyenne. Pour
maintenir leur similarité avec le cluster, ils sont également vérifiés.
En cas de forte similarité de deux modèles, les clusters d'origine sont
fusionnés. Les capnogrammes du groupe ``Corbeil'' sont régulièrement
réévalués avec les modèles existants et, si nécessaire, re-classés.

\hypertarget{a-scoring-system-for-capnogram-biofeedback-preliminary-findings-landis1998scoring}{%
\subsubsection{\texorpdfstring{A scoring system for capnogram
biofeedback: preliminary findings
{[}\protect\hyperlink{ref-landis1998scoring}{63}{]}}{A scoring system for capnogram biofeedback: preliminary findings {[}63{]}}}\label{a-scoring-system-for-capnogram-biofeedback-preliminary-findings-landis1998scoring}}

\hypertarget{recognition-of-cardiogenic-artifact-in-pediatric-capnograms-smith1994recognition}{%
\subsubsection{\texorpdfstring{Recognition of cardiogenic artifact in
pediatric capnograms
{[}\protect\hyperlink{ref-smith1994recognition}{64}{]}}{Recognition of cardiogenic artifact in pediatric capnograms {[}64{]}}}\label{recognition-of-cardiogenic-artifact-in-pediatric-capnograms-smith1994recognition}}

Une des principale problématique en anesthésie pour la surveillance
ventilatoire en utilisant la capnographie est l'apparition d'artefact.
Ces derniers sont généralement dû à des oscillations cardiogénqiues
entraiant un défaut de reconnaissance de l'EtCO2 mais également de la
fréquence respiratoire. Un capnographie de type \emph{Engström Eliza
capnographe} a été utilisé pour réaliser les enregistrement par voie
nasale sur des sujets présentant un syndrome obstructif. Le but de
l'étude était d'évaluer un algorithme de reconnaissance et de correction
des artéfacts par oscillations cardiogéniques, et de le comparer à un
autre système de détrection des artéfacts : CAPNOG. Les artéfacts par
oscillations cardiogérniques sont facillement reconnaissable à partir de
capnogramme seul. Il présentent les caractéristiques suivantes : ils ont
toujours une forme lisse, avec une montée et une descente rapide, sans
aucun plateau inspiratoire ni expiratoire. Le pic de chaque oscillation
est inférieur au pic précédent, bien que la première puisse parfois être
légèrement supérieure à la valeur d'EtCO2 de la respiration dont elle
est issue. Le minimum de la première oscillation est toujours supérieur
au minimum de la vraie phase inspiratoire précédente. Avec les
oscillations suivantes, ce minimum diminue généralement jusqu'à
atteindre les niveaux de gaz inspiré. La caractéristique la plus
spécifique des oscillations cardiogéniques est leur courte durée, avec
une fréquence égale à la fréquence cardiaque sous-jacente. L'algorithme
développé repose sur l'association de chaque expiration associée à
l'inspiration suivante en un cycle. L'algorithme analyse le tracé de
capnographie cycle par cycle en utilisant comme paramètres l'EtCO2, les
temps inspiratoire (Tb) et expiratoire (Ta) ,et le taux de CO2 minimal
lors de la phase inspiratoire (Minco2). L'algorithme compare les
respirations entre elle en réalisant un système de classement et de
validation en cycles normaux et en cycles présentant des artéfacts en
rapport avec des oscillations cardiogréniques. Lorsque l'agorithme
reconnait un clycle comme un artéfact d'oscillation cardiogénique, ce
dernier est intégré au dernier cycle valide en intégrant son temps et en
lui appliquant l'EtCO2 du cylce précédent. Ce cycle respiratoire
reconstitué sera considéré comme le dernier cycle valide. Pour qu'un
cycle soit identifier comme une oscillation cardiogénique, il doit
satisfaire les l'ensemble des critères que resprésente les conditions 1
à 4. Cette reconnaissance (analyse) se fait en deux étapes. Une première
applique les conditions 1 et 2 à l'ensemble des cycles. Secondairement,
les condition 3 et 4 ne sont appliqués que si le dernier souffle lu a
été évalué comme étant un véritable souffle.

\begin{itemize}
\tightlist
\item
  Condition 1 : Last-valid-breath.Tb + Current-breath.Ta
  \(\leqslant 1.2 sec\)
\item
  Condition 2 : Current-breath.EtCO22 \(\leqslant\)
  Last-valid-breath.EtCO2 + 0.15\%CO2
\item
  Condition 3 : Current-breath.Ta \textless{} Last-valid\_breath.Ta
\item
  Condition 4 : Current-breath.Minco2 + 0.15\%CO2 \textgreater{}
  Last-valid\_breath.Minco2
\end{itemize}

Cet algorithme a été dérivé d'un travail réalisé chez des enfants. Le
temps utilisé pour le critère 1 nécessiterait d'être adapté pour être
utilisé chez l'adulte ou le nourrisson. Par la suite les performances de
l'algorithme ont été testé sur un jeu de données qui n'a pas été utilisé
pour constituer l'algorithme. Un anesthésiste confirmé et habitué à
reconnaitre les artéfacts lié aux oscillations cardiogéniques a classé
manuellement les capnogrammes comme ``cardiogénique'' ou ``réel''. Ce
classement a également été réalisé à l'aide du programme CARDOS, de la
même manière. Au total 1698 cycles respiratoires ont été analysés. Seul
13 cas présentaient une différence de classification entre l'algorithme
et la classification manuel, avec dans 12 cas une identification par
l'algorithme comme des oscillations cardiogéniques lorsque
l'anesthésiste les cassaient comme ``réel''. L'algorithme atteint une
sensibilité de 99,6\%, une spécificité de 99,2\%, une valeur prédictive
positive de 95,0\%, une valeur prédictive négative de 99,9\%.

\hypertarget{calculating-the-effect-of-altered-respiratory-parameters-on-capnographic-indices-roy2007calculating}{%
\subsubsection{\texorpdfstring{Calculating the Effect of Altered
Respiratory Parameters on Capnographic Indices
{[}\protect\hyperlink{ref-roy2007calculating}{13}{]}}{Calculating the Effect of Altered Respiratory Parameters on Capnographic Indices {[}13{]}}}\label{calculating-the-effect-of-altered-respiratory-parameters-on-capnographic-indices-roy2007calculating}}

Le but de cet article était d'établir la relation quantitative entre les
paramètres physiques du système respiratoire et les indices de forme du
capnogramme. Pour atteindre cet objectif, un modèle mathématique non
linéaire de l'échange de CO2 dans les poumons a été simulé dans diverses
conditions. Ce modèle prenait en compte la respiration avec l'évolution
du CO2 en fonction du temps. Il est modélisé par 3 compartiements : un
compartiment de réinspiration, un compartiment d'espace mort des voies
respiratoires et un compartiment représentant les alvéoles. Par la
suite, les indices de forme ont été calculés à partir des simulations au
capnogramme. Les présents résultats pourraient constituer une ligne
directrice pour les cliniciens dans l'interprétation physiologique du
capnogramme.

\begin{itemize}
\tightlist
\item
  phase II (indice 1) : supérieur à T1 et compris entre EtCO2 10 et 50\%
  (0.1 et 0.5)
\item
  phase III (indice 2) : supérieur à T1 et correspondant à l'interval ou
  la dérivée seconde est inférieur à 15 mmHg/sec2
\item
  phase 0 (indice 3 : inférieur à T1 et correspondant à la portion entre
  EtCO2 75 et 25\% (0.75 et 0.25)
\end{itemize}

Le modéle suivant a été utilisé pour simuler le débit d'air avec
l'évolution du CO2 au cour du temps : \[ 
 \dot{V} =     \left\{ \begin{array}{rcl}
         V_{T}/T_{1} & t\leqslant T_{1}
         \\ -\gamma V_{T}\mathrm{e}^{-\gamma(t-T_{1})} & T_{1}<t\leqslant T
                \end{array}\right.
\]

Les modifications de la distribution ventilation-perfusion V/Q
influencent radicalement la forme du capnogramme {[}5{]}. Dans les
poumons normaux, une seule constante de temps suffit pour se rapprocher
de la vidange alvéolaire. Cependant, chez les patients atteints de
maladie pulmonaire obstructive, plusieurs constantes (taux) de vidange
sont nécessaires pour décrire l'expiration {[}12{]}. Cette modélisation
est nécessaire pour exprimer l'hétérogénéité de vidange du compartiement
alvéolaire des patients obstructif. Pour simuler l'hétérogénéité V/Q, la
phase expiratoire a été divisée en deux phases: une phase courte de TI
jusqu'à TE,1 avec une constante de vidage rapide \(\gamma1\); par la
suite, une phase plus longue de vidange lente des voies respiratoires
obstruées avec une constante \(\gamma2\). L'équation de débit d'air de
la phase expiratoire a été adaptée de la manière suivante:

\[ 
 \dot{V} =     \left\{ \begin{array}{rcl}
         -\gamma_{1}V_{T}\mathrm{e}^{-\gamma_{1}(t-T_{I})} & T_{I}<t\leqslant T_{E,1}
         \\ -\gamma_{2}V_{T_{E,1}}\mathrm{e}^{-\gamma_{2}(t-T_{E,1})} & T_{E,1}<t\leqslant T
                \end{array}\right.
\]

\(V(T_{E,1})\) est le débit d'air au temps \(T_{E,1}\) En utilisant les
paramètres suivant pour la simulation : \(T_{E,1} = 0.2 sec\),
\(\gamma1= 1.0 sec^{-1}\) et \(\gamma2= 2.0 sec^{-1}\). Les autres
paramètres ayant été conservés avec leur valeurs par défaut. L'angle
\(\alpha\) a augmenté à 139° pour cette simulation comme on peut le
retrouver dans la littérature chez un patient présentant un
bronchospasme. Ce modèle permet une bonne simulation des courbes de
capnographie aussi bien pour des patients normaux que pour des patients
obstructif moyennant une adapatation en considérant 2 constantes de
temps pour la vidange du compartiment alvéolaire et simulé
l'hétérogénéité du rapport ventilation pefrusion. Ce travail permet des
faire un lien intéressant antre les modifications de la courbe de
capnographie et la physiopathologie pouvant expliquer ces modifications.
Une hypothèse est que les alvéoles peuvent être modélisées par un nombre
limité de compartiments. Ceci est une simplification évidente, car les
voies respiratoires humaines se composent de millions d'alvéoles.
Celles-ci peuvent toutes avoir des propriétés différentes en termes de
vidange ainsi que de ratios d'échange de gaz et de
ventilation-perfusion. Une autre hypothèse du modèle est que les
tensions alvéolaires et artérielles de CO2 sont équilibrées. Par
conséquent, les limitations de diffusion du CO2 sur la membrane
alvéolaire-capillaire sont négligées. Bien que ces hypothèses puissent
être justifiées dans des circonstances physiologiques normales, elles
peuvent toutefois être moins valables dans des états pathologiques. Les
résultats de la simulation de différentes constantes de temps
alvéolaires indiquent que la forme d'onde du flux d'air influence
grandement la forme du capnogramme. Cette forme d'onde peut-être
différente en fonction du patient et s'il présente ou non une pathologie
pulmonaire et en autre obstructive.

\hypertarget{influence-of-chest-compression-artefact-on-capnogram-based-ventilation-detection-during-out-of-hospital-cardiopulmonary-resuscitation-leturiondo2018influence}{%
\subsubsection{\texorpdfstring{Influence of chest compression artefact
on capnogram-based ventilation detection during out-of-hospital
cardiopulmonary resuscitation
{[}\protect\hyperlink{ref-leturiondo2018influence}{65}{]}}{Influence of chest compression artefact on capnogram-based ventilation detection during out-of-hospital cardiopulmonary resuscitation {[}65{]}}}\label{influence-of-chest-compression-artefact-on-capnogram-based-ventilation-detection-during-out-of-hospital-cardiopulmonary-resuscitation-leturiondo2018influence}}

Les cas d'ACR ont été enregistrés avec des défibrillateurs-moniteur
Heartstart MRx (Philips, USA), équipés de la technologie de rétroaction
de RCP en temps réel \emph{(Q-CPR)}. La capnographie a été enregistré à
l'aide d'un capteur Sidestream (Microstream, Oridion Systems Ltd.,
Israël). La ventilation était assurée par BAVU avec réserve après
intubation. Les choix pour ce dernier étaient la sonde endotrachéale ou
le King LT-D (supraglottique). Les signaux de défibrillateur utilisés
dans l'étude étaient le capnogramme, le signal de profondeur de
compression (CD) mesuré par le pavillon thoracique Q-CPR et le signal
d'impédance transthoracique (TI) acquis à partir des électrodes de
défibrillation. Les enregistrements inclus dans l'études présentaient un
minimum de 500 compressions cardiaques et 20 minutes d'enregistrement
continues sur l'ensemble des signaux. Sur ces critères d'inclusions 301
cas d'ACR ont pu être inclus. Les intervalles avec un signal TI brut ou
un capnogramme non fiable causé par des déconnexions ou un bruit
excessif ont été ignorés/éliminés. Pour chaque épisode, les capnogrammes
ont été décalés dans le temps pour compenser le retard par rapport aux
signaux CD et TI. Trois ingénieurs biomédicaux expérimentés dans
l'analyse des signaux de défibrillateurs hors domicile ont participé au
processus d'annotation. Ils ont examiné conjointement un tiers des cas
et défini les règles d'annotation permettant d'identifier les
capnogrammes déformés par un artefact de compression thoracique et
d'annoter les ventilations à l'aide du signal TI. Les autres épisodes
ont été divisés au hasard en trois parties, chacune examinée par un seul
examinateur. Au terme de ce processus, les trois experts se sont à
nouveau réunis pour résoudre par consensus les annotations indécises.
Les experts ont annoté les intervalles dans lesquels les capnogrammes
étaient déformés par un artefact de compression thoracique, avec l'aide
du signal CD. Les épisodes ont été classés comme étant déformés si un
artefact de compression thoracique apparaissait pendant plus de 1 minute
du temps de compression thoracique. En outre, ils ont annoté la
localisation de l'artéfact par rapport à la phase respiratoire (par
exemple, apparaissant principalement sur la phase expiratoire ou sur la
phase inspiratoire). Les ventilations ont été annotées manuellement en
utilisant la composante basse fréquence du signal TI. Un filtre
passe-bas a été appliqué au signal TI brut pour supprimer les
oscillations rapides causées par des compressions thoraciques et
améliorer les fluctuations lentes causées par les ventilations. La Fig.
1 (panneau supérieur) montre le signal TI brut en gris avec le composant
TI basse fréquence superposé en bleu. Chaque ventilation a été annotée à
l'instant correspondant à une augmentation de chaque fluctuation de l'IT
(marquée d'une ligne rouge pointillée verticale sur la Fig. 1). Le
capnogramme est représenté dans le panneau inférieur pour confirmer
visuellement la présence de ventilations. Les annotations obtenues ont
été utilisées comme référence pour tester les performances de
l'algorithme de détection de ventilation automatisé basé sur le
capnogramme. L'algorithme utilisé dans cette étude traite le capnogramme
et a été conçu selon un modèle de machine à états finis. La figure 2
montre l'organigramme de l'algorithme (en haut) et la définition des
principaux paramètres de l'algorithme (en bas). Le principe de
l'aglorithme se base sur la recherche de modification/changement brusque
de courbe de capnographie. Le début du capnogramme \(t_{up}^{i}\) est
détrecté lorsqu'une la valeur du capnogramme dépasse un seuil fixé
\(Th_{amp}\) en mmHg. De même pour la fin du capnogramme \(t_{dw}^{i}\)
lorsque la valeur du capnogramme passe sous le seuil le même seuil
\(Th_{amp}\). Pour détecter une ventilation, la durée de l'intervalle
\(D_{ex}=t_{dw}^{i}-t_{up}^{i}\) et la durée de l'intervalle
\(D_{in}=t_{up}^{i+1}-t_{dw}^{i}\) doivent dépasser les seuils
\(Th_{ex}\) et \(Th_{in}\), respectivement. Si les deux conditions sont
remplies, la ventilation est annotée au moment où survient le début de
l'inspiration, \(t_{dw}^{i}\). Pour prendre en compte les effets de
double ventilation observés (Fig. 2, en bas à droite), l'algorithme
élimine toute ventilation pour laquelle l'intervalle \(D_{in}\) est
inférieur à \(Th_{in}\) et recherche le prochain mouvement descendant et
montant jusqu'à ce que \(D_{in}\) dépasse \(Th_{in}\). La performance
diagnostique de l'algorithme a été évalué à l'aide des données annotés
par les experts comme valeur de référence. Les sensibilité et valeur
prédictive positive (VPP) ont été calculées. Se a été défini comme la
proportion de ventilations annotées détectées par l'algorithme. Le VPP
était la proportion de détections qui étaient bien des ventilations
annotées. Une tolérance de \(\pm0.5s\) entre les temps de détrections et
d'annotation a été accpeté. L'algorithme a été entraîné avec un
sous-ensemble de cas sans perturbation du signal de capnographie,
cherchant la sensibilité maximal tout en assurant une VPP \textgreater{}
98\%. Afin d'évaluer l'influence des artéfacts sur l'estimation du taux
de ventilation, nous avons calculé, pour chaque épisode, la valeur du
taux de ventilation par minute, actualisée toutes les 10 s. Ces mesures
du taux de ventilation ont été calculées à l'aide du gold-standard
(ventilations annotées) et des ventilations détectées par notre
algorithme. Nous avons également calculé des alarmes d'hyperventilation
à partir des mesures du taux de ventilation par minute. Les résultats
ont été obtenus pour des seuils d'hyperventilation fixés à 10, 15 et 20
par minute. Nous avons ensuite testé la capacité de notre algorithme à
détecter correctement l'hyperventilation. Dans ce cas, la Se a été
défini comme la proportion d'alarmes d'hyperventilation annotées
fournies par l'algorithme, et le VPP comme la proportion d'alarmes
d'hyperventilation donnée qui ont été effectivement annotées. Enfin, la
morphologie de l'artefact a été caractérisée par l'analyse spectrale de
capnogrammes propres et déformés/artéfactés. Nous avons calculé la
densité spectrale de puissance (DSP) du capnogramme et localisé les
composantes de fréquence/spectrales associées aux artefacts. Nous avons
utilisé le taux de compression thoracique dérivé du signal CD comme
référence. Sur les 301 capnogrammes selectionnés 69 ont été éliminés par
manque de fiabilité de l'enregistrement. Au final sur les 232 cas retenu
la durée moyenne des enregistrements était de 31 \((\pm9.5)\) min, avec
une moyenne de 2301 \((\pm1230)\) compressions thoraciques annotées par
enregistrement. Quatre-vingt-dix-huit enregistrement (42\%) étaient
artéfactés. Les artéfacts ont été classés en trois type : observé
principalement sur le plateau expiratoire du capnogramme (type I), au
niveau de la ligne de base (type II) et étendu au plateau et à la ligne
de base (type III). Après analyse spectrale pour caractériser la nature
en forme d'onde de l'artefact de compression thoracique un pic primaire
est clairement observé à 1,94 Hz, sans composantes harmoniques. Cette
valeur correspond à la fréquence fondamentale des artefacts,
\(f_{art}\), et correspond au taux de compression moyen dans cet
intervalle (\(f_{art}\) x 60 = 116 compressions par minute). Cela montre
que les artefacts est principalement des signaux sinusoïdaux et qu'ils
sont directement causés par des compressions thoraciques pendant la RCP.
L'ensemble des trois types d'artéfacts ont été retrouvé lorsqu'un
contrôle avancé des voies aériennes était en place quelqu'il soit
(intubation ou autre), avec une incidence plus élevé avec les
dispositifs supra-glottique. Ils n'étaient pas visible lors de
l'utilisation d'un masque de ventilation simple. L'algorithme de
détection de la ventilation a été entraîné avec un sous-ensemble de 30
cas sans artefacts. Les valeurs optimales pour les paramètres de
l'algorithme \(Th_{ex}\) et \(Th_{in}\) ont été obtenues pour un Se/PPV
de 99,8/99,0\%. Les performances de l'algotihme après application à la
population test globale étaient, en mediane (IQR), de 99.4\% (97.8--100)
pour la Se, et de 98.6\% (96.4--99.5) pour la VPP. Les tracès artéfactés
présentaient des performances moindre mais acceptable pour le artéfacts
de type I et II. Une franche dégradation des peformances de l'algorithme
étaient observées pour les artéfacts de type III avec pour la Se 85.2\%
(59.2--92.7) et la VPP 76.9\% (47.0--90.5). Des résultats similaires ont
été retrouvés concernant la détrection d'hypeventilation avec de moins
bonnes perfomances lorsque le seuil d'hyperventilation augmentait. Les
artéfacts de type III étaient également associé à une franche
dégradation des performances. La surveillance de la fréquence
ventilatoire est une des recommandations utilisant le capnogramme
pendant une RCP. Cependant, la présence d'oscillations haute fréquence
dans le capnogramme lors de compressions thoraciques peut compromettre
l'interprétation du signal de capnographie. En présence d'artéfacts de
type III, l'algorithme est pris à défaut avec une surestimation
importante de la fréquence ventilatoire. Compte tenu que les artéfacts
sont associés aux compressions thoraciques avec un aspect sinusauidale
associé à un fréquence spectrale identifiable, un pré-traitement par
filtrage semble envisageable. Ce filtrage avec pour objectif la
préservation de la forme d'onde du capnogramme, optimiserait très
probablement les performances de l'algorithme.

\hypertarget{enhancing-ventilation-detection-during-cardiopulmonary-resuscitation-by-filtering-chest-compression-artifact-from-the-capnography-waveform-gutierrez2018enhancing}{%
\subsubsection{\texorpdfstring{Enhancing ventilation detection during
cardiopulmonary resuscitation by filtering chest compression artifact
from the capnography waveform
{[}\protect\hyperlink{ref-gutierrez2018enhancing}{66}{]}}{Enhancing ventilation detection during cardiopulmonary resuscitation by filtering chest compression artifact from the capnography waveform {[}66{]}}}\label{enhancing-ventilation-detection-during-cardiopulmonary-resuscitation-by-filtering-chest-compression-artifact-from-the-capnography-waveform-gutierrez2018enhancing}}

\hypertarget{automatic-quantitative-analysis-of-human-respired-carbon-dioxide-waveform-for-asthma-and-non-asthma-classification-using-support-vector-machine-singh2018automatic}{%
\subsubsection{\texorpdfstring{Automatic Quantitative Analysis of Human
Respired Carbon Dioxide Waveform for Asthma and Non-Asthma
Classification Using Support Vector Machine
{[}\protect\hyperlink{ref-singh2018automatic}{67}{]}}{Automatic Quantitative Analysis of Human Respired Carbon Dioxide Waveform for Asthma and Non-Asthma Classification Using Support Vector Machine {[}67{]}}}\label{automatic-quantitative-analysis-of-human-respired-carbon-dioxide-waveform-for-asthma-and-non-asthma-classification-using-support-vector-machine-singh2018automatic}}

Cette nouvelle étude sur la reconnaissance des patients asthmatique et
non-asthmatique, se base dans le domaine temporel, sur le paramètre
principal de la surface sous la courbe. Plusieurs sufraces sous la
courbe sont mesurées/calculées et ainsi étudiées. Au total 5 paramètres
sont étudiés les aires sous la courbes : \emph{AR1+AR2}, \emph{AR3},
\emph{AR4} et \(\frac{\mathrm{d}CO_{2}}{\mathrm{d}t}\). Soixante treiz
patients ont été inclus dont 43 étaient des patients asthmatiques et 30
non-asthmatiques. Réalisé au sein du Département des Urgences de
l'Hopital de Penang en Malaisie, les enregistrements de capnographie ont
été réalisé à l'aide d'un appareil, basé sur la technologie Sidestream,
permettant l'échantillonage simple de la mesure du CO2 expiré en temps
réel, par le biais d'une canule nasale {[}voir\_20{]}. La fréquence
d'échantillonnage était de 100 Hz avec une durée d'enregistrement de 2,5
minutes. Un filtrage avec un un filtre à moyenne glissante a été
appliqué à la courbe pour lisser cette dernière.

Filtrage de la courbe par filtre à moyenne glissante :
\[y(p)=\frac{1}{2p+1}(y(p+q)+y(p+q-1)...+y(p-q))\]

\begin{equation}
  y(p)=\frac{1}{2p+1}(y(p+q)+y(p+q-1)...+y(p-q))
\label{eq:moyenne glissante}
\end{equation}

Un algorithme de seuil a été adopté pour segmenter automatiquement
chaque phase en sous-phases afin d'extraire les caractéristiques
suggérées dans des études antérieures. Le seuil utilisé était de 4 mmHg
permettant d'identifier et de limiter la plage correspondant à un cycle
respiratoire. Cette méthode de seuillage a été suggérée par You et Howe
{[}\protect\hyperlink{ref-you1994expiratory}{51}{]} qui indiquent que le
CO2 expiré doit provenir des poumons pour atteindre un niveau de 4 à 10
mmHg, ce qui correspond à une partie de la phase expiratoire ascendante.
La phase expiratoire a été indentifié comme la partie ascendante du
cycle respiratoire jusq'à sa valeur maximale en CO2. La phase
descendante (inspiration) a été considérée entre le point maximal de CO2
jusqu'à un seuil de CO2 de 10 et 4 mmHg, toujours à l'aide de la méthode
de seuillage. Les caractéristiques ont été extraites en fonction de
cette segmentation de la courbe et pour chaque cycle respiratoire (cycle
par cycle).

\[A_{i}=\frac{\mathrm{d}t}{6}\sum_{j=0}^{i}(C_{j-1}(t)+4C_{j}(t)+C_{j+1}(t))\]
Le caclul des différentes aires (\emph{AR1}, \emph{AR2}, \emph{AR3},
\emph{AR4}) sont calculté à l'aide de la formule : \begin{equation}
  A_{i}=\frac{\mathrm{d}t}{6}\sum_{j=0}^{i}(C_{j-1}(t)+4C_{j}(t)+C_{j+1}(t))
\label{eq:ARi}
\end{equation}

Où \(C(t)\) et \(\mathrm{d}t\) représentent respectivement la valeur du
CO2 et l'interval d'échantillonage. Le calcul des différentes aires est
fait en moyennant les aires de 6 cycles, 1 cylce avant et 1 cycle
considéré et 4 fois le cycle considéré.

La méthode de sélection des caractéristiques qui a été utilisée est
divisée/composée en trois catégories, à savoir une méthode intégrée
(intégrative), méthode d'encapsulation ou d'emblalage (d'encapsulation),
méthode de filtrage. Les methodes d'encapsulation et d'intégration
interagissent avec le classificateur et incluent une interaction entre
les sous-ensembles de fonctions et la sélection du modèle.

En outre, ils peuvent prendre en compte les dépendances de
fonctionnalités. Cependant, ces méthodes présentent certaines
limitations communes, telles qu'un risque plus élevé de surajustement
que les techniques de filtrage et un coût de calcul élevé. Contrairement
à ces méthodes, la technique de filtrage s'adapte facilement à une
grande feuille de données, est simple en calcul, rapide et ne dépend pas
du classificateur. En outre, l'algorithme de sélection de
caractéristiques ne peut être analysé qu'une seule fois et plusieurs
méthodes de classificateur peuvent être évaluées. Par conséquent, nous
avons proposé d'utiliser l'algorithme de sélection de caractéristiques
basé sur les filtres présenté précédemment dans {[}33{]} pour
sélectionner l'ensemble optimal de caractéristiques de signaux de CO2
permettant de différencier les condicitons d'asthme et de non-asthme. La
technique de filtrage évalue les caractéristiques en prenant en compte
les propriétés intrinsèques du signal de CO2. Il est basé sur le score
des caractéristiques significatives et les fonctionnalités présentant
des scores faibles sont supposées être supprimées d'un test ROC
{[}33{]}, qui représente une fraction des vrais positifs et des faux
positifs, pour un système de classification binaire. La courbe ROC a
considéré deux groupes spécifiques, asthmatique et non asthmatique, pour
chaque caractéristique. Ainsi, la courbe ROC coïncide avec l'AUC
diagonale (Az, 0,5), mais les caractéristiques peuvent différencier deux
groupes lorsque la courbe ROC atteint le coin supérieur gauche,
c'est-à-dire qu'elle est proche de 1. Habituellement, un AUC de 0,5, 0,7
à 0,8, 0,8 à 0,9 et plus de 0,9 préconise une discrimination qualifiée
respectivement de mauvaise , acceptable, excellente et remarquable,
comme le suggèrent Hosmer et Lemeshow, Ware. Ainsi, sur la base des
valeurs maximale et minimale de l'AUC, les caractéristiques ont été
choisies dans un but de classification.

Une Méthode Statistique de classifieur supervisé de type classifieur SVM
basé sur la fonction de base radiale (RBF) a été utilisé pour la
classification des patients. Il utilise un hyperplan pour différencier
les groupes asthmatiques et non asthmatiques en maximisant la distance
par rapport aux limites de décision.

\begin{center}\rule{0.5\linewidth}{\linethickness}\end{center}

\hypertarget{discussion}{%
\subsection{Discussion}\label{discussion}}

\hypertarget{population-et-pathologie-etudiees}{%
\subsubsection{Population et Pathologie
étudiées}\label{population-et-pathologie-etudiees}}

Patient normaux, obstructif, restrictif, fibrose. Problèmatique est
avoir des élements de mesures qui nous permets d'analyser et d'évaluer
plus finement le fonctionnement pulmonaire, à but diagnostique
(pathologie ou différencation entre pathologie BPCO et Insuffisance
Cardiaque par exemple), à but pronostique (évolution de la maladie dans
le temps, aggravation ou amélioration, comme surveillance, après
traitement), pour mieux adapter le traitement (problématique de
ventilation ou de perfusion, adapter le traitement en fonction du malade
et de sa pathologie). Les informations ou mesures permettants d'évaluer
la fonction pulmonaire : - FR (fréquence respiratoire) - VT (volume
courant) - Q (débit expiratoire) - SpO2 (\% des globules rouges
oxygénés) - PaO2 (pression partielle artérielle en O2) - VO2 (volume
d'O2 consommé) - PaCO2 (pression partielle artéreille en CO2) - EtCO2
(pression de CO2 en fin d'expiration) - VCO2 (volume de CO2 expiré) -
V/Q (rapport ventilation perfusion) - VD (espace mort : anatomique et
physiologique sont important) Les grands fonction du poumons : - la
ventilation (volume, fréquence, hétérogénéité de ventilation, dépendant
des résistances et de la compliance pulmonaire) - la perfusion
(débit/volume, hétérogénéité de perfusion) - la diffusion des gazs
(surface d'échange, résistances à la diffusion)

\hypertarget{pretraitement}{%
\subsubsection{Prétraitement}\label{pretraitement}}

\begin{itemize}
\item
  problèmatique d'analyse de la courbe. Deux modes principaux :
\item
  Un mode d'analyse (éparce) en découpant la courbe capnogramme par
  capnogramme. - Avec secondairement une analyse de chaque capnogramme
  pour un enregistrement pour avoir des mesures des caractéristiques de
  la courbe capnogrammme par capnogramme permettant d'en estimé une
  moyenne, une médiane et une déviation standard (ou un écart type).
  Problème de selection des capnogrammes correctes et abérrants. Mesure
  de la variablité : méthode et intérêt. - OU analyse après production
  d'un profil moyen en moyennant les courbes après supperposition
  centrées sur une valeur, un maqueur, un criètre des différents
  capnogramme facielement identifiable : une valeur fixé (15 mmHg comme
  chez Mieloszik
  {[}\protect\hyperlink{ref-mieloszyk2014automated}{61}{]}) ou la valeur
  maximale de la pente identifiable par le pic positif (valeur max) de
  la dérivée première.
\item
  Un mode d'analyse globale en prenant un compte un ensemble définit de
  la courbe avec plusieurs capnogramme analysé. Utilisé dans l'analyse
  par transformée de Fourier mais pourrait s'appliquer à l'analyse par
  transformée en ondelettes.
\end{itemize}

{[}\protect\hyperlink{ref-roy2007calculating}{13}{]} : méthode de
segmentation du capnogramme pour les différentes phases à étudier.

\hypertarget{features-ou-caracteristiques-analysees}{%
\subsubsection{Features ou caractèristiques
analysées}\label{features-ou-caracteristiques-analysees}}

\hypertarget{methode-de-danalyse-de-la-courbe}{%
\subsubsection{Méthode de d'analyse de la
courbe}\label{methode-de-danalyse-de-la-courbe}}

\begin{itemize}
\tightlist
\item
  mesure simple de paramètres (temps, EtCO2, pente, surface sous la
  courbe)
\item
  Ajustement par rappor à une équation (modéle mécanistique) * équation
  associé à un modèle mécanistique proche de la physiologie ou *
  équation décrivant le plus fidélement la courbe avec une meilleur
  adaptabilité au differente forme et déformation de la courbe (intérêt
  pour distinguer des pathologies BPCO/Emphysème/Asthme)
\item
  trasnfromée de Fourier (avec également la PSD : power spectral
  density)
\item
  transformée en ondelette
\item
  Bayesian model
\item
  nouveaux paramètres de Hjorth
\end{itemize}

\hypertarget{methode-de-classification}{%
\subsubsection{Méthode de
Classification}\label{methode-de-classification}}

\begin{itemize}
\tightlist
\item
  régression logistic
\item
  régression linéaire simple
\item
  ACP (analyse en composantes principales)
\item
  LPC analysis (linear predictive coding)
\item
  AR modeling (autoregressive modeling)
\item
  PSD (power spectral density)
\item
  GRBF network (gaussian radial basis function network)
\item
  l'Analyse Linéaire Discriminante (égale à la LPC)
\item
  SVM (séparateurs à vaste marge ou machine à vecteurs supports)
\item
  LDA (analyse linéaire discriminante)
\end{itemize}

\hypertarget{proposition-dapproche}{%
\subsubsection{Proposition d'approche}\label{proposition-dapproche}}

\begin{itemize}
\item
  voir si l'âge à une influence sur la courbe de capnographie chez des
  patients normaux.
\item
  l'incidence de la variablité de la courbe de capnographie, des
  capnogrammes au cours du temps. Effet perturbateur mais également
  intêret pour l'analyse et la classification. Variation plus importante
  ou moins importante des capnogrammes en fonction des pathologies ?
  (variabilité intra-classe ou intra-patient, variabilité inter-class ou
  inter-patient)
\item
  Analyse et comparaison de différent modèle mécanistique, différente
  équation de courbe.
\item
  Rercher de capnogramme typique en fonction des pathologies, l'associé
  à des ondelettes mères dans une transformée en ondelettes.
\item
  Approche ``multi-échelle'' ou ``multi-filtre'' : en utilisant des
  analyses différentes de la courbes pour en extraire des paramètres ou
  caractères différents pour identifier les différentes pathologies
  comme l'Insuffisance Cardiaque, les syndromes obstructifs dans leur
  divers aspect, les pathologies restrictives, les fibroses et les HTAP.
  Un peu comme du Deep Learning avec l'application de masque comme des
  réseaux convolutifs pour exprimer de façon les différentes
  pathologies. Par exemple, par une transformer de fourier, une
  transformer en ondelettes (en y associant l'inspiration), une approche
  mécanistique avec un ajustement de la courbe à un modèle, mesure de
  paramètres simple comme l'aire sous la courbe, etc\ldots{}
\item
  évaluation statistique comme le ``Compress Sensing'' : reconnaitre les
  maladies en fonction des capnogramme (+/- capnographie volumétrique et
  différence PaCO2-EtCO2)
\item
  évaluation des meilleurs critères(ou caractéristiques), des meilleurs
  association de critères(ou caractéristiques) comme des rapport de
  pentes SR=S2/S1x100
\item
  évaluation d'une opération mathématique sur les critères(ou
  caractéristiques) comme un ``Log'' pour mieux dispercé (étalé pour
  mieux séparer) l'information. Voir comment se comporte les paramètres
  (il est plus difficile de séparer les patients normaux des
  sub-pathologiques)
\item
  évaluation en appliquant un filtre ou non, entre filtre, ceux qui
  donnent les meilleurs pefromances diagnostiques.
\item
  combinaison de plusieurs paramètres d'analyse, de plusieurs type
  d'analyse (parametre, modele mécanistique, décomposition
  fréquencielle)
\item
  évaluation des paramètres ou algorithme pour reconnaitre et séparer
  les cycles respiratoires, les phases inspiratoires et expiratoire, les
  phases de la courbes, en vu de l'analyse.
\item
  gestion des artéfacts, filtrage ou élimination des artéfacts ou des
  capnogrammes artéfactés.
\item
  quel classifieur utiliser pour classer les patients.
\item
  évaluation de la sévérité du trouble obstructif, corrélation de
  paramètres avec le VEMS ou autres mesures spirométriques
\item
  analyse de la partie descendante de la courbe : l'inspiration qui
  peut-être informatif et associé à la compliance (comme dans les
  syndrome restrictif)
\end{itemize}

\hypertarget{hypothese-physiopathologique-et-capnographie}{%
\subsubsection{Hypothèse physiopathologique et
capnographie}\label{hypothese-physiopathologique-et-capnographie}}

Les débits réduits dans le syndrome obstructif prolongent la phase
transitionnelle entre l'espace mort anatomique et les gaz alvéolaire car
le volume de l'espace mort prend plus de temps à s'exhaler, ce qui fait
que la pente et l'angle de décollage diminuent en proportion directe de
la gravité de l'obstruction. Normalement l'évacuation des gaz
alvéolaires s'effectue de façon homogène avec une évacuation synchrone
des alvéoles qui présente une discrète hétérogénéité de concentration en
CO2. On peut considérer que l'ensemble des alvéoles présente une
concentration relativement homogène. Ce qui entraine une absence de
pente voir une pente minime lors de l'évacuation des gaz alvéolaire sur
le capnogramme. Dans l'OD, certaines bronches terminales sont rétrécies,
entraînant une hypoventilation locale des alvéoles et une augmentation
associée de leur concentration en CO2. Les alvéoles attachés à des
bronches terminales non obstruées sont relativement hyperventilés et ont
des concentrations réduites de CO2. Pendant l'expiration, les alvéoles
non obstruées se vident devant les alvéoles obstruées, conduisant à une
augmentation progressive de la concentration en CO2 pendant
l'expiration. Cette vidange inégale des alvéoles donne au plateau
alvéolaire une apparence caractéristique en pente dans la DO.

Dans l'asthme, l'obstruction des voies respiratoires provoque des
diminutions régionales du débit d'air et, par conséquent, de la
ventilation alvéolaire. Ceci est responsable de ``l'hétérogénéité
parallèle'' des rapports ventilation-perfusion (rapports V / Q). Sur le
capnogramme, cela provoque une déformation de la courbe normale, marquée
par la perte de verticalité de la phase 2, l'ouverture de l'angle entre
les phases 2 et 3 (angle alpha également appelé angle Q) et
l'augmentation de l'inclinaison de la phase 3.

{[}\protect\hyperlink{ref-babik2012effects}{68}{]}{[}\protect\hyperlink{ref-abid2015model}{11}{]}
La forme du capnogramme pourrait ne pas être uniquement lié à des
modifications de résistance mais également à la modification de la
compliance pulmonaire.

{[}\protect\hyperlink{ref-nassar2016capnography}{69}{]} * Maladies
pulmonaires obstructives Les maladies pulmonaires obstructives sont
caractérisées par des rapports V/Q très variables, produisant des
signatures capnographiques typiques (Fig. 4) .35 La transition de phase
II de l'espace mort anatomique avec les gaz alvéolaire est perturbée par
la contribution précoce d'unité à haut rapport V/Q, ce qui tend à
réduire la raideur de son ascension. Au cours de la phase III, les
unités avec différents rapports V/Q continuent de se vider de manière
désynchronisée (les régions à rapport V/Q plus élevées avec PACO2 basse
contribuent plus précocément, alors que les régions à faible rapport V/Q
avec une PACO2 plus élevées prédominent plur tardivement) . Cette
combinaison augmente l'angle alpha entre les phases II et III {[}36{]}.
Dans les maladies sévèrement obstructives, le capnogramme prend l'aspect
d'un aileron de requin (Fig. 5A). La pente de phase III étant plus raide
la valeur de l'EtCO2 en fin d'expiration dépent fortement du temps
expiratoire. Combinée à l'espace mort anatomique inhérent à l'emphysème,
cette caractéristique provoque des divergences marquées entre les
valeurs de la PCO2 terminale et artérielle. Ces caractéristiques
qualitatives des maladies obstructives sur la capnographie temporelle
correspondent à des aspects quantitatifs de la capnographie en volume.
Une augmentation de la pente de phase III a été démontrée dans la
plupart des maladies avec obstruction des voies aériennes (BPCO, asthme
et bronchectasie) {[}37,38{]}. Le degré de pente est en corrélation avec
la sévérité de l'obstruction du flux d'air mesurée par spirométrie
{[}37-42{]} et le degré d'emphysème observé sur les images
thoraciques{[}43{]}.

L'analyse automatisé de multiples indices capnographiques peut augmenter
la sensibilité et la spécificité du diagnostic des maladies pulmonaires
obstructives. Une combinaison de durée d'expiration, ET-CO2, pente
expiratoire et temps passé à ET-CO2 permet de distinguer les patients
atteints de BPCO de ceux souffrant d'insuffisance cardiaque congestive
{[}44{]}. La capnographie peut être encore plus sensible que la
spirométrie pour détecter les petites voies aériennes précoces
changements, comme cela a été démontré pour la technique similaire du
lavage à l'azote à un seul souffle {[}45,46{]}.

\begin{itemize}
\tightlist
\item
  Embolie Pulmonaire L'EP compromet la perfusion des alvéoles affectées
  avec beaucoup moins d'impact sur la ventilation, ce qui suggère que la
  capnographie pourrait être utile sur le plan diagnostique. Selon la
  physiologie décrite précédemment, on pourrait s'attendre à ce que le
  PE (1) augmente l'espace mort alvéolaire (et donc les fractions
  d'espace mort physiologique et alvéolaire), (2) abaisse la valeur de
  CO2 expirée tout au long de la phase III et ET-CO2 (puisque le gaz
  expiré par les alvéoles V/Q élevées contient peu de CO2) et (3)
  n'élèvent pas la pente de phase III (peut-être permettant une
  distinction entre BPCO et autres maladies des voies aériennes avec
  hétérogénéité V/Q) (Fig. 4). Contrairement à la capnographie
  temporelle, les liens de capnographie volumique expirent plus
  étroitement les valeurs de CO2 vers les poumons volumes et permet le
  calcul des fractions physiologiques de l'espace mort (en utilisant
  l'équation de Bohr modifiée et le gaz sanguin artériel simultané). En
  testant la valeur diagnostique chez des patients suspects d'EP, la
  plupart des études ont utilisé la capnographie volumique, mais des
  principes similaires sous-tendent les changements observés avec la
  capnographie {[}23{]}. Dans l'une des premières études, il a été
  montré que l'EP augmentait la fraction physiologique de l'espace mort
  et la différence de CO2 artériel d'extrémité terminale24. Ces
  résultats ont été confirmés dans des études ultérieures {[}25,26{]}.
  En outre, la fraction de l'espace mort alvéolaire est en corrélation
  avec le défaut de perfusion pulmonaire sur la scintigraphie pulmonaire
  et la pression artérielle pulmonaire sur l'imagerie angiographique26.
  Toute différence entre artères et ET-CO2 dépend du volume courant, ce
  qui a conduit plusieurs chercheurs de mesure en extrapolant la phase
  III à un volume courant égal à 15\% de la capacité pulmonaire totale
  prédite. La différence entre la valeur de CO2 expirée à ce volume
  extrapolé et la PaCO2 convient pour le calcul de la fraction d'espace
  mort mort. Quand elle dépasse 12\% à 15\%, cette valeur peut mieux
  séparer les patients avec et sans PE, encore plus que la fraction
  totale de l'espace mort ou le gradient ET-CO2 / PaCO2 {[}27-31{]}.
  D'autres {[}26{]} ont combiné la fraction d'espace mort mort ou ET
  -CO2 / PaCO2 gradient avec une valeur D-dimère négative pour réduire
  la probabilité post-test de PE suffisamment pour éviter d'autres tests
  de diagnostic. Les différences de PaCO2 et d'ET-CO2 sont courantes
  dans les cas de maladie grave, mais cette différence n'est causée
  qu'occasionnellement par l'EP. La plupart des causes alternatives sont
  des maladies pulmonaires avec une grande hétérogénéité V/Q et des
  élévations anormalement fortes au cours de la phase III pour les
  raisons précitées. En revanche, l'EP ne devrait pas entraîner une
  forte dégradation de la phase III (en fait, elle aplatit la pente pour
  des raisons complexes liées à la redistribution du flux sanguin) 32 et
  la thrombolyse augmente la pente33. à la DE avec suspicion d'EP, la
  pente de la phase III était significativement plus plate chez les
  patients avec PE confirmée {[}27{]}. Dans une analyse groupée de 14
  essais et de 2 291 sujets, la capnographie a montré une sensibilité de
  0,80 pour le diagnostic de l'EP, alors que la spécificité était de
  0,49,34. L'utilisation de la capnographie à des fins diagnostiques
  présente plusieurs limites. Premièrement, la plupart des données
  cliniques se rapportent à capnographie en volume; capnographie
  temporelle ne permettre pas le calcul de la fraction d'espace mort
  tardive à un volume courant normalisé. Un autre problème concerne
  l'état d'équilibre: l'EP peut perturber la circulation, entraînant des
  modifications de l'apport de CO2 dans les poumons qui invalident
  l'hypothèse de l'état d'équilibre. De plus, la plupart des études ont
  été menées sur des patients cliniquement stables aux urgences plutôt
  que sur ceux qui étaient ventilés mécaniquement dans une unité de
  soins intensifs. Enfin, dans la mesure où des modifications des
  paramètres du ventilateur ou une interaction patient-ventilateur
  peuvent modifier la forme d'onde capnographique, les artefacts de soin
  ne doivent pas être confondus avec des signaux de pathologie.
  Néanmoins, dans le bon contexte clinique, le capnogramme peut fournir
  des indices pour augmenter ou diminuer les suspicions d'EP, en
  particulier lorsqu'il est associé à une valeur négative de D-dimères.
\end{itemize}

{[}\protect\hyperlink{ref-mcswain2010end}{70}{]} Montre dans son article
que l'EtCO2 et la PaCO2 sont fortement corrélée pour des valeur de
ration VD/VT \textless{} 0.4 donc faible, avec un gradient en moyenne de
l'ordre de 0.3 mmHg. Par contre plus le rapport entre espace mort
physiologique sur volume courant augmente plus la corrélation baisse et
le gradient augmente jusqu'à 18 mmHg en moyenne pour un rapport VD/VT
\textgreater{} 0.7. Cela va dans le sens d'une augmentation de l'espace
mort physiologique lorsque le gardient entre EtCO2 et PaCO2 augmente. Ce
qui pourrait s'expliquer par une augmentation de l'hétérogénéité du
rapport ventilation-pefrusion V/Q.
{[}\protect\hyperlink{ref-lujan2008capnometry}{71}{]} Lujan retouve les
mêmes résultats avec des gradients entre PaCO2 et EtCO2 augmentés ches
les patients présentant un trouble obstuctif en spirométrie d'intensité
différente. Pour les patients ne présentant pas d'anomaie spirométrique
le gradient moyen étais de 1.7 +/- 2.9 mmHg et pour les patients
présentant un syndrome obstructif important les gradient était de 8.2
+/- 5.6 mmHg. {[}\protect\hyperlink{ref-roy2007calculating}{13}{]} : la
simulation de la distribution inhomogène de R V / Q a entraîné une
augmentation de l'angle A comparable aux observations chez les patients
présentant des voies respiratoires obstruées {[}2, 5{]}.

{[}\protect\hyperlink{ref-tusman2005effect}{72}{]} le CO2 présente des
caractéristiques particulières telles que sa grande solubilité, son
transfert inverse de l'O2 du sang à l'alvéole et son mécanisme de
transport complexe dans le sang. À cet égard, le SBT-CO2 dépend
davantage des changements dans les relations ventilation-perfusion, ce
qui rend presque impossible la différenciation des effets de la
ventilation sur ceux de la perfusion sur la forme de SIII. En
capnographie volumétrique, les résultats de cette étude montrent que SII
et SIII ont changé à mesure que le débit sanguin pulmonaire augmentait
progressivement au cours de la phase de sevrage du de la circulation
extra-corporelle. Cependant, lorsque SII et SIII ont été normalisés en
les divisant par le CO2 alvéolaire expiré moyen, ces pentes sont restées
inchangées. La normalisation des pentes de cette manière annule l'effet
de l'amplitude du débit de perfusion pulmonaire sur le SBT-CO2. Ainsi,
tout changement dans SnII et SnIII peut être interprété comme étant
causé par des altérations de la distribution de la ventilation et de la
perfusion, quelle que soit l'amplitude du débit sanguin pulonaire. Les
patients asthmatiques et emphysémateux (18, 25, 30) présentent une
diminution du SII par rapport aux sujets normaux. Chez les patients
ventilés mécaniquement avec une atélectasie induite par l'anesthésie,
nous avons montré qu'après une manœuvre de recrutement pulmonaire dans
laquelle le poumon est ré-expansé, le SII augmente (27). Le fait que le
recrutement du poumon augmente SII indique que la réouverture de petites
voies aériennes évidées entraîne une répartition plus homogène de la
ventilation. Par conséquent, une distribution inhomogène de la
ventilation semble influencer SII. Malgré l'influence connue de la
distribution de la ventilation sur SII, nos résultats ont montré que le
PBF a un effet supplémentaire sur cette pente. SII a montré ses valeurs
les plus élevées avant et après CPB. Les valeurs les plus faibles ont
été trouvées lors du débit de pompage le plus élevé (PBF le plus
faible), en observant une augmentation progressive à mesure que le PBF
augmentait. Nos résultats concernant le comportement de VTCO2,
concordent avec ceux observés dans la simulation théorique réalisée par
Schwardt et al. (24). Ces auteurs ont démontré que le débit cardiaque et
la tension veineuse de CO2 avaient une relation directe avec VTCO2, les
valeurs br, car la concentration alvéolaire moyenne en CO2 dépend
principalement de la vitesse à laquelle les molécules de CO2 sont
délivrées par le circulation pulmonaire. Les SII et SIII non normalisés
étaient affectés par les différents degrés de FBP, alors que les pentes
normalisées ne l'étaient pas. Les pentes normalisées décrivent les
inhomogénéités de ventilation et de perfusion chez les patients sous
ventilation mécanique, indépendamment des modifications de l'ampleur du
débit sanguin pulmonaire. Le rôle de la normalisation des pentes sur la
distribution des débit sanguin pulmonaire reste à déterminer.

\hypertarget{divers}{%
\subsubsection{Divers}\label{divers}}

\begin{itemize}
\item
  Critères de qualités de l'analyse de la courbe
\item
  Variabilités des courbes : paramètres, intra-class,
  intra-patient\ldots{}
\item
\end{itemize}

La pente de la phase III du capnogramme (SIII) concerne (est le reflet
de) la vidange progressive des alvéoles, un décalage de
ventilation/perfusion et une inhomogénéité de la ventilation. SIII
dépend non seulement de la géométrie des voies aériennes, mais également
de la compliance respiratoire dynamique (Crs).

Intérêt du CO2 est qu'il ne necessite pas un apport externe comme un
médicament que l'on injecterait. Il est présent constitutionnellement
dans le corps et surtout n'est pas dépendant de l'inspiration. Il est
présent avec une proportion négligeable dans l'air atmosphérique
contrairement à l'Azote N2 avec lequel est réalisé des tests
respiratoire pour mesurer l'hétérogénéité de ventilation. Une contrainte
avec le CO2 est que son taux artériel varie en fonction de son
évacuation par le poumon mais également en fonction de sa production.

Analyser ces signaux EEG se décompose en trois étapes successives qui
sont le prétraitement, l'extraction de caractéristiques et la
classification. - prétraitement - extraction de caractéristiques -
classification L'extraction de caractéristiques consiste à décrire les
signaux EEG par un ensemble essentiel (idéallement petit) de valeurs
décrivant l'information pertinente qu'ils contiennent, en vue de les
classifier.

\pagebreak

\begin{center}\rule{0.5\linewidth}{\linethickness}\end{center}

\hypertarget{conclusion}{%
\section{Conclusion}\label{conclusion}}

modification/apparence ``Aileron de Requin'' ou ``Shark Fin''

La forme d'onde capnographique montre des changements dans le
bronchospasme qui reflètent l'hétérogénéité de l'air expiré. Ceci est vu
comme une diminution de la pente de la phase 2 et une augmentation de la
pente de la phase 3 et l'augmentation résultante de l'angle alpha (angle
\(\alpha\)).

\pagebreak

\begin{center}\rule{0.5\linewidth}{\linethickness}\end{center}

\pagebreak

\begin{center}\rule{0.5\linewidth}{\linethickness}\end{center}

\hypertarget{annexes}{%
\section{Annexes}\label{annexes}}

\pagebreak

\begin{center}\rule{0.5\linewidth}{\linethickness}\end{center}

\hypertarget{mots-cles}{%
\section{Mots clés}\label{mots-cles}}

capnography waveforms

capnogram waveform

capnogram signal

\pagebreak

\begin{center}\rule{0.5\linewidth}{\linethickness}\end{center}

\hypertarget{divers-1}{%
\section{Divers}\label{divers-1}}

{[}\protect\hyperlink{ref-delmas2011readmissions}{73}{]}
{[}\protect\hyperlink{ref-delmas2010asthma}{74}{]}
{[}\protect\hyperlink{ref-delmas2017augmentation}{75}{]}
{[}\protect\hyperlink{ref-tattersfield2001asthma}{76}{]}
{[}\protect\hyperlink{ref-hart2002asthma}{77}{]}
{[}\protect\hyperlink{ref-salmeron2001asthma}{78}{]}

{[}\protect\hyperlink{ref-anto2001epidemiology}{79}{]}
{[}\protect\hyperlink{ref-charvanne2008bronchopneumopathie}{80}{]}
{[}\protect\hyperlink{ref-roche2007donnees}{81}{]}
{[}\protect\hyperlink{ref-fuhrman2007mortalite}{82}{]}

{[}\protect\hyperlink{ref-chilton1952mathematical}{20}{]}
{[}\protect\hyperlink{ref-yaron1996utility}{48}{]}
{[}\protect\hyperlink{ref-you1994expiratory}{51}{]}
{[}\protect\hyperlink{ref-krauss2005capnogram}{50}{]}
{[}\protect\hyperlink{ref-hisamuddin2009correlations}{45}{]}
{[}\protect\hyperlink{ref-mieloszyk2014automated}{61}{]}
{[}\protect\hyperlink{ref-kazemi2013frequency}{59}{]}
{[}\protect\hyperlink{ref-kazemi2016new}{60}{]}
{[}\protect\hyperlink{ref-betancourt2014segmented}{55}{]}
{[}\protect\hyperlink{ref-howe2011use}{31}{]}
{[}\protect\hyperlink{ref-egleston1997capnography}{49}{]}
{[}\protect\hyperlink{ref-smalhout1975atlas}{52}{]}
{[}\protect\hyperlink{ref-kelsey1962expiratory}{53}{]}

Site GLI \emph{(Global Lung Function Initiative)} :

\begin{verbatim}
          http://gligastransfer.org.au/calcs/tlco.html
\end{verbatim}

\pagebreak

\hypertarget{tableaux}{%
\section{Tableaux}\label{tableaux}}

\newpage
\begin{landscape}

\begin{center}\rule{0.5\linewidth}{\linethickness}\end{center}

\hypertarget{synthese-des-etudes-sur-la-capnographie-standard}{%
\subsection{Synthèse des études sur la Capnographie
Standard}\label{synthese-des-etudes-sur-la-capnographie-standard}}

\begin{longtable}[]{@{}ccccclll@{}}
\caption{Synthèse des études sur la Capnographie
Standard}\tabularnewline
\toprule
\begin{minipage}[b]{0.14\columnwidth}\centering
Ref.\strut
\end{minipage} & \begin{minipage}[b]{0.09\columnwidth}\centering
Auteur\strut
\end{minipage} & \begin{minipage}[b]{0.08\columnwidth}\centering
Année\strut
\end{minipage} & \begin{minipage}[b]{0.11\columnwidth}\centering
Phathologies\strut
\end{minipage} & \begin{minipage}[b]{0.08\columnwidth}\centering
Effectifs\strut
\end{minipage} & \begin{minipage}[b]{0.09\columnwidth}\raggedright
Prétraitement\strut
\end{minipage} & \begin{minipage}[b]{0.08\columnwidth}\raggedright
Méth. Caract.\strut
\end{minipage} & \begin{minipage}[b]{0.11\columnwidth}\raggedright
Caractéristiques\strut
\end{minipage}\tabularnewline
\midrule
\endfirsthead
\toprule
\begin{minipage}[b]{0.14\columnwidth}\centering
Ref.\strut
\end{minipage} & \begin{minipage}[b]{0.09\columnwidth}\centering
Auteur\strut
\end{minipage} & \begin{minipage}[b]{0.08\columnwidth}\centering
Année\strut
\end{minipage} & \begin{minipage}[b]{0.11\columnwidth}\centering
Phathologies\strut
\end{minipage} & \begin{minipage}[b]{0.08\columnwidth}\centering
Effectifs\strut
\end{minipage} & \begin{minipage}[b]{0.09\columnwidth}\raggedright
Prétraitement\strut
\end{minipage} & \begin{minipage}[b]{0.08\columnwidth}\raggedright
Méth. Caract.\strut
\end{minipage} & \begin{minipage}[b]{0.11\columnwidth}\raggedright
Caractéristiques\strut
\end{minipage}\tabularnewline
\midrule
\endhead
\begin{minipage}[t]{0.14\columnwidth}\centering
{[}\protect\hyperlink{ref-you1994expiratory}{51}{]}\strut
\end{minipage} & \begin{minipage}[t]{0.09\columnwidth}\centering
You \& al.\strut
\end{minipage} & \begin{minipage}[t]{0.08\columnwidth}\centering
1994\strut
\end{minipage} & \begin{minipage}[t]{0.11\columnwidth}\centering
\strut
\end{minipage} & \begin{minipage}[t]{0.08\columnwidth}\centering
\strut
\end{minipage} & \begin{minipage}[t]{0.09\columnwidth}\raggedright
\strut
\end{minipage} & \begin{minipage}[t]{0.08\columnwidth}\raggedright
\strut
\end{minipage} & \begin{minipage}[t]{0.11\columnwidth}\raggedright
\strut
\end{minipage}\tabularnewline
\bottomrule
\end{longtable}

\end{landscape}

\newpage
\begin{landscape}

\begin{center}\rule{0.5\linewidth}{\linethickness}\end{center}

\hypertarget{synthese-des-etudes-sur-la-capnographie-standard-1}{%
\subsection{Synthèse des études sur la Capnographie
Standard}\label{synthese-des-etudes-sur-la-capnographie-standard-1}}

\begin{longtable}[]{@{}ccccrrrr@{}}
\caption{Synthèse des études sur la Capnographie
Standard}\tabularnewline
\toprule
\begin{minipage}[b]{0.15\columnwidth}\centering
Ref.\strut
\end{minipage} & \begin{minipage}[b]{0.10\columnwidth}\centering
Auteur\strut
\end{minipage} & \begin{minipage}[b]{0.08\columnwidth}\centering
Année\strut
\end{minipage} & \begin{minipage}[b]{0.11\columnwidth}\centering
Périodiques\strut
\end{minipage} & \begin{minipage}[b]{0.10\columnwidth}\raggedleft
Performances\strut
\end{minipage} & \begin{minipage}[b]{0.10\columnwidth}\raggedleft
Gold Strandard\strut
\end{minipage} & \begin{minipage}[b]{0.07\columnwidth}\raggedleft
Algorithme\strut
\end{minipage} & \begin{minipage}[b]{0.08\columnwidth}\raggedleft
Méth. Class.\strut
\end{minipage}\tabularnewline
\midrule
\endfirsthead
\toprule
\begin{minipage}[b]{0.15\columnwidth}\centering
Ref.\strut
\end{minipage} & \begin{minipage}[b]{0.10\columnwidth}\centering
Auteur\strut
\end{minipage} & \begin{minipage}[b]{0.08\columnwidth}\centering
Année\strut
\end{minipage} & \begin{minipage}[b]{0.11\columnwidth}\centering
Périodiques\strut
\end{minipage} & \begin{minipage}[b]{0.10\columnwidth}\raggedleft
Performances\strut
\end{minipage} & \begin{minipage}[b]{0.10\columnwidth}\raggedleft
Gold Strandard\strut
\end{minipage} & \begin{minipage}[b]{0.07\columnwidth}\raggedleft
Algorithme\strut
\end{minipage} & \begin{minipage}[b]{0.08\columnwidth}\raggedleft
Méth. Class.\strut
\end{minipage}\tabularnewline
\midrule
\endhead
\begin{minipage}[t]{0.15\columnwidth}\centering
{[}\protect\hyperlink{ref-you1994expiratory}{51}{]}\strut
\end{minipage} & \begin{minipage}[t]{0.10\columnwidth}\centering
You \& al.\strut
\end{minipage} & \begin{minipage}[t]{0.08\columnwidth}\centering
1994\strut
\end{minipage} & \begin{minipage}[t]{0.11\columnwidth}\centering
\strut
\end{minipage} & \begin{minipage}[t]{0.10\columnwidth}\raggedleft
\strut
\end{minipage} & \begin{minipage}[t]{0.10\columnwidth}\raggedleft
\strut
\end{minipage} & \begin{minipage}[t]{0.07\columnwidth}\raggedleft
\strut
\end{minipage} & \begin{minipage}[t]{0.08\columnwidth}\raggedleft
\strut
\end{minipage}\tabularnewline
\begin{minipage}[t]{0.15\columnwidth}\centering
{[}\protect\hyperlink{ref-you1994expiratory}{51}{]}\strut
\end{minipage} & \begin{minipage}[t]{0.10\columnwidth}\centering
You \& al.\strut
\end{minipage} & \begin{minipage}[t]{0.08\columnwidth}\centering
1994\strut
\end{minipage} & \begin{minipage}[t]{0.11\columnwidth}\centering
\strut
\end{minipage} & \begin{minipage}[t]{0.10\columnwidth}\raggedleft
\strut
\end{minipage} & \begin{minipage}[t]{0.10\columnwidth}\raggedleft
\strut
\end{minipage} & \begin{minipage}[t]{0.07\columnwidth}\raggedleft
\strut
\end{minipage} & \begin{minipage}[t]{0.08\columnwidth}\raggedleft
\strut
\end{minipage}\tabularnewline
\bottomrule
\end{longtable}

\end{landscape}

\newpage
\begin{landscape}

\begin{center}\rule{0.5\linewidth}{\linethickness}\end{center}

\hypertarget{synthese-des-etudes-sur-la-capnographie-standard-2}{%
\subsection{Synthèse des études sur la Capnographie
Standard}\label{synthese-des-etudes-sur-la-capnographie-standard-2}}

\begin{longtable}[]{@{}cllccl@{}}
\caption{Synthèse des études sur la Capnographie
Standard}\tabularnewline
\toprule
\begin{minipage}[b]{0.09\columnwidth}\centering
Auteur\strut
\end{minipage} & \begin{minipage}[b]{0.05\columnwidth}\raggedright
Année\strut
\end{minipage} & \begin{minipage}[b]{0.08\columnwidth}\raggedright
Effectif\strut
\end{minipage} & \begin{minipage}[b]{0.14\columnwidth}\centering
Gold standard\strut
\end{minipage} & \begin{minipage}[b]{0.40\columnwidth}\centering
Caractéristiques\strut
\end{minipage} & \begin{minipage}[b]{0.08\columnwidth}\raggedright
Ref.\strut
\end{minipage}\tabularnewline
\midrule
\endfirsthead
\toprule
\begin{minipage}[b]{0.09\columnwidth}\centering
Auteur\strut
\end{minipage} & \begin{minipage}[b]{0.05\columnwidth}\raggedright
Année\strut
\end{minipage} & \begin{minipage}[b]{0.08\columnwidth}\raggedright
Effectif\strut
\end{minipage} & \begin{minipage}[b]{0.14\columnwidth}\centering
Gold standard\strut
\end{minipage} & \begin{minipage}[b]{0.40\columnwidth}\centering
Caractéristiques\strut
\end{minipage} & \begin{minipage}[b]{0.08\columnwidth}\raggedright
Ref.\strut
\end{minipage}\tabularnewline
\midrule
\endhead
\begin{minipage}[t]{0.09\columnwidth}\centering
babik\strut
\end{minipage} & \begin{minipage}[t]{0.05\columnwidth}\raggedright
2012\strut
\end{minipage} & \begin{minipage}[t]{0.08\columnwidth}\raggedright
68\strut
\end{minipage} & \begin{minipage}[t]{0.14\columnwidth}\centering
Res. \& Comp.\strut
\end{minipage} & \begin{minipage}[t]{0.40\columnwidth}\centering
Pente III\strut
\end{minipage} & \begin{minipage}[t]{0.08\columnwidth}\raggedright
{[}\protect\hyperlink{ref-babik2012effects}{68}{]}\strut
\end{minipage}\tabularnewline
\begin{minipage}[t]{0.09\columnwidth}\centering
blanch\strut
\end{minipage} & \begin{minipage}[t]{0.05\columnwidth}\raggedright
1994\strut
\end{minipage} & \begin{minipage}[t]{0.08\columnwidth}\raggedright
41\strut
\end{minipage} & \begin{minipage}[t]{0.14\columnwidth}\centering
Résistance\strut
\end{minipage} & \begin{minipage}[t]{0.40\columnwidth}\centering
Pente III\strut
\end{minipage} & \begin{minipage}[t]{0.08\columnwidth}\raggedright
{[}\protect\hyperlink{ref-blanch1994relationship}{83}{]}\strut
\end{minipage}\tabularnewline
\begin{minipage}[t]{0.09\columnwidth}\centering
brown\strut
\end{minipage} & \begin{minipage}[t]{0.05\columnwidth}\raggedright
2013\strut
\end{minipage} & \begin{minipage}[t]{0.08\columnwidth}\raggedright
20\strut
\end{minipage} & \begin{minipage}[t]{0.14\columnwidth}\centering
Spiro. \& TDM\strut
\end{minipage} & \begin{minipage}[t]{0.40\columnwidth}\centering
Pente III\strut
\end{minipage} & \begin{minipage}[t]{0.08\columnwidth}\raggedright
{[}\protect\hyperlink{ref-brown2013forced}{84}{]}\strut
\end{minipage}\tabularnewline
\begin{minipage}[t]{0.09\columnwidth}\centering
egleston\strut
\end{minipage} & \begin{minipage}[t]{0.05\columnwidth}\raggedright
1997\strut
\end{minipage} & \begin{minipage}[t]{0.08\columnwidth}\raggedright
38\strut
\end{minipage} & \begin{minipage}[t]{0.14\columnwidth}\centering
Clinique\strut
\end{minipage} & \begin{minipage}[t]{0.40\columnwidth}\centering
Pente S1, S2 \& SR\strut
\end{minipage} & \begin{minipage}[t]{0.08\columnwidth}\raggedright
{[}\protect\hyperlink{ref-egleston1997capnography}{49}{]}\strut
\end{minipage}\tabularnewline
\begin{minipage}[t]{0.09\columnwidth}\centering
evered\strut
\end{minipage} & \begin{minipage}[t]{0.05\columnwidth}\raggedright
2003\strut
\end{minipage} & \begin{minipage}[t]{0.08\columnwidth}\raggedright
12\strut
\end{minipage} & \begin{minipage}[t]{0.14\columnwidth}\centering
Spiro. \& Res.\strut
\end{minipage} & \begin{minipage}[t]{0.40\columnwidth}\centering
Pente III \& angle alpha\strut
\end{minipage} & \begin{minipage}[t]{0.08\columnwidth}\raggedright
{[}\protect\hyperlink{ref-evered2003can}{85}{]}\strut
\end{minipage}\tabularnewline
\begin{minipage}[t]{0.09\columnwidth}\centering
hisamuddin\strut
\end{minipage} & \begin{minipage}[t]{0.05\columnwidth}\raggedright
2009\strut
\end{minipage} & \begin{minipage}[t]{0.08\columnwidth}\raggedright
100\strut
\end{minipage} & \begin{minipage}[t]{0.14\columnwidth}\centering
DEP\strut
\end{minipage} & \begin{minipage}[t]{0.40\columnwidth}\centering
Pente II, III \& angle alpha\strut
\end{minipage} & \begin{minipage}[t]{0.08\columnwidth}\raggedright
{[}\protect\hyperlink{ref-hisamuddin2009correlations}{45}{]}\strut
\end{minipage}\tabularnewline
\begin{minipage}[t]{0.09\columnwidth}\centering
howe\strut
\end{minipage} & \begin{minipage}[t]{0.05\columnwidth}\raggedright
2011\strut
\end{minipage} & \begin{minipage}[t]{0.08\columnwidth}\raggedright
30\strut
\end{minipage} & \begin{minipage}[t]{0.14\columnwidth}\centering
DEP\strut
\end{minipage} & \begin{minipage}[t]{0.40\columnwidth}\centering
Pente II, III \& angle alpha\strut
\end{minipage} & \begin{minipage}[t]{0.08\columnwidth}\raggedright
{[}\protect\hyperlink{ref-howe2011use}{31}{]}\strut
\end{minipage}\tabularnewline
\begin{minipage}[t]{0.09\columnwidth}\centering
ioan\strut
\end{minipage} & \begin{minipage}[t]{0.05\columnwidth}\raggedright
2014\strut
\end{minipage} & \begin{minipage}[t]{0.08\columnwidth}\raggedright
8\strut
\end{minipage} & \begin{minipage}[t]{0.14\columnwidth}\centering
Res. \& Comp.\strut
\end{minipage} & \begin{minipage}[t]{0.40\columnwidth}\centering
Pente II, III, angle alpha, D1,D2\strut
\end{minipage} & \begin{minipage}[t]{0.08\columnwidth}\raggedright
{[}\protect\hyperlink{ref-ioan2014frequency}{86}{]}\strut
\end{minipage}\tabularnewline
\begin{minipage}[t]{0.09\columnwidth}\centering
kean\strut
\end{minipage} & \begin{minipage}[t]{0.05\columnwidth}\raggedright
2010\strut
\end{minipage} & \begin{minipage}[t]{0.08\columnwidth}\raggedright
34\strut
\end{minipage} & \begin{minipage}[t]{0.14\columnwidth}\centering
Spiro. \& clinique\strut
\end{minipage} & \begin{minipage}[t]{0.40\columnwidth}\centering
8 critères de You\strut
\end{minipage} & \begin{minipage}[t]{0.08\columnwidth}\raggedright
{[}\protect\hyperlink{ref-kean2010feature}{56}{]}\strut
\end{minipage}\tabularnewline
\begin{minipage}[t]{0.09\columnwidth}\centering
kline\strut
\end{minipage} & \begin{minipage}[t]{0.05\columnwidth}\raggedright
1998\strut
\end{minipage} & \begin{minipage}[t]{0.08\columnwidth}\raggedright
139\strut
\end{minipage} & \begin{minipage}[t]{0.14\columnwidth}\centering
Scinti \& TDM pulm\strut
\end{minipage} & \begin{minipage}[t]{0.40\columnwidth}\centering
Surface sous la courbe\strut
\end{minipage} & \begin{minipage}[t]{0.08\columnwidth}\raggedright
{[}\protect\hyperlink{ref-kline1998preliminary}{42}{]}\strut
\end{minipage}\tabularnewline
\begin{minipage}[t]{0.09\columnwidth}\centering
krauss\strut
\end{minipage} & \begin{minipage}[t]{0.05\columnwidth}\raggedright
2005\strut
\end{minipage} & \begin{minipage}[t]{0.08\columnwidth}\raggedright
262\strut
\end{minipage} & \begin{minipage}[t]{0.14\columnwidth}\centering
Spiro.\strut
\end{minipage} & \begin{minipage}[t]{0.40\columnwidth}\centering
Pente II, III, EtCO2, FR, Tps Insp \& expi.\strut
\end{minipage} & \begin{minipage}[t]{0.08\columnwidth}\raggedright
{[}\protect\hyperlink{ref-krauss2005capnogram}{50}{]}\strut
\end{minipage}\tabularnewline
\begin{minipage}[t]{0.09\columnwidth}\centering
lopez\strut
\end{minipage} & \begin{minipage}[t]{0.05\columnwidth}\raggedright
2011\strut
\end{minipage} & \begin{minipage}[t]{0.08\columnwidth}\raggedright
21\strut
\end{minipage} & \begin{minipage}[t]{0.14\columnwidth}\centering
Clinique\strut
\end{minipage} & \begin{minipage}[t]{0.40\columnwidth}\centering
Pente II, III, EtCO2, Tps Insp \& expi., PcCO2-EtCO2\strut
\end{minipage} & \begin{minipage}[t]{0.08\columnwidth}\raggedright
{[}\protect\hyperlink{ref-lopez2011capnography}{87}{]}\strut
\end{minipage}\tabularnewline
\begin{minipage}[t]{0.09\columnwidth}\centering
lukic\strut
\end{minipage} & \begin{minipage}[t]{0.05\columnwidth}\raggedright
2006\strut
\end{minipage} & \begin{minipage}[t]{0.08\columnwidth}\raggedright
5\strut
\end{minipage} & \begin{minipage}[t]{0.14\columnwidth}\centering
Clinique\strut
\end{minipage} & \begin{minipage}[t]{0.40\columnwidth}\centering
18 critères\strut
\end{minipage} & \begin{minipage}[t]{0.08\columnwidth}\raggedright
{[}\protect\hyperlink{ref-lukic2006novel}{88}{]}\strut
\end{minipage}\tabularnewline
\begin{minipage}[t]{0.09\columnwidth}\centering
mieloszyk\strut
\end{minipage} & \begin{minipage}[t]{0.05\columnwidth}\raggedright
2014\strut
\end{minipage} & \begin{minipage}[t]{0.08\columnwidth}\raggedright
139\strut
\end{minipage} & \begin{minipage}[t]{0.14\columnwidth}\centering
Clinique\strut
\end{minipage} & \begin{minipage}[t]{0.40\columnwidth}\centering
Pente III, EtCO2, Tps exp., Tps EtCO2\strut
\end{minipage} & \begin{minipage}[t]{0.08\columnwidth}\raggedright
{[}\protect\hyperlink{ref-mieloszyk2014automated}{61}{]}\strut
\end{minipage}\tabularnewline
\begin{minipage}[t]{0.09\columnwidth}\centering
mieloszyk\strut
\end{minipage} & \begin{minipage}[t]{0.05\columnwidth}\raggedright
2015\strut
\end{minipage} & \begin{minipage}[t]{0.08\columnwidth}\raggedright
56\strut
\end{minipage} & \begin{minipage}[t]{0.14\columnwidth}\centering
Cliniique\strut
\end{minipage} & \begin{minipage}[t]{0.40\columnwidth}\centering
EtCO2, pente S1, S2, S3 \& SR\strut
\end{minipage} & \begin{minipage}[t]{0.08\columnwidth}\raggedright
{[}\protect\hyperlink{ref-mieloszyk2015clustering}{89}{]}\strut
\end{minipage}\tabularnewline
\begin{minipage}[t]{0.09\columnwidth}\centering
yaron\strut
\end{minipage} & \begin{minipage}[t]{0.05\columnwidth}\raggedright
1996\strut
\end{minipage} & \begin{minipage}[t]{0.08\columnwidth}\raggedright
48\strut
\end{minipage} & \begin{minipage}[t]{0.14\columnwidth}\centering
DEP\strut
\end{minipage} & \begin{minipage}[t]{0.40\columnwidth}\centering
Pente III\strut
\end{minipage} & \begin{minipage}[t]{0.08\columnwidth}\raggedright
{[}\protect\hyperlink{ref-yaron1996utility}{48}{]}\strut
\end{minipage}\tabularnewline
\begin{minipage}[t]{0.09\columnwidth}\centering
you\strut
\end{minipage} & \begin{minipage}[t]{0.05\columnwidth}\raggedright
1994\strut
\end{minipage} & \begin{minipage}[t]{0.08\columnwidth}\raggedright
40\strut
\end{minipage} & \begin{minipage}[t]{0.14\columnwidth}\centering
Sprio.\strut
\end{minipage} & \begin{minipage}[t]{0.40\columnwidth}\centering
S1, S2, S3, SR, AR, SD1, SD2, SD3\strut
\end{minipage} & \begin{minipage}[t]{0.08\columnwidth}\raggedright
{[}\protect\hyperlink{ref-you1994expiratory}{51}{]}\strut
\end{minipage}\tabularnewline
\bottomrule
\end{longtable}

\end{landscape}

\newpage
\begin{landscape}

\begin{center}\rule{0.5\linewidth}{\linethickness}\end{center}

\hypertarget{synthese-des-etudes-sur-la-capnographie-standard-3}{%
\subsection{Synthèse des études sur la Capnographie
Standard}\label{synthese-des-etudes-sur-la-capnographie-standard-3}}

\begin{longtable}[]{@{}cllcc@{}}
\caption{Synthèse des études sur la Capnographie
Standard}\tabularnewline
\toprule
\begin{minipage}[b]{0.31\columnwidth}\centering
Auteur\strut
\end{minipage} & \begin{minipage}[b]{0.05\columnwidth}\raggedright
Année\strut
\end{minipage} & \begin{minipage}[b]{0.06\columnwidth}\raggedright
Effectif\strut
\end{minipage} & \begin{minipage}[b]{0.12\columnwidth}\centering
Gold standard\strut
\end{minipage} & \begin{minipage}[b]{0.32\columnwidth}\centering
Caractéristiques\strut
\end{minipage}\tabularnewline
\midrule
\endfirsthead
\toprule
\begin{minipage}[b]{0.31\columnwidth}\centering
Auteur\strut
\end{minipage} & \begin{minipage}[b]{0.05\columnwidth}\raggedright
Année\strut
\end{minipage} & \begin{minipage}[b]{0.06\columnwidth}\raggedright
Effectif\strut
\end{minipage} & \begin{minipage}[b]{0.12\columnwidth}\centering
Gold standard\strut
\end{minipage} & \begin{minipage}[b]{0.32\columnwidth}\centering
Caractéristiques\strut
\end{minipage}\tabularnewline
\midrule
\endhead
\begin{minipage}[t]{0.31\columnwidth}\centering
Babik et al.{[}\protect\hyperlink{ref-babik2012effects}{68}{]}\strut
\end{minipage} & \begin{minipage}[t]{0.05\columnwidth}\raggedright
2012\strut
\end{minipage} & \begin{minipage}[t]{0.06\columnwidth}\raggedright
68\strut
\end{minipage} & \begin{minipage}[t]{0.12\columnwidth}\centering
Res. \& Comp.\strut
\end{minipage} & \begin{minipage}[t]{0.32\columnwidth}\centering
Pente III\strut
\end{minipage}\tabularnewline
\begin{minipage}[t]{0.31\columnwidth}\centering
Blanch et
al.{[}\protect\hyperlink{ref-blanch1994relationship}{83}{]}\strut
\end{minipage} & \begin{minipage}[t]{0.05\columnwidth}\raggedright
1994\strut
\end{minipage} & \begin{minipage}[t]{0.06\columnwidth}\raggedright
41\strut
\end{minipage} & \begin{minipage}[t]{0.12\columnwidth}\centering
Résistance\strut
\end{minipage} & \begin{minipage}[t]{0.32\columnwidth}\centering
Pente III\strut
\end{minipage}\tabularnewline
\begin{minipage}[t]{0.31\columnwidth}\centering
Brown et al.{[}\protect\hyperlink{ref-brown2013forced}{84}{]}\strut
\end{minipage} & \begin{minipage}[t]{0.05\columnwidth}\raggedright
2013\strut
\end{minipage} & \begin{minipage}[t]{0.06\columnwidth}\raggedright
20\strut
\end{minipage} & \begin{minipage}[t]{0.12\columnwidth}\centering
Spiro. \& TDM\strut
\end{minipage} & \begin{minipage}[t]{0.32\columnwidth}\centering
Pente III\strut
\end{minipage}\tabularnewline
\begin{minipage}[t]{0.31\columnwidth}\centering
Egleston et
al.{[}\protect\hyperlink{ref-egleston1997capnography}{49}{]}\strut
\end{minipage} & \begin{minipage}[t]{0.05\columnwidth}\raggedright
1997\strut
\end{minipage} & \begin{minipage}[t]{0.06\columnwidth}\raggedright
38\strut
\end{minipage} & \begin{minipage}[t]{0.12\columnwidth}\centering
Clinique\strut
\end{minipage} & \begin{minipage}[t]{0.32\columnwidth}\centering
Pente S1, S2 \& SR\strut
\end{minipage}\tabularnewline
\begin{minipage}[t]{0.31\columnwidth}\centering
Evered et al.{[}\protect\hyperlink{ref-evered2003can}{85}{]}\strut
\end{minipage} & \begin{minipage}[t]{0.05\columnwidth}\raggedright
2003\strut
\end{minipage} & \begin{minipage}[t]{0.06\columnwidth}\raggedright
12\strut
\end{minipage} & \begin{minipage}[t]{0.12\columnwidth}\centering
Spiro. \& Res.\strut
\end{minipage} & \begin{minipage}[t]{0.32\columnwidth}\centering
Pente III \& angle alpha\strut
\end{minipage}\tabularnewline
\begin{minipage}[t]{0.31\columnwidth}\centering
Hisamuddin et
al.{[}\protect\hyperlink{ref-hisamuddin2009correlations}{45}{]}\strut
\end{minipage} & \begin{minipage}[t]{0.05\columnwidth}\raggedright
2009\strut
\end{minipage} & \begin{minipage}[t]{0.06\columnwidth}\raggedright
100\strut
\end{minipage} & \begin{minipage}[t]{0.12\columnwidth}\centering
DEP\strut
\end{minipage} & \begin{minipage}[t]{0.32\columnwidth}\centering
Pente II, III \& angle alpha\strut
\end{minipage}\tabularnewline
\begin{minipage}[t]{0.31\columnwidth}\centering
Howe et al.{[}\protect\hyperlink{ref-howe2011use}{31}{]}\strut
\end{minipage} & \begin{minipage}[t]{0.05\columnwidth}\raggedright
2011\strut
\end{minipage} & \begin{minipage}[t]{0.06\columnwidth}\raggedright
30\strut
\end{minipage} & \begin{minipage}[t]{0.12\columnwidth}\centering
DEP\strut
\end{minipage} & \begin{minipage}[t]{0.32\columnwidth}\centering
Pente II, III \& angle alpha\strut
\end{minipage}\tabularnewline
\begin{minipage}[t]{0.31\columnwidth}\centering
Ioan et al.{[}\protect\hyperlink{ref-ioan2014frequency}{86}{]}\strut
\end{minipage} & \begin{minipage}[t]{0.05\columnwidth}\raggedright
2014\strut
\end{minipage} & \begin{minipage}[t]{0.06\columnwidth}\raggedright
8\strut
\end{minipage} & \begin{minipage}[t]{0.12\columnwidth}\centering
Res. \& Comp.\strut
\end{minipage} & \begin{minipage}[t]{0.32\columnwidth}\centering
Pente II, III, angle alpha, D1,D2\strut
\end{minipage}\tabularnewline
\begin{minipage}[t]{0.31\columnwidth}\centering
Kean et al.{[}\protect\hyperlink{ref-kean2010feature}{56}{]}\strut
\end{minipage} & \begin{minipage}[t]{0.05\columnwidth}\raggedright
2010\strut
\end{minipage} & \begin{minipage}[t]{0.06\columnwidth}\raggedright
34\strut
\end{minipage} & \begin{minipage}[t]{0.12\columnwidth}\centering
Spiro. \& clinique\strut
\end{minipage} & \begin{minipage}[t]{0.32\columnwidth}\centering
8 critères de You\strut
\end{minipage}\tabularnewline
\begin{minipage}[t]{0.31\columnwidth}\centering
Kline et al.{[}\protect\hyperlink{ref-kline1998preliminary}{42}{]}\strut
\end{minipage} & \begin{minipage}[t]{0.05\columnwidth}\raggedright
1998\strut
\end{minipage} & \begin{minipage}[t]{0.06\columnwidth}\raggedright
139\strut
\end{minipage} & \begin{minipage}[t]{0.12\columnwidth}\centering
Scinti \& TDM pulm\strut
\end{minipage} & \begin{minipage}[t]{0.32\columnwidth}\centering
Surface sous la courbe\strut
\end{minipage}\tabularnewline
\begin{minipage}[t]{0.31\columnwidth}\centering
Krauss et al.{[}\protect\hyperlink{ref-krauss2005capnogram}{50}{]}\strut
\end{minipage} & \begin{minipage}[t]{0.05\columnwidth}\raggedright
2005\strut
\end{minipage} & \begin{minipage}[t]{0.06\columnwidth}\raggedright
262\strut
\end{minipage} & \begin{minipage}[t]{0.12\columnwidth}\centering
Spiro.\strut
\end{minipage} & \begin{minipage}[t]{0.32\columnwidth}\centering
Pente II, III, EtCO2, FR, Tps Insp \& expi.\strut
\end{minipage}\tabularnewline
\begin{minipage}[t]{0.31\columnwidth}\centering
Lopez et al.{[}\protect\hyperlink{ref-lopez2011capnography}{87}{]}\strut
\end{minipage} & \begin{minipage}[t]{0.05\columnwidth}\raggedright
2011\strut
\end{minipage} & \begin{minipage}[t]{0.06\columnwidth}\raggedright
21\strut
\end{minipage} & \begin{minipage}[t]{0.12\columnwidth}\centering
Clinique\strut
\end{minipage} & \begin{minipage}[t]{0.32\columnwidth}\centering
Pente II, III, EtCO2, Tps Insp \& expi., PcCO2-EtCO2\strut
\end{minipage}\tabularnewline
\begin{minipage}[t]{0.31\columnwidth}\centering
Lukic et al.{[}\protect\hyperlink{ref-lukic2006novel}{88}{]}\strut
\end{minipage} & \begin{minipage}[t]{0.05\columnwidth}\raggedright
2006\strut
\end{minipage} & \begin{minipage}[t]{0.06\columnwidth}\raggedright
5\strut
\end{minipage} & \begin{minipage}[t]{0.12\columnwidth}\centering
Clinique\strut
\end{minipage} & \begin{minipage}[t]{0.32\columnwidth}\centering
18 critères\strut
\end{minipage}\tabularnewline
\begin{minipage}[t]{0.31\columnwidth}\centering
Mieloszyk et
al.{[}\protect\hyperlink{ref-mieloszyk2014automated}{61}{]}\strut
\end{minipage} & \begin{minipage}[t]{0.05\columnwidth}\raggedright
2014\strut
\end{minipage} & \begin{minipage}[t]{0.06\columnwidth}\raggedright
139\strut
\end{minipage} & \begin{minipage}[t]{0.12\columnwidth}\centering
Clinique\strut
\end{minipage} & \begin{minipage}[t]{0.32\columnwidth}\centering
Pente III, EtCO2, Tps exp., Tps EtCO2\strut
\end{minipage}\tabularnewline
\begin{minipage}[t]{0.31\columnwidth}\centering
Mieloszyk et
al.{[}\protect\hyperlink{ref-mieloszyk2015clustering}{89}{]}\strut
\end{minipage} & \begin{minipage}[t]{0.05\columnwidth}\raggedright
2015\strut
\end{minipage} & \begin{minipage}[t]{0.06\columnwidth}\raggedright
56\strut
\end{minipage} & \begin{minipage}[t]{0.12\columnwidth}\centering
Cliniique\strut
\end{minipage} & \begin{minipage}[t]{0.32\columnwidth}\centering
EtCO2, pente S1, S2, S3 \& SR\strut
\end{minipage}\tabularnewline
\begin{minipage}[t]{0.31\columnwidth}\centering
Yaron et al.{[}\protect\hyperlink{ref-yaron1996utility}{48}{]}\strut
\end{minipage} & \begin{minipage}[t]{0.05\columnwidth}\raggedright
1996\strut
\end{minipage} & \begin{minipage}[t]{0.06\columnwidth}\raggedright
48\strut
\end{minipage} & \begin{minipage}[t]{0.12\columnwidth}\centering
DEP\strut
\end{minipage} & \begin{minipage}[t]{0.32\columnwidth}\centering
Pente III\strut
\end{minipage}\tabularnewline
\begin{minipage}[t]{0.31\columnwidth}\centering
You et al.{[}\protect\hyperlink{ref-you1994expiratory}{51}{]}\strut
\end{minipage} & \begin{minipage}[t]{0.05\columnwidth}\raggedright
1994\strut
\end{minipage} & \begin{minipage}[t]{0.06\columnwidth}\raggedright
40\strut
\end{minipage} & \begin{minipage}[t]{0.12\columnwidth}\centering
Sprio.\strut
\end{minipage} & \begin{minipage}[t]{0.32\columnwidth}\centering
S1, S2, S3, SR, AR, SD1, SD2, SD3\strut
\end{minipage}\tabularnewline
\bottomrule
\end{longtable}

\end{landscape}

\newpage

\begin{center}\rule{0.5\linewidth}{\linethickness}\end{center}

\hypertarget{references}{%
\section*{References}\label{references}}
\addcontentsline{toc}{section}{References}

\hypertarget{refs}{}
\leavevmode\hypertarget{ref-hogg2013small}{}%
{[}1{]} J. C. Hogg, J. E. McDonough, and M. Suzuki, ``Small airway
obstruction in copd: New insights based on micro-ct imaging and mri
imaging,'' \emph{Chest}, vol. 143, no. 5, pp. 1436--1443, 2013.

\leavevmode\hypertarget{ref-busse2000pathophysiology}{}%
{[}2{]} W. W. Busse, S. Banks-Schlegel, and S. E. Wenzel,
``Pathophysiology of severe asthma,'' \emph{Journal of Allergy and
Clinical Immunology}, vol. 106, no. 6, pp. 1033--1042, 2000.

\leavevmode\hypertarget{ref-tillie2004physiopathologie}{}%
{[}3{]} I. Tillie-Leblond, C. Iliescu, and A. Deschildre,
``Physiopathologie de la réaction inflammatoire dans l'asthme,''
\emph{Archives de pediatrie}, vol. 11, pp. S58--S64, 2004.

\leavevmode\hypertarget{ref-west1988physiopathologie}{}%
{[}4{]} J. B. West, \emph{Physiopathologie respiratoire}. Pradel, 1988.

\leavevmode\hypertarget{ref-plantier2016mechanisms}{}%
{[}5{]} L. Plantier, A. Pradel, and C. Delclaux, ``Mechanisms of
non-specific airway hyperresponsiveness: Methacholine-induced
alterations in airway architecture,'' \emph{Revue des maladies
respiratoires}, vol. 33, no. 8, pp. 735--743, 2016.

\leavevmode\hypertarget{ref-hogg2004pathophysiology}{}%
{[}6{]} J. C. Hogg, ``Pathophysiology of airflow limitation in chronic
obstructive pulmonary disease,'' \emph{The Lancet}, vol. 364, no. 9435,
pp. 709--721, 2004.

\leavevmode\hypertarget{ref-wagner1977ventilation}{}%
{[}7{]} P. Wagner, D. Dantzker, R. Dueck, J. Clausen, and J. West,
``Ventilation-perfusion inequality in chronic obstructive pulmonary
disease.'' \emph{The Journal of clinical investigation}, vol. 59, no. 2,
pp. 203--216, 1977.

\leavevmode\hypertarget{ref-mcdonough2011small}{}%
{[}8{]} J. E. McDonough \emph{et al.}, ``Small-airway obstruction and
emphysema in chronic obstructive pulmonary disease,'' \emph{New England
Journal of Medicine}, vol. 365, no. 17, pp. 1567--1575, 2011.

\leavevmode\hypertarget{ref-reid1958secondary}{}%
{[}9{]} L. Reid, ``The secondary lobule in the adult human lung, with
special reference to its appearance in bronchograms,'' \emph{Thorax},
vol. 13, no. 2, p. 110, 1958.

\leavevmode\hypertarget{ref-vogelmeier2017global}{}%
{[}10{]} C. F. Vogelmeier \emph{et al.}, ``Global strategy for the
diagnosis, management, and prevention of chronic obstructive lung
disease 2017 report. GOLD executive summary,'' \emph{American journal of
respiratory and critical care medicine}, vol. 195, no. 5, pp. 557--582,
2017.

\leavevmode\hypertarget{ref-abid2015model}{}%
{[}11{]} A. Abid, R. J. Mieloszyk, G. C. Verghese, B. S. Krauss, and T.
Heldt, ``Model-based estimation of pulmonary compliance and resistance
parameters from time-based capnography,'' in \emph{Engineering in
medicine and biology society (embc), 2015 37th annual international
conference of the ieee}, 2015, pp. 1687--1690.

\leavevmode\hypertarget{ref-abid2017model}{}%
{[}12{]} A. Abid, R. J. Mieloszyk, G. C. Verghese, B. S. Krauss, and T.
Heldt, ``Model-based estimation of respiratory parameters from
capnography, with application to diagnosing obstructive lung disease,''
\emph{IEEE Transactions on Biomedical Engineering}, vol. 64, no. 12, pp.
2957--2967, 2017.

\leavevmode\hypertarget{ref-roy2007calculating}{}%
{[}13{]} T. K. Roy, J. O. Den Buijs, and L. Warner, ``Calculating the
effect of altered respiratory parameters on capnographic indices,'' in
\emph{World congress on medical physics and biomedical engineering
2006}, 2007, pp. 123--126.

\leavevmode\hypertarget{ref-wagner1978ventilation}{}%
{[}14{]} P. Wagner, D. Dantzker, V. Iacovoni, W. Tomlin, and J. West,
``Ventilation-perfusion inequality in asymptomatic asthma,''
\emph{American Review of Respiratory Disease}, vol. 118, no. 3, pp.
511--524, 1978.

\leavevmode\hypertarget{ref-ballester1989ventilation}{}%
{[}15{]} E. Ballester, A. Reyes, J. Roca, R. Guitart, P. Wagner, and R.
Rodriguez-Roisin, ``Ventilation-perfusion mismatching in acute severe
asthma: Effects of salbutamol and 100\% oxygen.'' \emph{Thorax}, vol.
44, no. 4, pp. 258--267, 1989.

\leavevmode\hypertarget{ref-wagner1996gas}{}%
{[}16{]} P. Wagner, G. Hedenstierna, and R. Rodriguez-Roisin, ``Gas
exchange, expiratory flow obstruction and the clinical spectrum of
asthma,'' \emph{European Respiratory Journal}, vol. 9, no. 6, pp.
1278--1282, 1996.

\leavevmode\hypertarget{ref-luft1943neue}{}%
{[}17{]} K. Luft, ``Über eine neue methode der registrierenden
gasanalyse mit hilfe der absorption ultraroter strahlen ohne spektrale
zerlegung,'' \emph{Z. tech. Phys}, vol. 24, pp. 97--104, 1943.

\leavevmode\hypertarget{ref-bhavani1992capnometry}{}%
{[}18{]} K. Bhavani-Shankar, H. Moseley, A. Kumar, and Y. Delph,
``Capnometry and anaesthesia,'' \emph{Canadian Journal of anaesthesia},
vol. 39, no. 6, pp. 617--632, 1992.

\leavevmode\hypertarget{ref-aitken1928fluctuation}{}%
{[}19{]} R. Aitken and A. Clark-Kennedy, ``On the fluctuation in the
composition of the alveolar air during the respiratory cycle in muscular
exercise,'' \emph{The Journal of physiology}, vol. 65, no. 4, pp.
389--411, 1928.

\leavevmode\hypertarget{ref-chilton1952mathematical}{}%
{[}20{]} A. B. Chilton and R. W. Stacy, ``A mathematical analysis of
carbon dioxide respiration in man,'' \emph{The bulletin of mathematical
biophysics}, vol. 14, no. 1, pp. 1--18, 1952.

\leavevmode\hypertarget{ref-yamamoto1960mathematical}{}%
{[}21{]} W. S. Yamamoto, ``Mathematical analysis of the time course of
alveolar co2,'' \emph{Journal of applied physiology}, vol. 15, no. 2,
pp. 215--219, 1960.

\leavevmode\hypertarget{ref-jaffe2008infrared}{}%
{[}22{]} M. B. Jaffe, ``Infrared measurement of carbon dioxide in the
human breath:`Breathe-through' devices from tyndall to the present
day,'' \emph{Anesthesia \& Analgesia}, vol. 107, no. 3, pp. 890--904,
2008.

\leavevmode\hypertarget{ref-bellville1959respiratory}{}%
{[}23{]} J. W. Bellville and J. Seed, ``Respiratory carbon dioxide
response curve computer: It gives more complete alveolar
ventilation-pcoco2 response curves than could formerly be obtained,''
\emph{Science}, vol. 130, no. 3382, pp. 1079--1083, 1959.

\leavevmode\hypertarget{ref-berengo1961single}{}%
{[}24{]} A. Berengo and A. Cutillo, ``Single-breath analysis of carbon
dioxide concentration records,'' \emph{Journal of Applied Physiology},
vol. 16, no. 3, pp. 522--530, 1961.

\leavevmode\hypertarget{ref-murphy1899analogue}{}%
{[}25{]} T. Murphy, ``Analogue-digital data processing of respiratory
parameters,'' in \emph{Afips}, 1899, p. 253.

\leavevmode\hypertarget{ref-noe1963computer}{}%
{[}26{]} F. Noe, ``Computer analysis of curves from an infrared co2
analyzer and screen-type airflow meter,'' \emph{Journal of Applied
Physiology}, vol. 18, no. 1, pp. 149--157, 1963.

\leavevmode\hypertarget{ref-bao1992expert}{}%
{[}27{]} W. Bao, P. King, J. Zheng, and B. Smith, ``Expert capnogram
analysis,'' \emph{IEEE Engineering in Medicine and Biology Magazine},
vol. 11, no. 1, pp. 62--66, 1992.

\leavevmode\hypertarget{ref-ventzas1994capnex}{}%
{[}28{]} D. Ventzas, ``CAPNEX: An expert system for capnography (co2
respiration analysis),'' \emph{Transactions of the Institute of
Measurement and Control}, vol. 16, no. 5, pp. 233--244, 1994.

\leavevmode\hypertarget{ref-yosefy2004end}{}%
{[}29{]} C. Yosefy, E. Hay, Y. Nasri, E. Magen, and L. Reisin, ``End
tidal carbon dioxide as a predictor of the arterial pco2 in the
emergency department setting,'' \emph{Emergency Medicine Journal}, vol.
21, no. 5, pp. 557--559, 2004.

\leavevmode\hypertarget{ref-jabre2010place}{}%
{[}30{]} P. Jabre, X. Combes, and F. Adnet, ``Place de la surveillance
de la capnographie dans les détresses respiratoires aiguës,''
\emph{Réanimation}, vol. 19, no. 7, pp. 633--639, 2010.

\leavevmode\hypertarget{ref-howe2011use}{}%
{[}31{]} T. A. Howe, K. Jaalam, R. Ahmad, C. K. Sheng, and N. H. N. Ab
Rahman, ``The use of end-tidal capnography to monitor non-intubated
patients presenting with acute exacerbation of asthma in the emergency
department,'' \emph{The Journal of emergency medicine}, vol. 41, no. 6,
pp. 581--589, 2011.

\leavevmode\hypertarget{ref-rayburn1994neural}{}%
{[}32{]} D. B. Rayburn, T. M. Fitzpatrick, and S. Van Albert, ``Neural
network evaluation of slopes from sequential volume segments of
expiratory carbon dioxide curves,'' in \emph{Neural networks, 1994. IEEE
world congress on computational intelligence., 1994 ieee international
conference on}, 1994, vol. 6, pp. 3530--3533.

\leavevmode\hypertarget{ref-tusman2011capnography}{}%
{[}33{]} G. Tusman, F. SUAREZ-SIPMANN, S. Bohm, J. Borges, and G.
Hedenstierna, ``Capnography reflects ventilation/perfusion distribution
in a model of acute lung injury,'' \emph{Acta anaesthesiologica
Scandinavica}, vol. 55, no. 5, pp. 597--606, 2011.

\leavevmode\hypertarget{ref-fletcher1981concept}{}%
{[}34{]} R. Fletcher, B. Jonson, G. Cumming, and J. Brew, ``The concept
of deadspace with special reference to the single breath test for carbon
dioxide,'' \emph{British journal of anaesthesia}, vol. 53, no. 1, pp.
77--88, 1981.

\leavevmode\hypertarget{ref-fowler1948lung}{}%
{[}35{]} W. S. Fowler, ``Lung function studies. II. The respiratory dead
space,'' \emph{American Journal of Physiology-Legacy Content}, vol. 154,
no. 3, pp. 405--416, 1948.

\leavevmode\hypertarget{ref-brewer2008anatomic}{}%
{[}36{]} L. M. Brewer, J. A. Orr, and N. L. Pace, ``Anatomic dead space
cannot be predicted by body weight,'' \emph{Respiratory care}, vol. 53,
no. 7, pp. 885--891, 2008.

\leavevmode\hypertarget{ref-tang2007systematic}{}%
{[}37{]} Y. Tang, M. Turner, and A. Baker, ``Systematic errors and
susceptibility to noise of four methods for calculating anatomical dead
space from the co2 expirogram,'' \emph{British journal of anaesthesia},
vol. 98, no. 6, pp. 828--834, 2007.

\leavevmode\hypertarget{ref-suarez2013corrections}{}%
{[}38{]} F. Suarez-Sipmann, A. Santos, S. H. Böhm, J. B. Borges, G.
Hedenstierna, and G. Tusman, ``Corrections of enghoff's dead space
formula for shunt effects still overestimate bohr's dead space,''
\emph{Respiratory physiology \& neurobiology}, vol. 189, no. 1, pp.
99--105, 2013.

\leavevmode\hypertarget{ref-tusman2011validation}{}%
{[}39{]} G. Tusman, F. S. Sipmann, J. B. Borges, G. Hedenstierna, and S.
H. Bohm, ``Validation of bohr dead space measured by volumetric
capnography,'' \emph{Intensive care medicine}, vol. 37, no. 5, pp.
870--874, 2011.

\leavevmode\hypertarget{ref-tusman2009model}{}%
{[}40{]} G. Tusman, A. Scandurra, S. H. Böhm, F. Suarez-Sipmann, and F.
Clara, ``Model fitting of volumetric capnograms improves calculations of
airway dead space and slope of phase iii,'' \emph{Journal of clinical
monitoring and computing}, vol. 23, no. 4, pp. 197--206, 2009.

\leavevmode\hypertarget{ref-jung2008modalites}{}%
{[}41{]} B. Jung, G. Chanques, M. Sebbane, D. Verzilli, and S. Jaber,
``Les modalités de l'intubation en urgence et ses complications,''
\emph{Réanimation}, vol. 17, no. 8, pp. 753--760, 2008.

\leavevmode\hypertarget{ref-kline1998preliminary}{}%
{[}42{]} J. A. Kline and M. Arunachlam, ``Preliminary study of the
capnogram waveform area to screen for pulmonary embolism,'' \emph{Annals
of emergency medicine}, vol. 32, no. 3, pp. 289--296, 1998.

\leavevmode\hypertarget{ref-wiegand2000effectiveness}{}%
{[}43{]} U. K. Wiegand, V. Kurowski, E. Giannitsis, H. A. Katus, and H.
Djonlagic, ``Effectiveness of end-tidal carbon dioxide tension for
monitoring of thrombolytic therapy in acute pulmonary embolism,''
\emph{Critical care medicine}, vol. 28, no. 11, pp. 3588--3592, 2000.

\leavevmode\hypertarget{ref-ozier2011pivotal}{}%
{[}44{]} A. Ozier \emph{et al.}, ``The pivotal role of airway smooth
muscle in asthma pathophysiology,'' \emph{Journal of allergy}, vol.
2011, 2011.

\leavevmode\hypertarget{ref-hisamuddin2009correlations}{}%
{[}45{]} N. N. Hisamuddin, A. Rashidi, K. Chew, J. Kamaruddin, Z.
Idzwan, and A. Teo, ``Correlations between capnographic waveforms and
peak flow meter measurement in emergency department management of
asthma,'' \emph{International journal of emergency medicine}, vol. 2,
no. 2, pp. 83--89, 2009.

\leavevmode\hypertarget{ref-langhan2008quantitative}{}%
{[}46{]} M. L. Langhan, M. R. Zonfrillo, and D. M. Spiro, ``Quantitative
end-tidal carbon dioxide in acute exacerbations of asthma,'' \emph{The
Journal of pediatrics}, vol. 152, no. 6, pp. 829--832, 2008.

\leavevmode\hypertarget{ref-den2006bayesian}{}%
{[}47{]} J. O. Den Buijs, L. Warner, N. W. Chbat, and T. K. Roy,
``Bayesian tracking of a nonlinear model of the capnogram,'' in
\emph{Engineering in medicine and biology society, 2006. EMBS'06. 28th
annual international conference of the ieee}, 2006, pp. 2871--2874.

\leavevmode\hypertarget{ref-yaron1996utility}{}%
{[}48{]} M. Yaron, P. Padyk, M. Hutsinpiller, and C. B. Cairns,
``Utility of the expiratory capnogram in the assessment of
bronchospasm,'' \emph{Annals of emergency medicine}, vol. 28, no. 4, pp.
403--407, 1996.

\leavevmode\hypertarget{ref-egleston1997capnography}{}%
{[}49{]} C. Egleston, H. B. Aslam, and M. Lambert, ``Capnography for
monitoring non-intubated spontaneously breathing patients in an
emergency room setting.'' \emph{Emergency Medicine Journal}, vol. 14,
no. 4, pp. 222--224, 1997.

\leavevmode\hypertarget{ref-krauss2005capnogram}{}%
{[}50{]} B. Krauss \emph{et al.}, ``Capnogram shape in obstructive lung
disease,'' \emph{Anesthesia \& Analgesia}, vol. 100, no. 3, pp.
884--888, 2005.

\leavevmode\hypertarget{ref-you1994expiratory}{}%
{[}51{]} B. You, R. Peslin, C. Duvivier, V. D. Vu, and J. Grilliat,
``Expiratory capnography in asthma: Evaluation of various shape
indices,'' \emph{European Respiratory Journal}, vol. 7, no. 2, pp.
318--323, 1994.

\leavevmode\hypertarget{ref-smalhout1975atlas}{}%
{[}52{]} B. Smalhout and Z. Kalenda, \emph{An atlas of capnography}.
Institute of Anaesthesiology, University Hospital Utrecht, 1975.

\leavevmode\hypertarget{ref-kelsey1962expiratory}{}%
{[}53{]} J. Kelsey, E. Oldham, and S. Horvath, ``Expiratory carbon
dioxide concentration curve a test of pulmonary function,''
\emph{Diseases of the Chest}, vol. 41, no. 5, pp. 498--503, 1962.

\leavevmode\hypertarget{ref-dubois1952alveolar}{}%
{[}54{]} A. DuBois, R. Fowler, A. Soffer, and W. Fenn, ``Alveolar co2
measured by expiration into the rapid infrared gas analyzer,''
\emph{Journal of applied physiology}, vol. 4, no. 7, pp. 526--534, 1952.

\leavevmode\hypertarget{ref-betancourt2014segmented}{}%
{[}55{]} J. P. Betancourt \emph{et al.}, ``Segmented wavelet
decomposition for capnogram feature extraction in asthma
classification,'' \emph{Journal of Advanced Computational Intelligence
and Intelligent Informatics}, vol. 18, no. 4, pp. 480--488, 2014.

\leavevmode\hypertarget{ref-kean2010feature}{}%
{[}56{]} T. T. Kean, A. Teo, and M. Malarvili, ``Feature extraction of
capnogram for asthmatic patient,'' in \emph{Computer engineering and
applications (iccea), 2010 second international conference on}, 2010,
vol. 2, pp. 251--255.

\leavevmode\hypertarget{ref-hjorth1970eeg}{}%
{[}57{]} B. Hjorth, ``EEG analysis based on time domain properties,''
\emph{Electroencephalography and clinical neurophysiology}, vol. 29, no.
3, pp. 306--310, 1970.

\leavevmode\hypertarget{ref-kazemi2012investigation}{}%
{[}58{]} M. Kazemi and M. Malarvili, ``Investigation of capnogram signal
characteristics using statistical methods,'' \emph{IEEE Transactions on
Biomedical Engineering}, pp. 343--348, 2012.

\leavevmode\hypertarget{ref-kazemi2013frequency}{}%
{[}59{]} M. Kazemi, M. B. Krishnan, and T. A. Howe, ``Frequency analysis
of capnogram signals to differentiate asthmatic and non-asthmatic
conditions using radial basis function neural networks.'' \emph{Iranian
Journal of Allergy, Asthma and Immunology}, vol. 12, no. 3, pp.
236--246, 2013.

\leavevmode\hypertarget{ref-kazemi2016new}{}%
{[}60{]} M. Kazemi and A. H. Teo, ``New prognostic index to detect the
severity of asthma automatically using signal processing techniques of
capnogram,'' \emph{Journal of Intelligent Procedures in Electrical
Technology}, vol. 7, no. 26, 2016.

\leavevmode\hypertarget{ref-mieloszyk2014automated}{}%
{[}61{]} R. J. Mieloszyk \emph{et al.}, ``Automated quantitative
analysis of capnogram shape for copd--normal and copd--chf
classification,'' \emph{IEEE Transactions on Biomedical Engineering},
vol. 61, no. 12, pp. 2882--2890, 2014.

\leavevmode\hypertarget{ref-bleil2009online}{}%
{[}62{]} M. Bleil, A. Opp, R. Linder, S. Boye, H. Gehring, and U. G.
Hofmann, ``Online-classification of capnographic curves using artificial
neural networks,'' in \emph{4th european conference of the international
federation for medical and biological engineering}, 2009, pp.
1096--1099.

\leavevmode\hypertarget{ref-landis1998scoring}{}%
{[}63{]} B. Landis and P. M. Romano, ``A scoring system for capnogram
biofeedback: Preliminary findings,'' \emph{Applied psychophysiology and
biofeedback}, vol. 23, no. 2, pp. 75--91, 1998.

\leavevmode\hypertarget{ref-smith1994recognition}{}%
{[}64{]} T. C. Smith, A. Green, and P. Hutton, ``Recognition of
cardiogenic artifact in pediatric capnograms,'' \emph{Journal of
clinical monitoring}, vol. 10, no. 4, pp. 270--275, 1994.

\leavevmode\hypertarget{ref-leturiondo2018influence}{}%
{[}65{]} M. Leturiondo \emph{et al.}, ``Influence of chest compression
artefact on capnogram-based ventilation detection during out-of-hospital
cardiopulmonary resuscitation,'' \emph{Resuscitation}, vol. 124, pp.
63--68, 2018.

\leavevmode\hypertarget{ref-gutierrez2018enhancing}{}%
{[}66{]} J. J. Gutiérrez \emph{et al.}, ``Enhancing ventilation
detection during cardiopulmonary resuscitation by filtering chest
compression artifact from the capnography waveform,'' \emph{PloS one},
vol. 13, no. 8, p. e0201565, 2018.

\leavevmode\hypertarget{ref-singh2018automatic}{}%
{[}67{]} O. P. Singh, R. Palaniappan, and M. Malarvili, ``Automatic
quantitative analysis of human respired carbon dioxide waveform for
asthma and non-asthma classification using support vector machine,''
\emph{IEEE Access}, vol. 6, pp. 55245--55256, 2018.

\leavevmode\hypertarget{ref-babik2012effects}{}%
{[}68{]} B. Babik, Z. Csorba, D. Czövek, P. N. Mayr, G. Bogáts, and F.
Peták, ``Effects of respiratory mechanics on the capnogram phases:
Importance of dynamic compliance of the respiratory system,''
\emph{Critical care}, vol. 16, no. 5, p. R177, 2012.

\leavevmode\hypertarget{ref-nassar2016capnography}{}%
{[}69{]} B. S. Nassar and G. A. Schmidt, ``Capnography during critical
illness,'' \emph{Chest}, vol. 149, no. 2, pp. 576--585, 2016.

\leavevmode\hypertarget{ref-mcswain2010end}{}%
{[}70{]} S. D. McSwain \emph{et al.}, ``End-tidal and arterial carbon
dioxide measurements correlate across all levels of physiologic dead
space,'' \emph{Respiratory care}, vol. 55, no. 3, pp. 288--293, 2010.

\leavevmode\hypertarget{ref-lujan2008capnometry}{}%
{[}71{]} M. Lujan, E. Canturri, A. Moreno, M. Arranz, L. Vigil, and C.
Domingo, ``Capnometry in spontaneously breathing patients: The influence
of chronic obstructive pulmonary disease and expiration maneuvers,''
\emph{Medical Science Monitor}, vol. 14, no. 9, pp. CR485--CR492, 2008.

\leavevmode\hypertarget{ref-tusman2005effect}{}%
{[}72{]} G. Tusman \emph{et al.}, ``Effect of pulmonary perfusion on the
slopes of single-breath test of co2,'' \emph{Journal of Applied
Physiology}, vol. 99, no. 2, pp. 650--655, 2005.

\leavevmode\hypertarget{ref-delmas2011readmissions}{}%
{[}73{]} M.-C. Delmas, C. Marguet, C. Raherison, J. Nicolau, and C.
Fuhrman, ``Readmissions for asthma in france in 2002--2005,''
\emph{Revue des maladies respiratoires}, vol. 28, no. 9, pp. e115--e122,
2011.

\leavevmode\hypertarget{ref-delmas2010asthma}{}%
{[}74{]} M. Delmas and C. Fuhrman, ``Asthma in france: A review of
descriptive epidemiological data,'' \emph{Revue des maladies
respiratoires}, vol. 27, no. 2, pp. 151--159, 2010.

\leavevmode\hypertarget{ref-delmas2017augmentation}{}%
{[}75{]} M.-C. Delmas, N. Guignon, B. Leynaert, M. Moisy, C. Marguet,
and C. Fuhrman, ``Augmentation de la prévalence de l'asthme chez le
jeune enfant en france,'' \emph{Revue des Maladies Respiratoires}, vol.
34, no. 5, pp. 525--534, 2017.

\leavevmode\hypertarget{ref-tattersfield2001asthma}{}%
{[}76{]} A. TATTERSFIELD \emph{et al.}, ``Asthma severity and adequacy
of management in accident and emergency departments in france: A
prospective study. Commentary,'' \emph{Lancet}, vol. 358, no. 9282, pp.
629--635, 2001.

\leavevmode\hypertarget{ref-hart2002asthma}{}%
{[}77{]} S. P. Hart and S. Ebihara, ``Asthma severity and adequacy of
management,'' \emph{Lancet}, vol. 359, no. 9300, p. 75, 2002.

\leavevmode\hypertarget{ref-salmeron2001asthma}{}%
{[}78{]} S. Salmeron \emph{et al.}, ``Asthma severity and adequacy of
management in accident and emergency departments in france: A
prospective study,'' \emph{The Lancet}, vol. 358, no. 9282, pp.
629--635, 2001.

\leavevmode\hypertarget{ref-anto2001epidemiology}{}%
{[}79{]} J. Anto, P. Vermeire, J. Vestbo, and J. Sunyer, ``Epidemiology
of chronic obstructive pulmonary disease,'' \emph{European Respiratory
Journal}, vol. 17, no. 5, pp. 982--994, 2001.

\leavevmode\hypertarget{ref-charvanne2008bronchopneumopathie}{}%
{[}80{]} M. Charvanne and F. Roos, ``Bronchopneumopathie chronique
obstructive (bpco) et travail,'' \emph{Âge}, vol. 25, no. 50, p. 75,
2008.

\leavevmode\hypertarget{ref-roche2007donnees}{}%
{[}81{]} N. Roche, M. Zureik, A. Vergnenègre, H. HUCHON, and F.
Neukirch, ``Données récentes sur la prévalence de la bronchopneumopathie
chronique obstructive en france,'' \emph{Bull Epidémiol Hebd}, pp.
27--28, 2007.

\leavevmode\hypertarget{ref-fuhrman2007mortalite}{}%
{[}82{]} C. Fuhrman, M.-C. Delmas, J. Nicolau, and E. Jougla,
``Mortalité liée à la bpco en france métropolitaine, 1979-2003,''
\emph{BEH thématique}, vol. 27, pp. 242--244, 2007.

\leavevmode\hypertarget{ref-blanch1994relationship}{}%
{[}83{]} L. Blanch, R. Fernandez, P. Saura, F. Baigorri, and A. Artigas,
``Relationship between expired capnogram and respiratory system
resistance in critically iii patients during total ventilatory
support,'' \emph{Chest}, vol. 105, no. 1, pp. 219--223, 1994.

\leavevmode\hypertarget{ref-brown2013forced}{}%
{[}84{]} R. H. Brown \emph{et al.}, ``Forced expiratory capnography and
chronic obstructive pulmonary disease (copd),'' \emph{Journal of breath
research}, vol. 7, no. 1, p. 017108, 2013.

\leavevmode\hypertarget{ref-evered2003can}{}%
{[}85{]} L. Evered, F. Ducharme, G. M. Davis, and M. Pusic, ``Can we
assess asthma severity using expiratory capnography in a pediatric
emergency department?'' \emph{Canadian Journal of Emergency Medicine},
vol. 5, no. 3, pp. 169--170, 2003.

\leavevmode\hypertarget{ref-ioan2014frequency}{}%
{[}86{]} I. Ioan \emph{et al.}, ``Frequency dependence of capnography in
anesthetized rabbits,'' \emph{Respiratory physiology \& neurobiology},
vol. 190, pp. 14--19, 2014.

\leavevmode\hypertarget{ref-lopez2011capnography}{}%
{[}87{]} E. Lopez, J. Mathlouthi, S. Lescure, B. Krauss, P.-H. Jarreau,
and G. Moriette, ``Capnography in spontaneously breathing preterm
infants with bronchopulmonary dysplasia,'' \emph{Pediatric pulmonology},
vol. 46, no. 9, pp. 896--902, 2011.

\leavevmode\hypertarget{ref-lukic2006novel}{}%
{[}88{]} K. Z. Lukic, B. Urch, M. Fila, M. E. Faughnan, and F.
Silverman, ``A novel application of capnography during controlled human
exposure to air pollution,'' \emph{Biomedical engineering online}, vol.
5, no. 1, p. 54, 2006.

\leavevmode\hypertarget{ref-mieloszyk2015clustering}{}%
{[}89{]} R. J. Mieloszyk, M. G. Guo, G. C. Verghese, G. Andolfatto, T.
Heldt, and B. S. Krauss, ``Clustering of capnogram features to track
state transitions during procedural sedation,'' in \emph{Engineering in
medicine and biology society (embc), 2015 37th annual international
conference of the ieee}, 2015, pp. 1699--1702.


\end{document}
